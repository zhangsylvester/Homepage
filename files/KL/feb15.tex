\begin{proof}[Proof of uniqueness]
We will first prove the uniqueness part of \Cref{thm:KL_basis}. Denote $\alpha(x,w)=\varepsilon_w\varepsilon_x q_w^{1\over 2}q_x^{-1}$. Assume%
\begin{enumerate}[(a),noitemsep]
	\item $\eta(C_w)=C_w$
	\item $P_{w,w}=1$
	\item $P_{x,w}(q)\in\mathbb{Z}[q]$ of degree $\leq {1\over 2}(\ell(x,w)-1)$ for $x\leq w$.
\end{enumerate}

where $\ell(x,w)=\ell(w)-\ell(x)$. We will employ induction on $\ell(x,w)$.

\underline{Base Case.} $x=w$ so that $P_{x,w}=1$.

\underline{Induction Step.}	Assume unique for $x<y\leq w$
\[C_w=\sum_{y\leq w}\alpha(y,w)\bar P_{y,w} T_y.\]
Apply $\eta$ we get
\begin{align*}\label{eq:uniq_proof}
C_w&=\sum_{y\leq w}\varepsilon_w\varepsilon_y q^{-{1\over 2} }q_yP_{y,w}(T_{y^{-1}})^{-1}\\&=\sum_{y\leq w} \varepsilon_w\varepsilon_y q^{-{1\over 2} } q_yP_{y,w}\varepsilon_yq_y^{-1}\sum_{x\leq y}\varepsilon_x R_{x,y} T_x\\
&=\varepsilon_xq_w^{-{1\over 2}}\sum_{x\leq y\leq w}\varepsilon_x R_{x,y} P_{y,w} T_x
\end{align*}
Comparing coefficients of $T_x$, we have
\begin{align}
\nonumber\varepsilon_w\varepsilon_xq_w^{1\over 2}q^{-1}\bar P_{x,w}&=\varepsilon_x q_w^{-{1\over 2}}\sum_{x\leq y\leq w}\varepsilon_x R_{x,y} P_{y,w}\\
	q_w^{1\over 2} q_x^{-{1\over 2}}\bar P_{x,w}-q_w^{-{1\over 2}} q_x^{1\over 2} P_{x,w}&=q_w^{-{1\over 2}}q_x^{1\over 2}\sum_{x<y\leq w}R_{x,y}P_{y,w}
\end{align}
By induction hypothesis the right hand side is unique. It can be seen that, assumption (c) implies that
\begin{itemize}
	\item $q_w{1\over 2}q_x^{-{1\over 2}}\bar P_{x,w}$ is a polynomial in $q^{1\over 2}$ without constant term.
	\item $q_w^{-{1\over 2}}q_x^{{1\over 2}} P_{x,w}$ is a polynomial in $-q^{1\over 2}$ without constant term.
\end{itemize}
Therefore there is no cancellation in the left hand side of \Cref{eq:uniq_proof}, thus $P_{x,w}$ can be uniquely recovered from the right hand side.\footnote{Note that in general, the uniqueness of $f(q)-f(1/q)$ does not imply the uniqueness of $f$, without the degree constraint.
}

\end{proof}
\begin{exercise}
	Prove that $P_{x,w}=1$ if $\ell(x,w)=1$.
\end{exercise}

\begin{theorem}
	$P_{x,w}(0)=1$.
\end{theorem}
\begin{proof}Recall that the mobius inversion formula for a poset is
	$$g(y)=\sum_{\hat 0\leq x\leq y}f(x)\iff f(y)=\sum_{\hat 0\leq x\leq y}\mu g(x)$$

\[q^{\ell(x,w)}\bar P_{x,w} = \sum_{y\in[x,w]}R_{x,y}P_{y,w}\]
Recall that $R_{x,y}(0)=(-1)^{\ell(x,w)}$.
So set $q=0$ we get
\[0=\sum_{y\in[x,w]} (-1)^{\ell(x,y)} P_{y,w}(0)\]
Then by Mobius inversion
\[P_{x,w}(0)=-\sum_{x<y\leq w}(-1)^{\ell(x,y)}\cdot 1=-\sum_{x<y\leq w}\mu (x,y)=\mu(x,x)=1\qedhere\]
\end{proof}