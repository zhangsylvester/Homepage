\documentclass{amsart}
\usepackage{parskip}
\usepackage{ytableau}
\DeclareMathOperator{\Ind}{Ind}

\usepackage{stmaryrd}
\usepackage[inline]{showlabels}
%Title Style
\usepackage{mathtools}
\usepackage{etoolbox}
\makeatletter
\patchcmd{\@settitle}{\uppercasenonmath\@title}{}{}{}
\patchcmd{\@setauthors}{\MakeUppercase}{}{}{}
\patchcmd{\section}{\scshape}{}{}{}
\makeatother
\makeatletter
\patchcmd{\@maketitle}
  {\ifx\@empty\@dedicatory}
  {\ifx\@empty\@date \else {\vskip2ex %vertical position of date
  \centering\footnotesize\@date\par\vskip1ex}\fi
   \ifx\@empty\@dedicatory}
  {}{}
\patchcmd{\@adminfootnotes}
  {\ifx\@empty\@date\else \@footnotetext{\@setdate}\fi}
  {}{}{}
\makeatother


%Page Style
\usepackage{mathabx}
\usepackage[colorlinks,backref=page,citecolor=blue]{hyperref}
\usepackage{accents}
\renewcommand{\familydefault}{ppl}
\setlength{\oddsidemargin}{0in}
\setlength{\evensidemargin}{0in}
\setlength{\marginparwidth}{0in}
\setlength{\marginparsep}{0in}
\setlength{\marginparpush}{0in}
\setlength{\topmargin}{0.3in}
\setlength{\headsep}{14pt}
\setlength{\footskip}{.3in}
\setlength{\textheight}{8.0in}
\setlength{\textwidth}{5.8in}
\setlength{\parskip}{4pt}
\linespread{1.2}



\theoremstyle{plain}
\newtheorem{theorem}{Theorem}
\newtheorem{restate}{Theorem}
\newtheorem{lemma}[theorem]{Lemma}
\newtheorem{prop}[theorem]{Proposition}
\newtheorem{conj}[theorem]{Conjecture}
\newtheorem{question}[theorem]{Question}
\newtheorem{corollary}[theorem]{Corollary}
\newtheorem{claim}[theorem]{Claim}
\theoremstyle{definition}
\newtheorem{remark}[theorem]{Remark}
\newtheorem{problem}[theorem]{Problem}
\newtheorem{example}[theorem]{Example}
\newtheorem{definition}[theorem]{Definition}
\DeclareMathOperator{\YY}{\mathbb{Y}}
\DeclareMathOperator{\Y}{\bf{Y}}
\DeclareMathOperator{\id}{id}
\DeclareMathOperator{\sh}{shape}
\DeclareMathOperator{\SYT}{SYT}


\title{Math 206A Final Exam}

\begin{document}
	\maketitle
\begin{itemize}
    
    \item The exam is take-home and due on  \textbf{December 12th at 11:59 PM} (Submit via email).
    \item Open-book/open-note; you may use any resources (including online) \textbf{except} for \textbf{AI tools} or \textbf{human consultation/collaboration}.
    \item Any external result or theorem used \textbf{must be properly cited}.
    \item  Your solution must be \textbf{self-contained}. Any theorem not proven in class \textbf{must be proven} in your submission.
    \item  Your Final Score $= \min(100, \text{Your Total Score})$, where $0 \leq \text{Your Total Score} \leq 200$.\footnote{This means you don't need to submit all problems, but I do think each of the problems is very interesting to work out :)} 
\end{itemize}
	
	
	
	
	
		
		\begin{problem}[20pts]
			Let $\lambda(w)$ denote the corresponding partition of a Grassmanian permutation $w$. Prove that the reduced words of $w$ are in bijection with Standard Young Tableaux of shape $\lambda(w)$.
		\end{problem}
		
		
		\begin{problem}[40 pts]
			Given a partition $\lambda$ and $j\geq 1$, let $m_j$ denote the multiplicity of the parts of $j$ in $\lambda$, so that $\lambda = (1^{m_1(\lambda)},2^{m_2(\lambda)},\dots)$
			
			(a) For each $j\in[n]$, prove that			
			\[	\tag{$*$}		\sum_{\lambda} q^{|\lambda|}m_j(\lambda) = {q^j\over 1-q^j}\prod_{i=1}^{\infty}{1\over 1-q^i}\]
			
			
			\[\tag{$**$}\sum_{\lambda\vdash n} q^{|\lambda|}\cdot\#\{k:m_k\geq j\}={q^j\over 1-q^j}\prod_{i=1}^{\infty}{1\over 1-q^i}\]
			
			(b) Deduce that for fixed $j$, one has 
			\[\sum_{\lambda\vdash n}m_j(\lambda)=\sum_{\lambda\vdash n}\#\{k:m_k\geq j\}\]
			
			(c) From (b), deduce that for any $n\geq 0$, one has
			\[\prod_{\lambda \vdash n}\prod_{i=1}^{\ell(\lambda)}\lambda_i=\prod_{\lambda\vdash n}\prod_{j=1}^{\infty}m_j(\lambda)!\]
			(d) From (c), deduce that
			\[\prod_{\lambda\vdash n}z_\lambda = \left(\prod_{\lambda\vdash n}\prod_{i=1}^{\ell(\lambda)}\lambda_i\right)^2\]
			
		\end{problem}
		
		\begin{problem}[40pts]
			Consider the basis $b_{\lambda}=\sum_{\mu\trianglelefteq\lambda,|\mu|=|\lambda|}m_{\mu}$ of $\Lambda$. Does each Schur function $s_{\lambda}$ expand non-negatively into this basis? 
		\end{problem}
		
		\begin{problem}[40pts]
			Consider the generating function $$H(t)={1\over \sum_{n\geq 0} (-1)^ne_n(x) t^n}.$$ Expand $H(t)$ in $e_n, h_n$ and usual $x_i$ variables. (Hint: use the Cauchy identity $\prod_{\lambda} s_\lambda(x) s_{\lambda}(y)=\prod_{i,j} {1\over 1-x_iy_j}$.)
		\end{problem}
		
		\begin{problem}[60pts] Denote the trivial representation of $S_4$ by $\rho_{(4)}$ and the sign representation by $\rho_{(1,1,1,1)}$.
		
		(a) Show that the defining representation (which permutes the coordinates in $\mathbb{C}^4$) decomposes
		\[\rho_{def}=\rho_{(4)}\oplus \rho_{(1,1,1,1)}\]
		
		(b) Show that the representation $\rho_p$ which permutes all unordered pairs $(i,j)\in {[4]\choose 2}$ decomposes
		\[\rho_{p}=\rho_{(4)}\oplus \rho_{(1,1,1,1)}\oplus \rho_{(2,2)}\]
		where $\rho_{(2,2)}$ is an irreducible representation of degree $2$. 
		
		(e) Set $\rho_{(2,1,1)}=\rho_{(1,1,1,1)}\otimes\rho_{(3,1)}$. Show that it's irreducible, and that $\rho_{(4)},\rho_{(3,1,1)}
,\rho_{(2,2)},\rho_{(2,1,1)}
,\rho_{(1,1,1,1)}$ give a complete list of irreducible representations of $S_4$.

(f) Write down the conjugacy classes and character table for $S_4$.
			
		\end{problem}
	
		
		



	
	
	
	
	
	
	
	
\end{document}