\documentclass{amsart}
\usepackage{ytableau}


\usepackage{stmaryrd}
\usepackage[inline]{showlabels}
%Title Style
\usepackage{mathtools}

\usepackage{etoolbox}




\makeatletter
\patchcmd{\@settitle}{\uppercasenonmath\@title}{}{}{}
\patchcmd{\@setauthors}{\MakeUppercase}{}{}{}
\patchcmd{\section}{\scshape}{}{}{}
\makeatother
\makeatletter
\patchcmd{\@maketitle}
  {\ifx\@empty\@dedicatory}
  {\ifx\@empty\@date \else {\vskip2ex %vertical position of date
  \centering\footnotesize\@date\par\vskip1ex}\fi
   \ifx\@empty\@dedicatory}
  {}{}
\patchcmd{\@adminfootnotes}
  {\ifx\@empty\@date\else \@footnotetext{\@setdate}\fi}
  {}{}{}
\makeatother


%Page Style
\usepackage{mathabx}
\usepackage[colorlinks,backref=page,citecolor=blue]{hyperref}
\usepackage{accents}
\renewcommand{\familydefault}{ppl}
\setlength{\oddsidemargin}{0in}
\setlength{\evensidemargin}{0in}
\setlength{\marginparwidth}{0in}
\setlength{\marginparsep}{0in}
\setlength{\marginparpush}{0in}
\setlength{\topmargin}{0.3in}
\setlength{\headsep}{14pt}
\setlength{\footskip}{.3in}
\setlength{\textheight}{8.0in}
\setlength{\textwidth}{5.8in}
\setlength{\parskip}{4pt}
\linespread{1.2}
\usepackage{cleveref}



\theoremstyle{plain}
\newtheorem{theorem}{Theorem}[section]
\newtheorem{restate}{Theorem}[section]
\newtheorem{lemma}[theorem]{Lemma}
\newtheorem{prop}[theorem]{Proposition}
\newtheorem{conj}[theorem]{Conjecture}
\newtheorem{question}[theorem]{Question}
\newtheorem{corollary}[theorem]{Corollary}
\newtheorem{claim}[theorem]{Claim}
\theoremstyle{definition}
\newtheorem{remark}[theorem]{Remark}
\newtheorem{example}[theorem]{Example}
\newtheorem{definition}[theorem]{Definition}

\newcommand{\todo}[1]{{\color{red}[To Do: #1]}}
\newcommand{\syl}[1]{{\color{orange}[S: #1]}}
\DeclareMathOperator{\zz}{\mathbb{Z}}
\DeclareMathOperator{\nn}{\mathbb{N}}
\DeclareMathOperator{\yy}{{\bf Y}}
\DeclareMathOperator{\YY}{\mathbb{Y}}
\DeclareMathOperator{\qq}{\mathbb{Q}}
\DeclareMathOperator{\id}{\textbf{1}}
\DeclareMathOperator{\xn}{\textbf{x}_n}
\DeclareMathOperator{\xx}{\textbf{x}}
\DeclareMathOperator{\sort}{sort}

\DeclareMathOperator{\Ln}{\Lambda^{(n)} }
\DeclareMathOperator{\comp}{\textbf{C}}

\ytableausetup{centertableaux}
\let\oldydiagram\ydiagram
\renewcommand{\ydiagram}[1]{\scalebox{0.5}{$\oldydiagram{#1}$}}


\title{Tableaux Combinatorics and Symmetric Functions}
\author{Sylvester W. Zhang$^\sharp$}
\thanks{$^\sharp$\href{mailto:sylvesterzhang@math.ucla.edu}{sylvesterzhang@math.ucla.edu} UCLA}
\date{\today}

\begin{document}


\maketitle



\begin{abstract}
	These are lecture notes for Math 206a (Algebraic combinatorics) at UCLA, Fall 2025.
\end{abstract}
\tableofcontents
Let $R$ be a commutative ring, usually taken to be $\qq$ or $\zz$.

Denote $\xn=(x_1,\cdots,x_n)$, and $\xx=(x_1,x_2,\cdots)$
\section{Partitions}
For $n\in\nn$, a \emph{composition} of $n$ is a sequence of integers $\alpha = (\alpha_1,\cdots,\alpha_l)$ such that $\alpha_1+\cdots+\alpha_l=n$ and $a_i\geq 1$. Further, we define \emph{weak compositions} by allowing zeros. Define $\comp_n$ the set of all weak compositions of $n$. Note that since $0$'s are allows, there are infinitely many weak compositions for each $n$. Sometimes we want to consider all compositions of any number, $\comp :=\bigoplus \comp_n$. In fact, $\comp$ is nothing but the semi-infinite product $\nn^{\infty}$. 

An integer partition of $n\in\nn $ is a special kind of composition $\lambda=(\lambda_1,\lambda_2,\cdots,\lambda_k)$ of $n$ with the property that $\lambda_1\geq\lambda_2\geq\cdots \geq \lambda_k$. We denote $\lambda\vdash n$, and $\ell(\lambda)=k$. We may think of $\lambda$ as an element of $\nn^\infty$ by setting $\lambda_i=0$ for $i>\ell(\lambda)$. For a weak composition $\alpha\in\comp_n$, we denote $\sort(\alpha)$ to be the unique partition in $\yy_n$ that's equal to $\alpha$ as a multi-set.


We will represent an integer partition using a \emph{Young diagram}, which is a stack of boxes so that the $i$-th row has $\lambda_i$ boxes. For example, the following is the Young diagram corresponds to $(4,3,1)$.
\[{\ydiagram{4,2,1}}\]

Denote $\yy_n$ the set of all partitions of $n$, and denote $\yy=\bigoplus_{i\in\mathbb{N}} \yy_i$.

There are two natural partial orders equipped with $\yy$. The first one, called \emph{Young's natural order}, is defined so that $\lambda\leq \mu$ when $\lambda_i\leq \mu_i$ for all $i$. In other words, the Young diagram of $\mu$ contains that of $\lambda$. In Young's order, $\lambda$ is covered by $\mu$ precisely when $\mu$ has exactly one more box than $\lambda$. For example, $(4,2,1)\lessdot (4,3,1)$.
\[{\ydiagram{4,1,1}}\lessdot {\ydiagram{4,2,1}}\]

This partial order turns out to be a distributive lattices, hence called the \emph{Young's lattice.}

Let $\YY=\qq\text{-span}(\yy)$ be the vector space over $\qq$ whose basis are Young diagrams. Define two natural operators on $\YY$ as follows:
\[U(\lambda)=\sum_{\lambda\lessdot \mu}\mu\quad\quad D(\lambda)=\sum_{\mu\lessdot \lambda}\mu\]
(The actions are defined on the basis then extend by linearity.) For example,
\[U\left(\ydiagram{2,1}\right)=\ydiagram{3,1}+\ydiagram{2,2}+\ydiagram{2,1,1}\]

These operators satisfy the following identity.
\begin{prop}
$DU-UD=\id$.	
\end{prop}
\begin{proof}
	Exercise. Hint: given any partition $\lambda$, notice that there's always one more outer corner than inner corner.
\end{proof}
The algebra of the up/down operators is the \emph{Weyl algebra}, denoted $\mathcal{W}$, which is the algebra of multiplication and differentiation on $k[x]$ since $[{d\over dx},x]=1$.

\begin{remark}
	For any poset one can define the operators $U$ and $D$, but not all satisfy the relation of the Weyl algebra. In \cite{stanley1988differential}, Stanley defined \emph{differential posets} to be those that affords a combinatorial representation of $\mathcal{W}$, with $\yy$ being the canonical example.
\end{remark}

\begin{prop}
	Consider $D^nU^n$ as an element of $\mathcal{W}$, one can rewrite so that all $U$'s appear before $D$'s. Then the identity coefficient is $n!$.
\end{prop}
\begin{proof}
	Exercise.
\end{proof}
\begin{example}
	$D^2U^2=D(UD+1)U=DUDU+DU=(UD+1)(UD+1)+(UD+1)=UDUD+3UD+2=U(UD+1)D+3UD+2=U^2D^2+4UD+2$.
\end{example}

The second partial order that we will introduce, is a partial order defined on $\yy_n$ (although extendable to $\yy$, it is more natural to consider as an order on $\yy_n$ for each $n$).

\begin{definition}
	$\lambda\trianglelefteq \mu$ if and only if $\lambda_1+\cdots+\lambda_k\leq \mu_1+\cdots+\mu_k$ for all $k$\footnote{Recall that $\lambda_i=0$ if $i>\ell(\lambda)$.}. This is called the \emph{dominance order}. 
\end{definition}
\section{Symmetric polynomials {\color{red}under construction}}
Let $R$ be a commutative ring, usually taken to be $\qq$ or $\zz$. We say a polynomial $f\in R[x_1,\cdots,x_n]$ is \emph{symmetric} to be all polynomials that are invariant under $S_n$, i.e. 
$$f(x_1,\cdots,x_n)=f(x_{\sigma(1)},\cdots,x_{\sigma(n)})$$
for all $\sigma\in S_n$.

\begin{definition}
	Define $\Lambda_{R,n}:=R[\xn]^{S_n}\subset R[\xn]$ the subring of all symmetric polynomials.
\end{definition}
Given any monomial $\prod x_i^{\alpha_i}$, one can find the smallest symmetric polynomial that contains it. This is exactly the sum of all possible rearrangements of $\alpha=\{\alpha_1,\cdots,\alpha_l\}$. All symmetric polynomials that arise in this way are called \emph{monomial symmetric polynomials}, defined as follows.
\begin{definition}
	For $\lambda\in \yy$ such that $\ell(\lambda)\leq n$. Define \emph{monomial symmetric polynomial} $m_\lambda$ to be
	\[m_\lambda=\sum_{\alpha\in\comp_n,\ \sort(\alpha)= \lambda} \xn^\alpha=\sum_{\alpha\text{ a rearrangement of }\lambda} \xn^\alpha\]
	
\end{definition}
It is clear that these polynomials form a basis of $\Lambda_{R,n}$. At the same time, there's another natural (but less obvious) basis for the ring symmetric polynomials --- the \emph{elementary symmetric polynomials.}
\begin{definition}[elementary symmetric polynomials] For $k\in\nn$, define \[e_k(\xn)=\sum_{1\leq i_1<i_2<\cdots<i_k}x_{i_1}\cdots x_{i_k}.\]
And for $\lambda\in\yy$, define $e_\lambda(\xn)=e_{\lambda_1}(\xn)\cdots e_{\lambda_{l}}(\xn)$
\end{definition}
That fact that $e_\lambda$ is also a basis for $\Lambda_{R,n}$, is called the \emph{Fundamental theorem of symmetric polynomials.}
\begin{theorem}[Fundamental Theorem of Symmetric Polynomials] \label{thm:ftsp}
	There is an isomorphism between the ring of symmetric polynomials $\Lambda_{R,n}$ and the polynomials ring of $n$ variables $R[t_1,\cdots,t_n]$, via the map $e_n\mapsto t_n$.
\end{theorem}
\begin{proof}
	
\end{proof}
Now consider the ring of symmetric polynomials of differently many variables. There is an obvious surjective homomorphism from $\Lambda_{R,j}$ to $\Lambda_{R,i}$  ($i\leq j$):
\[\rho_{i,j}:\Lambda_{R,j}\to \Lambda_{R,i}\quad f(x_1,\cdots,x_i,x_{i+1}\cdots,x_j)\mapsto f(x_1,\cdots,x_i,0,\cdots,0).\]
With the help of the fundamental theorem (\Cref{thm:ftsp}), one can write down injective homomorphisms $\phi_{i,j}$ of which $\rho_{i,j}$ are inverse to. For $i\leq j$, we define
\[\phi_{i,j}:\Lambda_{R,i}\to \Lambda_{R,j}\quad e_k(\xx_n)\to e_k(\xx_m)\]
It is easy to check that:
\begin{lemma}
	\begin{itemize}
		\item $\phi_{i,i}=\id$ for all $i$, and
		\item $\phi_{i,k}=f_{j,k}f_{i,j}$ for all $i\leq j\leq k$.
	\end{itemize}
	Thus $\{\Lambda_{R,i},\phi_{i,j}\}$ form a direct system.
\end{lemma}
We are thus able to take the direct limit of said direct system. The ring obtained this way is called the ring of symmetric functions.
\begin{definition}
	Recall the definition of a direct limit
	\[\varinjlim \Lambda_{R,n}=\left(\bigsqcup_{i\in\nn}\Lambda_{R,i}\right)\bigg/\sim\]
	where $\sim$ is the equivalence relation defined as follows. If $f\in \Lambda_{R,i}$ and $g\in \Lambda_{R,j}$ with $i<j$, then $f\sim g$ iff there exists $k>j$ such that $\phi_{i,k}(f)=\phi_{j,k}(g)$. We then define $\Lambda_{R}:=\varinjlim \Lambda_{R,n}$.
\end{definition}
Alternatively one can define symmetric functions as symmetric power series with countable degree, which is the approach we take in next section. The direct limit definition allows us to see $\Lambda_R$ as a formal infinite union of subrings that are rings of symmetric polynomials, therefore anything that happens in $\Lambda_R$ is actually just something in $\Lambda_{R,n}$. This will become clear in future sections.

\section{Symmetric Functions}
Let us now define the ring of symmetric functions more directly.
\begin{definition}
	Define the \emph{ring of symmetric functions} over $R$, denoted $\Lambda_R$, to be the subring of the ring of formal power series $R[[\xx_{\infty}]]$,
	\[\Lambda_R=\{f\in R[[\xx_\infty]]:
		f \text{ is invariant under }S_{\infty}\text{ and }\deg(f)\text{ is bounded}.
	\}\]
\end{definition}
%A priori it is not obvious why this definition agrees with the direct limit definition in the previous section.
Note that if a power series is symmetric, then each of its homogeneous component must be symmetric. Thus $\Lambda_n$ has the structure of a graded ring, which is a fundamental property.
\begin{prop}
Let $\Lambda_R^{(n)}$ be the subring of $\Lambda_n$ consisting of homogeneous power series of degree $n$. Then
\[\Lambda_R = \bigoplus_i \Lambda_R^{(n)}\]	
\end{prop}
The ring $\Ln$ (and thus $\Lambda_R$) has several important basis. We put forward their definitions, and discuss their properties for the rest of the section.

\begin{definition}[monomial symmetric functions]
For $\lambda\in \yy$, define 
\[m_\lambda = \sum_{\alpha\in\comp,\ \sort(\alpha)=\lambda}\xx^\alpha\]
where $\xx^{\alpha} =\prod_{i\geq 1} x_i^{\alpha_i}$.
	
\end{definition}
\begin{definition}[elementary symmetric functions] For $k\in\nn$, define
\[e_k := \sum_{1\leq i_1<i_2<\cdots<i_k} x_{i_1}\cdots x_{i_k}\]
 and for $\lambda\in \yy_n$, define $e_{\lambda}=e_{\lambda_1}e_{\lambda_2}\cdots e_{\lambda_l}$.
 
	
\end{definition}
\begin{definition}[complete homogeneous symmetric functions] For $k\in\nn$, define
\[h_k := \sum_{1\leq i_1\leq i_2\leq \cdots \leq i_k} x_{i_1}\cdots x_{i_k}\]
 and for $\lambda\in \yy_n$, define $h_{\lambda}=h_{\lambda_1}h_{\lambda_2}\cdots h_{\lambda_l}$.
 \end{definition}
 \begin{definition}[power-sum symmetric functions] For $k\in\mathbb{N}$, define
 \[p_k = m_{1^k}=x_1^k+x_2^k+\cdots\]
 and for $\lambda\in \yy_n$, define $p_\lambda= p_{\lambda_1}\cdots p_{\lambda_l}$.
 	
 \end{definition}



\section{Robinson-Schensted Correspondence}
\section{Greene-Kleitman Theory}
\bibliographystyle{amsalpha}
\bibliography{refs.bib}

\end{document}