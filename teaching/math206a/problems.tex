\documentclass{amsart}

\usepackage{stmaryrd}
\usepackage[inline]{showlabels}
%Title Style
\usepackage{mathtools}
\usepackage{etoolbox}
\makeatletter
\patchcmd{\@settitle}{\uppercasenonmath\@title}{}{}{}
\patchcmd{\@setauthors}{\MakeUppercase}{}{}{}
\patchcmd{\section}{\scshape}{}{}{}
\makeatother
\makeatletter
\patchcmd{\@maketitle}
  {\ifx\@empty\@dedicatory}
  {\ifx\@empty\@date \else {\vskip2ex %vertical position of date
  \centering\footnotesize\@date\par\vskip1ex}\fi
   \ifx\@empty\@dedicatory}
  {}{}
\patchcmd{\@adminfootnotes}
  {\ifx\@empty\@date\else \@footnotetext{\@setdate}\fi}
  {}{}{}
\makeatother


%Page Style
\usepackage{mathabx}
\usepackage[colorlinks,backref=page,citecolor=blue]{hyperref}
\usepackage{accents}
\renewcommand{\familydefault}{ppl}
\setlength{\oddsidemargin}{0in}
\setlength{\evensidemargin}{0in}
\setlength{\marginparwidth}{0in}
\setlength{\marginparsep}{0in}
\setlength{\marginparpush}{0in}
\setlength{\topmargin}{0.3in}
\setlength{\headsep}{14pt}
\setlength{\footskip}{.3in}
\setlength{\textheight}{8.0in}
\setlength{\textwidth}{5.8in}
\setlength{\parskip}{4pt}
\linespread{1.2}



\theoremstyle{plain}
\newtheorem{theorem}{Theorem}[section]
\newtheorem{restate}{Theorem}[section]
\newtheorem{lemma}[theorem]{Lemma}
\newtheorem{prop}[theorem]{Proposition}
\newtheorem{conj}[theorem]{Conjecture}
\newtheorem{question}[theorem]{Question}
\newtheorem{corollary}[theorem]{Corollary}
\newtheorem{claim}[theorem]{Claim}
\theoremstyle{definition}
\newtheorem{remark}[theorem]{Remark}
\newtheorem{problem}[theorem]{Problem}
\newtheorem{example}[theorem]{Example}
\newtheorem{definition}[theorem]{Definition}
\DeclareMathOperator{\YY}{\mathbb{Y}}
\DeclareMathOperator{\Y}{\bf{Y}}
\DeclareMathOperator{\id}{id}
\title{Math 206A Problem Sets}

\begin{document}
	\maketitle
	\section{Homework Assignment 1 (Due Oct 10 F)}
	
	\begin{problem}
		Recall that $\mathbb{Y} = \mathbb{Q}\text{-span}(\bf{Y})$, where $\bf{Y}$ is the Young's lattice, and the up/down operators $U,D$ defined by $U(\lambda)=\sum_{\lambda\lessdot \mu}\mu$ and  $D(\lambda)=\sum_{\mu\lessdot \lambda}\mu$. Prove that 
		\[[D,U]=\text{id}\]
	\end{problem}
	
	\begin{problem}
		Recall the Weyl algebra $\mathcal{W}$ is the $\mathbb{Z}$-algebra with unit $1$ and generators $U,D$ with relations $[D,U]=1$. Rewrite $D^nU^n$ as an $\mathcal{W}$-element such that no $D$ appears before an $U$. What is the identity coefficient?
		
		For example, $D^2U^2=D(UD+1)U=DUDU+DU=(UD+1)(UD+1)+(UD+1)=UDUD+3UD+2=U(UD+1)D+3UD+2=U^2D^2+4UD+2$. The identity coefficient is $2$.
	\end{problem}
	
	\begin{problem}
		Prove that
		\[\sum_{k\geq 0} h_k t^k = \prod_{i\geq0}{1\over 1-x_it}\]
	\end{problem}
	
	\begin{problem}
		Is the power series $f=\prod_{i\geq 1}(1+x_i+x_i^2)$ symmetric? If so, expand in the $e$-basis.
	\end{problem}
		\begin{problem}
		Expand $h_3e_4$ in the Schur basis. 	
	\end{problem}
	\begin{problem}
		[Bonus] Expand $h_me_n$ in the Schur basis. You may start experimenting with {\tt{SageMath}} to make a conjecture.
	\end{problem}
	
	
	\section{Homework Assignment 2 (Due Oct 20 M)}
	\begin{problem}
		Prove that if $K_{\lambda\mu}\neq 0$, then $\lambda\trianglelefteq \mu$ (the dominance order).
	\end{problem}

	
	\section{Homework Assignment 3 (Due Oct 29 W)}
	\section{Homework Assignment 4 (Due Nov 7 F)}
	\section{Homework Assignment 5 (Due Nov 17 M)}
	\section{Homework Assignment 6 (Due Nov 26 W)}
	\section{Homework Assignment 7 (Due Dec 8 F)}


	
	
	
	
	
	
	
	
\end{document}