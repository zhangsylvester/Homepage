\documentclass{amsart}
\usepackage{soul}
\usepackage[margin=1in]{geometry}
%packagrs

\usepackage{ifthen}
\usepackage[utf8]{inputenc}
\usepackage{setspace}
\usepackage[english]{babel}
\usepackage{enumerate}
\usepackage{thmtools}
\usepackage[shortlabels]{enumitem}
\usepackage[T1]{fontenc}
\usepackage{tikz}
\usetikzlibrary{shapes.geometric,intersections,decorations.markings,snakes}
\usetikzlibrary{calc,intersections,through,backgrounds}
\usetikzlibrary{patterns}


\renewcommand{\l}[2]{\lambda_{#1#2}}
\newcommand{\bs}[1]{\boldsymbol{#1}}
\newcommand{\tei}[0]{Teichm\"uller}


\DeclareMathOperator{\Hom}{Hom}
\DeclareMathOperator{\psl}{PSL}
\DeclareMathOperator{\rr}{\mathbb{R}}
\DeclareMathOperator{\osp}{Osp}
\DeclareMathOperator{\wt}{wt}
\DeclareMathOperator{\twt}{twt}
\DeclareMathOperator{\crs}{cross}
\DeclareMathOperator{\ber}{Ber}
\DeclareMathOperator{\st}{st}
\DeclareMathOperator{\tp}{t}
\DeclareMathOperator{\id }{id}

\newcommand{\mb}[1]{\mathbb{#1}}
\newcommand{\tld}[1]{\widetilde{#1}}
%
\renewcommand{\arraystretch}{1.45}
\newcommand{\ospmatrix}[9]{
\left(\begin{array}{cc|c}
#1&#2& #3 \\
#4 &#5& #6 
\\
\hline
#7 &#8 &#9
\end{array}
\right)
}

\newcommand{\A}[2]{A\left({#1}\middle| {#2}\right)}

\newcommand{\hookdoubleheadrightarrow}{%
  \hookrightarrow\mathrel{\mspace{-15mu}}\rightarrow
}


\begin{document}
	\leftline{\bf Math 4242 Homework 3}
	\begin{enumerate}
		\item 	Let $V=\rr^3$ and $W = \rr_{\leq 2}[x]$. Let $T(a,b,c)=a+b(x-1)+c(x-1)^2$. Is $T$ linear? If so, identify a basis for $V$ and $W$ and write down the matrix $\mathcal{M}(T)$.\begin{proof}
			$T$ is linear because $T((a_1,b_1,c_1)+(a_2,b_2,c_2))=(a_1+a_2)+(b_1+b_2)(x-1)+(c_1+c_2)(x-1)^2=T(a_1,b_1,c_1)+T(a_2,b_2,c_2)$. And $T(ka,kb,kc)=k(a+b(x-1)+c(x-1)^2)=kT(a,b,c)$.
			
			Consider $e_1,e_2,e_3$ the standard basis for $V$, and let $w_1=1,w_2=(x-1),w_3=(x-1)^2$, which form a basis for $W$. (Can check that they are linearly independent, and the number of vectors is equal to the dimension of $W$, so they must form a basis).
			
			Now we calculate $\mathcal M(T)$.
			
			$T(e_1)=1=w_1$, $T(e_2)=(x-1)=w_2$, and $T(e_3)=(x-1)^2=w_3$, thus $\mathcal M(T)=\begin{pmatrix}
				1&0&0\\0&1&0\\0&0&1
			\end{pmatrix}$.
			
			If we use the usual basis $1,x,x^2$ for $W$, then since
			$T(e_1)=1$, $T(e_2)=x-1$, and $T(e_3)=x^2-2x+1$, we have that 
			$\mathcal M(T)=\begin{pmatrix}
				1&-1&1\\
				0&1&-2\\
				0&0&1
			\end{pmatrix}$
			
		\end{proof}
		\item Consider the linear map $T:M_{2,2}(\mathbb{R})\to \mathbb{R}^2$ given by
		\[T\left(\begin{bmatrix}
			a&b\\c&d
		\end{bmatrix}\right)=(a-b,c+d)\] 
		Find a basis for $\text{Ker}(T)$ and $\text{Img}(T)$.
		\begin{proof}
			The kernal is $\ker(T)=\{\begin{bmatrix}
			a&a\\c&-c
		\end{bmatrix}: a,c\in \rr\}$ Note that any matrix in $\ker(T)$ can be written uniquely as 
		\[\begin{bmatrix}
			a&a\\c&-c
		\end{bmatrix}=a\begin{bmatrix}
			1&1\\0&0
		\end{bmatrix}+c\begin{bmatrix}
			0&0\\1&-1
		\end{bmatrix}.\]
		Therefore $\begin{bmatrix}
			1&1\\0&0
		\end{bmatrix}$ and $\begin{bmatrix}
			0&0\\1&-1
		\end{bmatrix}$ for a basis for $\ker(T)$.
		
		We know that $dim(\text{Img}(T))=\dim(M_{2,2})-\dim(\ker(T))=4-2=2$. Note that this is the same dimension as $\rr^2$, but $\text{img}(T)\subseteq \rr^2$, we must have that $\text{Img}(T)=\rr^2$. Thus a basis is $(1,0),(0,1)$.
		\end{proof}
		\item Suppose $T\in \text{End}(V)$ is an invertible map. Prove that if $v_1,\cdots,v_n$ is a basis, then $Tv_1,\cdots,Tv_n$ is also a basis.
		
	\begin{proof}
		Know that $\dim(V)=n$. It suffices to show that $Tv_1,\cdots,Tv_n$ is linearly independent.
		
		Suppose $0=a_1Tv_1+\cdots+a_nTv_n$, we need to show that the only possibility is that $a_1=\cdots=a_n=0$. Left multiply both sides by $T^{-1}$ (since $T$ is invertible), we have
		\[T^{-1}0=T^{-1}(a_1Tv_1+\cdots+a_nTv_n)\]
		This is equivalent to
		\[0=a_1v_1+\cdots+a_n v_n\]
		Using the fact that $v_1,\cdots,v_n$ are linearly independent, we conclude that the choice of $a_1,\cdots,a_n$ is unique. This completes the proof.
		
	\end{proof}
		\item Prove that (a) $(U+W)^0=U^0\cap W^0 $ (b) $(U\cap W)^0=U^0+W^0$.
		\begin{proof}
			(a) A linear function annihilates $U+W$
 if and only if it annihilates both $U$
 and $W$.
 
 (b) We first show that $(U\cap W)^0\subset U^0+W^0$. Take any $f\in (U\cap W)^0$, i.e. $f$ satisfy the property $f(v)=0$ for all $v\in U\cap W$. We want to write $f=g+h$ where $g\in U^0$ and $h\in W^0$. We can simply define $g(v)=f(v)$ for all $v\in U$ and $g(v)=0$ otherwise, and $h(v)=f(v)$ for all $v\in W$ and $h(v)=0$ otherwise. It can be easily checked that $f=g+h$ and $g\in U^0,h\in W^0$. Thus $(U\cap W)^0\subset U^0+W^0$.
 
 We next prove the other direction. Suppose $f=g+h$ with $g\in U^0$ and $h\in W^0$, we need to show that $f\in (U\cup W)^0$. It suffices to show that $f(v)=0$ if $v\in U\cup W$. Suppose $v\in U\cup W$, then $f(v)=g(v)+h(v)=0
 +0=0$, meaning that $f\in (U\cup W)^0$. we are done.
  \end{proof}
		\item Let $T:\mathbb{R}^3\to \mathbb{R}^2$ defined by
$T(x,y,z) = (2x+3y+4z,3x+4y+5z)$. Let $e_1,e_2,e_3$ denote the standard basis of $\mathbb{R}^3$ and $f_1,f_2$ denote the standard basis of $\mathbb{R}^2$.
(a) Describe the linear functionals $T^*(f_1^*)$ and $T^*(f_2^*)$.
(b) Write $T^*(f_1^*)$ and $T^*(f_2^*)$ as linear combinations of $e_1^*,e_2^*,e_3^*$.

(Note: There was a typo in the original problem, doing this problem in either way is fine.)

\begin{proof}
	(a) $T^*(f_1^*)(x,y,z)=f_1^*T(x,y,z)=f_1*(2x+3y+4z,3x+4y+5z)=2x+3y+4z$.
	
	$T^*(f_2^*)(x,y,z)=f_2^*T(x,y,z)=f_2*(2x+3y+4z,3x+4y+5z)=3x+4y+5z$.
	
	(b) $T^*(f_1^*)(x,y,z)=2x+3y+4z=2e_1^*+3e_2^*+4e_3^*$
	
	$T^*(f_2^*)(x,y,z)=3x+4y+5z=3e_1^*+4e_2^*+5e_3^*$

	
\end{proof}
\item Suppose $U$ is a subspace of $V$, and $\pi:V\to V/U$ the quotient map. Consider the dual of the quotient map $\pi^*\in \text{Hom}((V/U)^*,V^*)$. Show that $\text{Img}(\pi^*)=U^0$ and $\pi^*$ is an isomorphism $(V/U)^*\cong U^0$.

\begin{proof}
	Recall that the map $\pi$ is defined as $\pi(v)=v+U$, and $\pi^*(f)=f\circ \pi$ for $f: V/U\to \ff$. Moreover $\pi^*(f)(v)=f\circ \pi(v)=f(v+U)$ which equals to $0$ when $v+U=U$ i.e. $v\in U$. Thus $\text{img}(\pi^*)=U^0$. Therefore $\pi^*$ is surjective map from $(V/U)^*$ to $U^0$.
	
	To prove it's an isomorphism, we only need that it's injective, i.e. $\ker(\pi^*)=0$.
	
	Let $f$ be a linear functional on $V/U$. Suppose $\pi^*(f)=0$, then $f(v+U)=0$ for all $v$, which means that $f={\bf 0}$ (the zero vector in $(V/U)^*$), therefore $\ker(\pi^*)=\{0\}$, hence $\pi^*$ is injective.
	
	Therefore $\pi*$ is a bijective linear map from $(V/U)^*$ and $U^0$, thus an isomorphism.
	
\end{proof}
\item {OS 3.1.9}
\begin{proof}
Done in lecture.
\end{proof}
		\item {OS 3.1.17} 
		\begin{proof}
	
			
			\[\langle A,A\rangle=\text{tr}(A^TA)=\sum_{i=1}^n(A^TA)_{ii}=\sum_{i=1}^n\sum_{j=1}^mA^T_{ij}A_{ji}=\sum_{i=1}^m\sum_{j=1}^nA_{ij}^2 \ge 0\]
			and
			\[\langle A,A\rangle=\sum_{i=1}^m\sum_{j=1}^nA_{ij}^2 = 0\iff (A_{ij}=0 \quad \forall i,j) \iff A=0\]
			Since $\text{tr}(X^T)=\text{tr}(X), \quad \text{tr}(X+Y)=\text{tr}(X)+\text{tr}(Y), \quad \text{tr}(\lambda X)=\lambda\text{tr}(X)$ for every matrix $X,Y$, we have
			\begin{eqnarray*}
\langle A,B\rangle&=&\text{tr}(B^TA)=\text{tr}((B^TA)^T)=\text{tr}(A^TB)=\langle B,A\rangle,\\
\langle \lambda A+B,C\rangle&=&\text{tr}(C^T(\lambda A+B))=\text{tr}(\lambda C^TA+C^TB)=\lambda\text{tr}(C^TA)+\text{tr}(C^TB)\\
&=&\lambda\langle A,C\rangle+\langle B,C\rangle.
\end{eqnarray*}
		\end{proof}
	 	\end{enumerate}
	 	
	
\end{document}