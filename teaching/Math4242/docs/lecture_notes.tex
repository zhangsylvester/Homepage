\documentclass[12pt]{amsart}
%\usepackage[utf8]{inputenc}
\usepackage{adjustbox}
%\newcommand{\ff}[0]{\mathbb{F}}
\newcommand{\rr}[0]{\mathbb{R}}
\newcommand{\cc}[0]{\mathbb{C}}

\DeclareMathOperator{\s}{span}
\usepackage{amsmath}
\usepackage{amsfonts}
\usepackage{xcolor}
%\input{tikzsetup.tex}
\usepackage{ytableau}
\usepackage{mathtools, amssymb, old-arrows}
\usepackage[margin = 1in]{geometry}
%\input{theorems}
\usepackage{tikz-cd}

\renewcommand{\leq}{\leqslant}

\usepackage{mathbbol}
\usepackage{amssymb}            % AMS Math

\DeclareSymbolFontAlphabet{\amsmathbb}{AMSb}


\newlength\friezelen 
\settowidth{\friezelen}{$\xi_{m}$} % calculate width of widest element

\usepackage{array} % for "\newcolumntype" macro
\newcolumntype{Q}{>{\centering}p{\friezelen}<{}}

\usepackage{mathrsfs} 
\usepackage{mathtools}
\usepackage{verbatim}
\usepackage{comment}
\usetikzlibrary{arrows}
\tikzset {->-/.style={decoration={markings, mark=at position .5 with {\arrow{latex}}}, postaction={decorate}}}
\tikzset {-->-/.style={decoration={markings, mark=at position .5 with {\arrow[scale=2]{latex}}}, postaction={decorate}}}
\newcommand{\midarrow}{\tikz \draw[-triangle 90] (0,0) -- +(.1,0);}
\newcommand{\miduparrow}{\tikz \draw[-triangle 90] (0,0) -- +(.1,0);}
\newcommand{\midrevarrow}{\tikz \draw[-triangle 90] (0,0) -- +(-.1,0);}
\newcommand{\shiftright}[2]{\makebox[#1][r]{\makebox[0pt][l]{#2}}}

\newcommand{\lhdot}[0]{\adjustbox{lap={\width}{0.6em}}{$\cdot$}\lhd}
%\renewcommand{\ker}[0]{\textup{Ker}}

\usepackage{comment}
\usepackage{graphicx}
\usepackage{color}
\usepackage{etoolbox}
\usepackage[margin=1in]{geometry}
\usepackage{amsmath,amsthm,amssymb,graphicx,tikz,tikz-cd}
\usepackage{hyperref}
\hypersetup{
    colorlinks = true,
    citecolor = orange,%
}
\usepackage[noabbrev]{cleveref}
\usepackage{subcaption}

\makeatletter
\patchcmd{\@settitle}{\uppercasenonmath\@title}{}{}{}
\patchcmd{\@setauthors}{\MakeUppercase}{}{}{}
\patchcmd{\section}{\scshape}{}{}{}
\makeatother
\makeatletter
\patchcmd{\@maketitle}
  {\ifx\@empty\@dedicatory}
  {\ifx\@empty\@date \else {\vskip2ex %vertical position of date
  \centering\footnotesize\@date\par\vskip1ex}\fi
   \ifx\@empty\@dedicatory}
  {}{}
\patchcmd{\@adminfootnotes}
  {\ifx\@empty\@date\else \@footnotetext{\@setdate}\fi}
  {}{}{}
\makeatother

\DeclareMathOperator{\fH}{\mathfrak{H}}
\renewcommand{\l}[2]{\lambda_{#1,#2}}


\DeclareMathOperator{\fA}{\mathfrak{A}}
\DeclareMathOperator{\fC}{\mathfrak{C}}
\DeclareMathOperator{\spin}{spin}
\newcommand{\fh}[0]{\mathfrak{h}}
\newcommand{\fhh}[0]{\mathfrak{\tilde h}}
\newcommand{\fc}[0]{\mathfrak{C}}
\newcommand{\fcc}[0]{\mathfrak{\tilde C}}
\newcommand{\fff}[0]{\mathbb{F}}
\newcommand{\bbb}[0]{\mathbb{B}}
\renewcommand{\geq}{\geqslant}
\renewcommand{\ker}{\Ker}
\DeclareMathOperator{\inv}{inv}

\DeclareMathOperator{\pd}{\textup{\textbf{PD}}}
\DeclareMathOperator{\rpd}{\textup{\textbf{RPD}}}
\DeclareMathOperator{\Ker}{Ker}


\newcommand{\syl}[1]{{\color{orange}[Sylvester: #1]}}
\newcommand{\new}[1]{{#1}}
%\newcommand{\calP}[0]{\mathcal{P}}
\newcommand{\calPn}[0]{\mathcal{P}^{(n)}}
\newcommand{\fa}[1]{\mathfrak{a}_{#1}}
\newcommand{\cal}[1]{\mathcal{#1}}
\newcommand{\zsharp}[0]{\mathbb{Z}_{\sharp}}
\newcommand{\bS}[0]{\overleftarrow{\mathfrak{S}}}
\renewcommand{\ss}[0]{\mathfrak{S}}
\newcommand{\sz}[0]{S_{\mathbb{Z}}}
\DeclareMathOperator{\des}{Des}
\DeclareMathOperator{\bb}{{\bf B}}
\DeclareMathOperator{\ff}{{\bf F}}
\DeclareMathOperator{\qq}{\mathbb{Q}}
\DeclareMathOperator{\rr}{\mathbb{R}}
\DeclareMathOperator{\cc}{\mathbb{C}}
\DeclareMathOperator{\xx}{{\bf{x}}}
\DeclareMathOperator{\ms}{\mathfrak{S}}
\DeclareMathOperator{\mf}{\mathfrak{F}}
\DeclareMathOperator{\bR}{\overleftarrow{R}}
\DeclareMathOperator{\s}{span}
\DeclareMathOperator{\Hom}{Hom}
\DeclareMathOperator{\End}{End}
\DeclareMathOperator{\img}{Img}
\newcommand{\calM}[0]{\mathcal{M}}
\DeclareMathOperator{\id}{id}
\renewcommand{\emptyset}{\varnothing}
\newcommand{\rtile}[0]{
\begin{tikzpicture}[scale=0.6, tikzfig, baseline = -0.5 em]
	\begin{pgfonlayer}{nodelayer}
		\node [style=none] (204) at (15.25, -1) {};
		\node [style=none] (205) at (15.25, 0.25) {};
		\node [style=none] (206) at (16.5, 0.25) {};
	\end{pgfonlayer}
	\begin{pgfonlayer}{edgelayer}
		\draw [style=red, rounded corners=0.2cm] (204.center) to (205.center)  to (206.center);
	\end{pgfonlayer}
\end{tikzpicture}
}

\newcommand{\jtile}[0]{
\begin{tikzpicture}[scale=0.6, tikzfig, baseline = 0.5 em]
	\begin{pgfonlayer}{nodelayer}
		\node [style=none] (201) at (14.25, 0.25) {};
		\node [style=none] (202) at (14.25, 1.5) {};
		\node [style=none] (203) at (13, 0.25) {};
	\end{pgfonlayer}
	\begin{pgfonlayer}{edgelayer}
		\draw [style=red, rounded corners=0.2cm] (203.center) to (201.center) to (202.center);
	\end{pgfonlayer}
\end{tikzpicture}
}

\newtheorem{theorem}{Theorem}[section]
\newtheorem{lemma}[theorem]{Lemma}
\newtheorem{proposition}[theorem]{Proposition}
\newtheorem{corollary}[theorem]{Corollary}

\theoremstyle{remark}
\newtheorem{remark}[theorem]{Remark}
\newtheorem{example}[theorem]{Example}
\newtheorem{definition}[theorem]{Definition}


%%augmented matrix
\makeatletter
\renewcommand*\env@matrix[1][*\c@MaxMatrixCols c]{%
  \hskip -\arraycolsep
  \let\@ifnextchar\new@ifnextchar
  \array{#1}}
\makeatother

\numberwithin{equation}{section}


\title{MATH 4242 Applied Linear Algebra}
%\author[M. Shimozono]{Mark Shimozono$^\flat$}
%\thanks{$^\flat$\url{mshimo@math.vt.edu} Virginia Tech.}
\author[S. Zhang]{Sylvester W. Zhang$^\natural$}
\thanks{$^\natural$\url{swzhang@umn.edu} University of Minnesota.}

\date{Summer 2024}

\begin{document}
\maketitle	

\tableofcontents

\section{Systems of Linear Equations}
A $m\times n$ system of linear equation is of the form
\begin{align*}
	a_{11}x_1+\cdots +a_{n1}x_n&=b_1\\
	a_{21}x_1+\cdots +a_{n2}x_n&=b_n\\
	\cdots\quad \cdots \quad\cdots&\\
	a_{m1}x_1+\cdots+a_{mn}x_n&=b_n
\end{align*}
Such equation can be represented using product of matrices.
\[\begin{bmatrix}
	a_{11}&a_{21}&\cdots &a_{m1}\\
	a_{21}&a_{22}&\cdots &a_{m2}\\
	\cdots&\cdots&\cdots &\cdots\\
	a_{m1}&a_{m2}&\cdots &a_{mn}
\end{bmatrix} \begin{bmatrix}
	x_1\\x_2\\\vdots\\x_n
\end{bmatrix} = \begin{bmatrix} b_1\\b_2\\ \vdots \\b_n \end{bmatrix}\]
or by an augmented matrix.

\[\begin{bmatrix}[cccc|c]
	a_{11}&a_{21}&\cdots &a_{m1}&b_1\\
	a_{21}&a_{22}&\cdots &a_{m2}&b_2\\
	\cdots&\cdots&\cdots &\cdots&\\
	a_{m1}&a_{m2}&\cdots &a_{mn}&b_n
\end{bmatrix} \]
\begin{definition}
	We have three types of elementary row operations.
	\begin{enumerate}
		\item Multiply the $i$-th equation (or the $i$-th row of the augmented matrix), then add it to the $j$-th equation (or the $j$-th row of the augmented matrix).
		\item Permute the equations (or the rows of the augmented matrix)
		\item Multiply one equation (or one row of the augmented matrix) by a non-zero number.
	\end{enumerate}
\end{definition}

\subsection{Systems of $n\times n$ Equations.}
Matrices considered in this sections are all $n\times n$.
\begin{definition}
	A matrix is \emph{regular} if it can be turned into a upper triangular matrix such that every entry on the diagonal is non-zero.
\end{definition}
%\begin{definition}
%	Define an \emph{elementary lower-triangular matrix} to be a matrix with $1$'s on the diagonal and only one non-zero entry in the lower triangular part.
%\end{definition}
\begin{proposition}
	Let $E$ be the matrix with $1$'s on the diagonal and $E_{ij}=k\neq 0$ is the only other non-zero entry in the lower triangular part. Then for any matrix $M$, $EM$ is the matrix obtained by multiplying the $j$-th row of $M$ then adding to the $i$-th row of $M$.
\end{proposition}

\begin{proposition}
	A matrix $A$ is regular if and only if it has an $LU$ factorization, i.e.
	\[A=LU\]
	where $L$ is a lower uni-triangular matrix, and $U$ is a upper triangular matrix with non-zero diagonal entries.
\end{proposition}

\begin{definition}
	Let $w\in S_n$ be a permutation, then define $P_w=\{a_{ij}\}$ to be the matrix such that $$a_{i,j}=\begin{cases}
		1&j=w(i) \\ 0&  \text{otherwise.}
	\end{cases}$$
\end{definition}

\begin{proposition}
	For any matrix $M$, $P_wM$ is the matrix obtained by permuting the rows of $M$ according to the permutation $w$.
\end{proposition}

\begin{definition}\label{def:non-singular}
	A matrix $A$ is called \emph{non-singular} if it can be turned into a upper triangular matrix without non-zero diagonal entry via row operations of the first two types.
\end{definition}
\begin{proposition}
	A matrix $A$ is non-singular if and only if it has a permuted LU factorization:
	$PA=LU$ where $P$ is some permutation matrix.
\end{proposition}
\begin{definition}\label{def:transpose}
	Let $A=(a_{ij})$, defined transpose of $A$ to be $A^t:=(a_{ji})$.
\end{definition}
\begin{proposition}	Denote $A^t$ the transpose of $A$. We have that $AB = (BA)^t$.
\end{proposition}

\begin{proposition}
	A matrix $A$ is regular iff it admits an LDV factorization, $A=LDU$ where $L$ is lower-unitriangular matrix, $D$ is a diagonal matrix, and $U$ is a uni-upper triangular matrix.
\end{proposition}
\begin{definition}
	Let $A$ be an $n\times n$ matrix. Suppose $X$ is a matrix such that $XA=AX=I$ where $I$ is the identity matrix. Then $X$ is called the inverse of $A$ and denoted by $A^{-1}$. A matrix is called \emph{invertible} if $A^{-1}$ exists.
\end{definition}
\begin{proposition}
	A matrix is invertible if and only if it is non-singular.
\end{proposition}
\begin{remark}
	Inverse of a matrix can be found using Gauss–-Jordan Elimination --- see chapter 1 of Olver-Shakiban.
\end{remark}
	

\subsection{Systems of $m\times n$ Equations.}
\begin{definition}
A matrix is in row echelon form if it looks like,
	\[
\begin{pmatrix}
\bullet & * & * & * & * & * \\
0 & \bullet & * & * & * & * \\
0 & 0 & 0 & \bullet & * & * \\
0 & 0 & 0 & 0 & 0 & 0 \\
\end{pmatrix}
\]
where $\bullet$'s are non-zero entries (called \emph{pivots}) and $*$ represent generic entries. The pivots are the first non-zero entries in each rows. We require the pivots occupy the first several rows consecutively.
\end{definition}
\begin{proposition}
	Every matrix can be turned into a row echelon form using elementary row operations of type I and II. In other words, every matrix $A$ has a factorization $PA=LU$ where $P$ is a permutation matrix, $L$ is a lower uni-triangular matrix, and $U$ a matrix in row-echelon form.
\end{proposition}
\begin{definition}\label{def:rank}
	Since every matrix can be turned in to row-echelon form using elementary row operations, we define its \emph{rank} to be the number of pivots.
\end{definition}
\begin{proposition}
	A square $n\times n$ matrix is non-singular if its rank is $n$ (full-rank).
\end{proposition}



\section{Vector Spaces}
\subsection{Some Basic Setup}
\begin{definition}\footnote{You don't need to worry too much about the abstract structures of a field. The purpose of this definition is to make everything self-contained. You can basically think of a field as a set on which you can do some sort of arithmetic.}
	A field is a set $\fff$ with two binary operations $\times$ (multiplication) and $+$ (addition), satisfying the following axioms.
	\begin{itemize}
	    \item $a+b=b+a$ and $a\times b = b\times a$ for all $a,b\in\fff$.
		\item There exists an additive identity $0$ such that $0+a=a+0 = a$ for all $a\in\fff$.
		\item There exists a multiplication identity $1$ such that $1\times a= a\times 1=a$ for all $a\in\fff$.
		\item For every $a\in\fff$, there exists an element denoted $-a$, such that $a+(-a)=0$.
		\item $0\neq 1$.
		\item For every $a\in\fff$ and $a\neq 0$, there exists an element denoted $a^{-1}$, such that $a\times (a^{-1})=1$.
		\item For every $a,b,c\in\fff$, $a\times (b+c)=ab+ac$.
	\end{itemize}
For most part of this class, we will take $\fff=\rr $  or $\fff=\cc=\{a+bi|a,b\in\rr\text{ and }i^2 = -1\}$.
\end{definition}
\begin{definition}
	For a field $\fff$, denote $\fff[x]$ the ring\footnote{A ring is a field, where multiplication need not to be commutative, and multiplicative identity ($0$) need not exists.}of polynomials over $\fff$. 
	\[\fff[x]=\{a_0+a_1x+a_2x^2+\cdots a_nx^n|a_0,\cdots,a_n\in \fff,n\geq 0, x^mx^n=x^{m+n}\}\]\end{definition} 
\begin{proposition}\label{prop:C_alg_close}
	Every polynomial $a_0+a_1x+a_2x^2+\cdots+a_nx^n = 0$ with complex coefficient has at least one complex solution. Note that this is not true for real polynomials.
\end{proposition}
\begin{definition}
	A field $\fff$ is called \emph{algebraically closed} if every polynomial in $\fff[x]$ has a solution in $\fff$. (By \Cref{prop:C_alg_close}, $\cc$ is algebraically closed).
\end{definition}
\begin{proposition}
	The field of complex numbers $\cc$ is the algebraic closure of $\rr$. In other words, $\cc$ is the smallest algebraically closed field that contains $\rr$.
\end{proposition}
\subsection{Vector Spaces and Subspaces}
Let $\fff$ be a field.
\begin{definition}\label{def:vs}
	A set $V$ is called a vector space over $\fff$ if there exists an addition map
	\[add:V\times V\to V\]
	%\[(v_1,v_2)\mapsto v_1+v_2\]
	and a scalar multiplication map
	\[mult:\fff\times V\to V\]
	%\[(a,v)\mapsto av\]
	(Here $\times$ denote the Cartesian product of sets\footnote{For sets $A$ and $B$, defined $A\times B=\{(a,b)|a\in A,b\in B\}$}.) We will abbreviate them by $a(v_1,v_2)=v_1+v_2$ and $mult(a,v)=av$.
	
	Note that this definition (implicitly) requires that a vector space $V$ is closed under addition and scalar multiplication, i.e. $v_1+v_2=add(v_1,v_2)\in V$ and $av = mult(a,v)\in V$.
	
	Elements of a vector spaces are called \emph{vectors.}
\end{definition}

\begin{definition}\label{def:subspace}
	Let $V$ be a vector space over $\fff$. A subset $U$ of $V$ is a \emph{subspace} if it is closed under addition and scalar multiplication. (In other words, a subspace is a subset that is a vector space itself.)
\end{definition}
\begin{definition}\label{def:sum}
	Let $U_1,\cdots,U_m$ be subspaces of $V$. Then define their sum to be
	\[U_1+\cdots+U_m = \{u_1+\cdots+u_m|u_1\in U_1,\cdots,u_m\in U_m\}\]
\end{definition}
\begin{proposition}
	Let $U_1,\cdots,U_m$ be subspaces of $V$. Then $U_1+\cdots+U_m$ is also a subspace of $V$, furthermore, it's the smallest subspace of $V$ that contain all of $U_1,\cdots,U_m$.
\end{proposition}
\begin{definition}
	A sum of subspaces $U_1+\cdots+U_m$ of $V$ is a \emph{direct sum} if every vector $v\in U_1+\cdots+U_m$ can be uniquely written as $v=u_1+\cdots+u_m$ where $u_i\in U_i$ for each $i$. When a summation is direct, we denote it as $U_1\oplus\cdots\oplus U_m$.
\end{definition}
\subsection{Linear Combination, Span, and Dimension}
Let $V$ be a vector space over $\fff$.
\begin{definition}
	Let $v_1,v_2,\cdots,v_n\in V $, a vector $v\in V$ is a linear combination of $\{v_1,\cdots,v_n\}$ if there exists $a_1,\cdots,a_n\in\fff$ such that 
	\[v=a_1v_1+\cdots+a_nv_n\]
\end{definition}
\begin{definition}\label{def:lin_comb}
	Let $v_1,v_2,\cdots,v_n$ be a list of vectors in $V$, define their span to be the set of all linear combinations of $v_1,\cdots, v_n$.
	\[ \s(v_1,\cdots,v_n)=\{a_1 v_1+\cdots+a_nv_n|a_1,\cdots,a_n\in\fff \}\]
\end{definition}
\begin{proposition}
	For a list of vectors $v_1,\cdots,v_n\in V$, $\s(v_1,\cdots,v_n)$ is a subspace of $V$. Furthermore, it's the smallest subspace containing all of $v_1,\cdots,v_n$.
\end{proposition}
\begin{definition}
	A vector space $V$ is said to be \emph{finite dimensional} it it is the span of a finitely many vectors.
\end{definition}

\begin{definition}\label{def:linear_ind}
	$v_1,\cdots,v_m\in V$ are \emph{linearly independent} if the only way to write $0$ as a linear combination of $v_1,\cdots,v_n$ is
	\[0=0v_1+0v_2+\cdots+0v_n.\]
\end{definition}

\begin{proposition}
	$v_1,\cdots,v_m\in V$ are \emph{linearly independent} if and only if any vector $v\in \s(v_1,\cdots,v_m)$ can be uniquely written as a linear combination of $v_1,\cdots,v_n$.
\end{proposition}
\begin{definition}
	A list of vectors $v_1,\cdots,v_n$ is a basis of $V$ if
	\begin{itemize}
		\item $V=\s(v_1,\cdots,v_n)$
		\item $v_1,\cdots,v_n$ are linearly independent.
	\end{itemize}
\end{definition}
\begin{proposition}
	$v_1,\cdots,v_n$ is a basis of $V$ iff every vector $v\in V$ can be uniquely written as a linear combination of $v_1,\cdots,v_n$.
\end{proposition}
\begin{lemma}Let $v_1,\cdots,v_m\in V$ be a list of vectors that spans $V$, i.e. $\s (v_1,\cdots,v_m)=V$. Then $\{v_1,\cdots,v_m\}$ can be reduced to  a basis of $V$. In other words, there exists a basis $\{w_1,\cdots,w_n\}$ of $V$ such that $w_i\in \{v_1,\cdots,v_m\}$ for all $i$ and $n\leq m$.
\end{lemma}
\begin{lemma}
	Let $v_1,\cdots,v_k\in V$ be linearly independent. Then there exists a basis of $V$ in the form
	\[\{v_1,\cdots,v_k,w_1,\cdots,w_m\}\]
	Note that it's possible that $m=0$, in the case when $\{v_1\cdots v_k\}$ is already a basis.
\end{lemma}
\begin{corollary}
	If $U$ is a subspace of $V$, then there exists another subspace $W$ such that $V=U\oplus W$.
\end{corollary}
\begin{proposition}
	If $v_1,\cdots,v_n$ is a basis of $V$ and $w_1,\cdots,w_m$ is another basis of $V$. Then $n=m$.
\end{proposition}
\begin{definition}
	Define the dimension of a vector space to be the size of its basis.
\end{definition}
\begin{proposition}
	If $\{v_1,\cdots,v_n\}$ linearly independent and $n=\dim(V)$, then $\{v_1,\cdots,v_n\}$ is a basis.
\end{proposition}
\begin{proposition}
	If $U$ is a subspace of $V$, then $\dim(U)\leq \dim(V)$. Furthermore, $\dim(U)=\dim(V)$ iff $U=V$.
\end{proposition}
\begin{proposition}
	If $\s(v_1,\cdots,v_n)=V$ and $n=\dim(V)$, then $\{v_1,\cdots,v_n\}$ is a basis.
\end{proposition}
\begin{theorem}
	Let $V$ be a finite dimensional vector space and $V_1,V_2$ subspaces. Then
	\[\dim(V_1+V_2)=\dim(V_1)+\dim(V_2)-\dim(V_1\cap V_2)\]
\end{theorem}
\begin{corollary}
	If $V_1+V_2$ is a direct sum, then $\dim(V_1\oplus V_2)=\dim(V_1)+\dim(V_2)$.\footnote{We will see later that the converse is also true.}
\end{corollary}

\section{Linear Maps and Matrices}
\subsection{Linear Maps}
Let $V,W$ be vector spaces over $\fff$.

\begin{definition}
	A map $T:V\to W$ is linear if
	\begin{enumerate}
		\item $T(u+v)=T(u)+T(v)$ for all $u,v\in V$.
		\item $T(\lambda v)=\lambda T(v)$ for all $\lambda\in\fff$ and $v\in V$.
	\end{enumerate}
\end{definition}
\begin{definition}
	We denote the set of all linear maps from $V\to W$ by $\Hom(V,W)$. And define $\End(V)=\Hom(V,V)$. \end{definition}
	\begin{lemma}
		Let $v_1,\cdots,v_n$ be a basis for $V$ and $w_1,\cdots,w_n$ a basis for $W$ (i.e. $V,W$ same dimension). Then there exists a unique linear map $T\in\Hom(V,W)$ such that $T(v_i)=w_i$ for all $i$. The map is given by $T(a_1 v_1+\cdots+a_n v_n)=a_1w_1+\cdots+a_nw_n$.
	\end{lemma}
	\begin{proposition}
		The set $\Hom(V,W)$ is a vector space over $\fff$, with addition and scalar multiplication given as follows.
		\[(\varphi+\psi)(v):=\varphi(v)+\psi(v)\]
		\[(\lambda \varphi)(v):=\lambda \varphi(v)\]
	\end{proposition}
%	\begin{proposition}
%		Compositions of linear maps have the following properties. 
%		\[(T_1T_2)T_3 = T_1(T_2T_3)\]
%		\[(T_1+T_2)T_3=T_1T_3+T_1T_2\]
%	\end{proposition}
	\begin{lemma}
		Let $T\in\Hom(V,W)$, then $T(0_V)=0_W$.
	\end{lemma}
	%\subsection{Fundamental Subspaces}
	Let $T\in\Hom(V,W)$.
	
	\begin{definition}
		The \emph{kernal} (or null space) of $T$ is $\ker(T)=\{v\in V:Tv=0\}$
	\end{definition}
	\begin{proposition}
		$\ker(T)$ is a subspace of $V$.
	\end{proposition}
	\begin{proposition}
		$\ker(T)=\{0\}$ if and only if $T$ is injective. 
	\end{proposition}
	\begin{definition}
		The \emph{image} (or range) of $T$ is $\img(T)=\{Tv|v\in V\}$
	\end{definition}
	\begin{proposition}
		$\img(T)$ is a subspace of $W$.
	\end{proposition}
	\begin{proposition}
		$T$ is surjective iff $\img(T)=W$.
	\end{proposition}
	\begin{theorem}
		$\dim(V)=\dim(\ker(T))+\dim(\img(T))$.
	\end{theorem}
	\begin{proposition}
		\begin{enumerate}
			\item if $\dim(V)>\dim(W)$, then any $T\in\Hom(V,W)$ is not injective.
			\item if $\dim(V)<\dim(W)$, then any $T\in \Hom(V,W)$ is not surjective.
			\item if there exists a bijective $T\in\Hom (V,W)$, then $\dim(V)=\dim(W)$.
		\end{enumerate}
	\end{proposition}
	\subsection{Matrices from Linear Maps}Denote the set of all $m\times n$ matrix with entries in $\fff$ by $M_{m\times n}(\fff)$.
	Let $V,W$ be finite dimensional vector spaces over $\fff$.
	
	\begin{definition}
	Suppose $V$ has basis $v_1,\cdots,v_n$ and $W$ has basis $w_1,\cdots,w_m$.
		Let $T\in\Hom(V,W)$. Then define $\calM(T)$ to be the matrix $[a_{ij}]$
		such that $$T(v_k)=a_{1k}w_1+a_{2k}w_2+\cdots+a_{m,k}w_m.$$%
	\end{definition}
	\begin{remark}
		Note that the usage of $\calM$ requires a choice of basis for $V$ and $W$. In general we shall denote $\calM_{B_1,B_2}(T)$ where $B_1$ is the basis for $V$ and $B_2$ the basis for $W$. However in most case we will omit the subscript when the context is clear.
	\end{remark}
	\begin{proposition}
		Let $S,T\in\Hom(V,W)$, 
		\(\calM(S)+\calM(T)=\calM(S+T)\)
		
		Let $S\in\Hom(U,W)$ and $T\in\Hom(V,U)$, then $\calM(S)\calM(T)=\calM(ST)$.
	\end{proposition}
	\begin{definition}
		For any $A\in M_{m\times n}$, Let $A_{\bullet,k}$ denote the $k$-th column vector and $A_{k,\bullet}$ denote the $k$-th row vector.	
	\end{definition}
	\begin{proposition}
		$(AB)_{\bullet,k}=A (B_{\bullet,k})$
	\end{proposition}
	\begin{theorem}\label{thm:col-row-dim}
		For any $A\in M_{m\times n}$, we have
		\[\dim(\s(A_{1,\bullet},\cdots,A_{m,\bullet}))=\dim(\s(A_{\bullet,1},\cdots,A_{\bullet,n})) \]
	\end{theorem}


	\begin{proposition}
		The dimension of column span or row span of a matrix equals to its rank (see \Cref{def:rank}).
	\end{proposition}
	\begin{definition}
		A linear map $T\in \Hom(V,W)$ is invertible if there exists an linear map $S\in\Hom(W,V)$ such that $TS=\id_W$ and $ST=\id_V$.
	\end{definition}
	\begin{proposition}
		If a map $T$ is invertible, then its inverse is unique, denote it by $T^{-1}$.
	\end{proposition}
	\begin{proposition}
		A map $T$ is invertible iff $\calM(T)$ is non-singular \Cref{def:non-singular}. Furthermore, $\calM(T^{-1})=\calM(T)^{-1}$.
	\end{proposition}
		
	
	
	\begin{definition}
		A linear map is an \emph{isomorphism} if it's invertible. Two vectors spaces are \emph{isomorphic} if there exists an isomorphism between them.
	\end{definition}
	\begin{theorem}
		Two vector spaces over $\fff$ is isomorphic if and only if they have the same dimension. (In other words, vector spaces are classified by $\mathbb{N}$)
	\end{theorem}
	\begin{corollary}Let $\dim V=n$ and $\dim W=m$.
		The vector space $\Hom(V,W)\cong M_{m\times n}(\fff)$ are isomorphic, with the map $\calM$  being the isomorphism.
	\end{corollary}
	\begin{theorem} Let $V$ be a vector space with basis $B_1=\{v_1,\cdots,v_n\}$. Suppose it has another basis $B_2=\{w_1,\cdots,w_n\}$. Let $C=\calM_{B_1,B_2}(\id)$ where $\id\in\Hom(V,V)$ is the identity map. Then change of basis corresponds to conjugation by $C$.
	
	In particular, let $T\in\Hom(V,V)$ and $A=\calM_{B_1,B_1}(T)$ and $B=\calM_{B_2,B_2}(T)$. Then we have
	\[A=C^{-1}BC\]
	\end{theorem}
	\subsection{Quotient and Dual spaces}
	\begin{definition}
		Let $v\in V$ and $U\subseteq V$. Define  $v+U := \{v+u|u\in U\}$. This is called a \emph{coset}.
	\end{definition}
	\begin{definition}\label{def:quotient_1}
		Let $U\subseteq V$. Define the quotient space $V/U$ to be $\{v+U|v\in V\}$, with addition and scalar multiplication given by
		\[(v_1+U)+(v_2+U) = (v_1+v_2)+U\]
		\[\lambda(v+U)=\lambda v+U\]
	\end{definition}
	\begin{definition}[alternative definition]\label{def:quotient_2} Let $\sim$ be an equivalence relation on $V$. Define $[v]_\sim:=\{u\in V|u\sim v\}$ the equivalence class generated by $v$. Then we can define quotient space $V/\sim:=\{[v]_\sim|v\in V\}$.
	\end{definition}
	\begin{remark}
		For $U\subset V$, define an equivalence relation $\sim_U$ by $v\sim_U u \iff v-u\in U$. Then \Cref{def:quotient_1,def:quotient_2} agree, i.e. $V/U = V/\sim_U$.
	\end{remark}
	\begin{definition}
		For $U\subset V$, define the quotient map $\pi: V\to V/U$ by $\pi(v)=v+U$. Note that $\ker(\pi)=U$.
	\end{definition}
	\begin{proposition}
		$\dim V/U = \dim V-\dim U$.
	\end{proposition}
	\begin{theorem}
		For any $T\in\Hom(V,W)$, define $\tilde T\in\Hom(V/\ker(T),W)$ by  $\tilde T(v+\ker(T))=Tv$. Then $\tilde T\pi = T$, and defines an isomorphism between $V/\ker(T)$ and $\img(T)$.
	\end{theorem}
	
	\begin{definition}
		A linear map from $V$ to $\fff$ is called a \emph{linear functional}. Denote $\Hom(V,\fff)$ the set of all linear functionals on $V$.
	\end{definition}
	\begin{proposition}
		$\Hom(V,\fff)$ is a vector space, with addition and multiplication given by $(f+g)(v)=f(v)+g(v)$ and $(\lambda f)(v)=\lambda f(v)$. This is called the \emph{dual space} of $V$, and is denoted by $T^*$. 
	\end{proposition}
	\begin{proposition}
		$\dim V=\dim V^*$.
	\end{proposition}
	\begin{definition}
		Let $v_1,\cdots,v_n$ be a basis of $V$. Then define $v_i^*\in V^*$ to be the linear functional $v_i^*(v_j)=\delta_{i,j}$\footnote{Here $\delta_{i,j}$ is the Kronecker delta symbol: $\delta_{i,j}=1$ if $i=j$ and $0$ otherwise.}. For any $v=a_1v_1+\cdots a_nv_n\in V$, define $v^*=a_1v_1^*+\cdots+a_nv_n^*$.
	\end{definition}
	\begin{proposition}Let $v_1,\cdots,v_n$ be a basis. Then
		$v=v_1^*(v)v_1+\cdots+v_n^*(v)v_n$ for all $v\in V$.
	\end{proposition}
	\begin{proposition}
		$v_1^*,\cdots,v_n^*$ is a basis for $V^*$.
	\end{proposition}
	\begin{definition}
		Suppose $T\in \Hom(V,W)$. Define the \emph{dual linear map} $T^*\in\Hom(W^*,V^*)$ to be
		\[T^*(f) = f\circ T\]
	\end{definition}
	\begin{proposition}
	\begin{itemize}
		\item $(S+T)^* = S^*+T^*$
		\item $(\lambda T)^*=\lambda T^*$.
		\item $(ST)^*=T^*S^*$.
	\end{itemize}
		
	\end{proposition}
	\begin{definition}
		For any subspace $U\subseteq V$, define its annihilator $U^0:=\{f\in V^*: f(u)=0\text{ for all } u\in U\}$.
	\end{definition}
	\begin{proposition}
		$U^0$ is a subspace of $V^*$.
	\end{proposition}
	\begin{proposition}
		$\dim U^0=\dim V-\dim U$. Recall that this is also the dimension of $V/U$. In particular, there is an isomorphism $(V/U)^*\cong U^0$ given by $\pi^*$.
	\end{proposition}
	
	\begin{proposition}
		\textup{(a)} $U^0=\{0\}\iff U=V$ \textup{(b)} $U^0=V^*\iff U=\{0\}$.
	\end{proposition}
	\begin{theorem}
		\textup{(a)} $(\img T)^0=\ker T^*$ \textup{(b)} $(\ker T)^0=\img T^*$
	\end{theorem}
	\begin{corollary}
		$T$ is injective iff $T^*$ is surjective. $T$ is surjective iff $T^*$ is injective.
	\end{corollary}
	\begin{theorem}
		Let $T\in \Hom(V,W)$, and $T^*\in\Hom(W^*,V^*)$. Then $\calM(T)^t=\calM(T^*)$.
	\end{theorem}
	\section{Inner Product Spaces}Throughout this section, let $\fff=\rr$ or $\cc$.
	\subsection{Inner Products and Norms}
	\begin{definition}
		The dot product of two vectors in $\rr^n$ is a map from $\rr^n\times \rr^n\to \fff$, defined by
		\[(x_1,\cdots,x_n)\cdot (y_1,\cdots,y_n)=x_1y_1+\cdots+x_ny_n\]
	\end{definition}
	
%	\begin{definition}
%		A
%	\end{definition}
	
	
	\begin{definition}
		The dot product of two vectors in $\cc^n$ is a map from $\cc^n\times \cc^n\to \fff$, defined by
		\[(x_1,\cdots,x_n)\cdot (y_1,\cdots,y_n)=x_1\overline y_1+\cdots+x_n\overline y_n\]
		where $\overline{a+bi}=a-bi$ is the complex conjugate.
	\end{definition}
	\begin{definition}\label{def:inner_product}Let $V$ be vector space over $\fff$ ($\cc$ or $\rr$).
		A inner product on $V$ is a map $V\times V\to \fff$ which sends $(v,u)$ to $\langle v,u\rangle$ such that
		\begin{enumerate}
			\item $\langle v,v\rangle\geq 0$.
			\item $\langle v,v\rangle=0\iff v=0$.
			\item $\langle u+v,w\rangle=\langle u,w\rangle+\langle v,w\rangle$
			\item $\langle \lambda u,v\rangle=\lambda \langle u,v\rangle$.
			\item $\langle u,v\rangle=\overline{\langle v,u\rangle }$\footnote{When $\fff=\rr$, the `complex' conjugate of a real number is just itself.}
		\end{enumerate}
	\end{definition}
	\begin{proposition}[Bilinearity]\label{prop:bilinear} A inner product $\langle\ ,\ \rangle$ satisfy $\langle u,v+w\rangle = \langle u,v\rangle + \langle u,w\rangle$ and $\langle u,\lambda v\rangle =\overline{\lambda}\langle u,v\rangle$.
	\end{proposition}
	\begin{remark}
		A pairing satisfying (4) and (6) of \Cref{def:inner_product} and \Cref{prop:bilinear} together is known as being \emph{bilinear}. Usually a inner product is defined to be bilinear, however, as we see here one-sided linearity is enough to imply bilinearity.
	\end{remark}
	\begin{proposition}
		An inner product $\langle\ , \ \rangle$ on $V$ satisfy
		\begin{enumerate}
			\item Fix any $u\in V$, the map $v\mapsto \langle u,v\rangle$ is a linear functional.
			\item $\langle v,0\rangle =0= \langle 0,v\rangle $ for any $v\in V$.
		\end{enumerate}
	\end{proposition}
	\begin{definition}
		Given an inner product $\langle \ ,\ \rangle$, define the norm $\Vert\ \Vert$ to be the positive square-root $\Vert v\Vert=\sqrt{\langle v,v\rangle}$.
	\end{definition}
	
	\begin{proposition}
		Let $I=[a,b]\subset \rr$ be an closed interval on $\rr$. Let $V=C^0(I)$ denote all continuous $\rr$-valued functions defined on $I$ (domain is $I$). Then
		\[\langle f,g\rangle =\int_a^bf(x)g(x)\ dx\]
		defines an inner product on $V$. The norm $\Vert f\Vert=\sqrt{\langle f,g\rangle}$ is called the $L_2$ norm on $C^0(I)$.
	\end{proposition}
	\begin{definition}
		For $z\in\cc$, define the complex modulus to be $|z|= \sqrt{z\overline{z}}$. Note that when $z$ is real (i.e. no imaginary part), then $|z|$ is the absolute value.
	\end{definition}
	
	\begin{theorem}[Cauchy-Schwartz inequality]
		Let $V$ be an inner product space with inner product $\langle\ ,\ \rangle$ and norm $\Vert\ \Vert$. Then for any $u,v\in V$, we have
		\[|\langle u,v\rangle| \leq  \Vert u\Vert \Vert v\Vert\]
		Moreover, the equality occurs only when $u,v$ are linearly independent.
	\end{theorem}
	\begin{remark}
		The Cauchy-Schwartz inequality tells us that the ratio $|\langle u,v\rangle |\over \Vert u\Vert \Vert v\Vert$ is in between $-1$ and $1$. Therefore we can define the `abstract' angle between two vectors $v,u$ to be
		\[\theta_{u,v}=\arccos {|\langle u,v\rangle |\over \Vert u\Vert \Vert v\Vert}\]
	\end{remark}
	\begin{definition}\label{def:ortho}
		We say two vectors $u,v$ are orthogonal if $\langle u,v\rangle = 0$.
	\end{definition}
	\begin{proposition}
		If $v,u$ orthogonal in $V$, then $\Vert v\Vert^2+\Vert u\Vert ^2 = \Vert v+u\Vert^2$.
	\end{proposition}
	\begin{remark}
		\Cref{def:ortho} generalizes the usual notion of orthogonality in $\rr^2$ in a sense that when two vectors are orthogonal, then the angle between then is $\theta_{u,v}=\pi/2$.
	\end{remark}
	\begin{theorem}Let $V$ be an inner product space, and $u,v\in V$. Then
		$\Vert u+v\Vert\leq \Vert u\Vert +\Vert v\Vert$.
	\end{theorem}
	
	\begin{definition}
		We can define norms more generally without requiring an inner product. A norm on $V$ is a map $\Vert\cdot \Vert :V\to \mathbb{R}_{\geq 0}$ such that
		\begin{itemize}
			\item $\Vert v\Vert\geq 0$ and $\Vert v\Vert =0$ only when $v=0$.
			\item $\Vert \lambda v \Vert = \vert\lambda\vert \Vert v\Vert$
			\item $ \Vert v+u\Vert \leq \Vert v\Vert +\Vert u\Vert$
		\end{itemize}
	\end{definition}
	\begin{theorem}
		Any norm $\Vert \cdot \Vert $ on $\fff^n$ induces a norm on $\End(V)\cong  M_{n\times n}(\fff)\cong \fff^{m n}$.
		For $T\in\End(V)$, define 
		\[\Vert T\Vert = \max\{\Vert Tv\Vert :\Vert v\Vert = 1\}\] 
		called the natural norm.
	\end{theorem}	
	\begin{proposition}
		For $T,S\in\End(V)$ and $v\in V$, we have $\Vert Tv\Vert \leq \Vert T\Vert \Vert v\Vert $.
	
	And $\Vert ST\Vert \leq\Vert S\Vert \Vert T\Vert$.
	\end{proposition}
	\subsection{Orthonormal Basis}
	\begin{definition}Let $V$ be a real or complex inner product space.
		A basis $v_1,\cdots,v_n$ of $V$ is called \emph{orthonormal} if $\langle v_i,v_j\rangle =\delta_{i,j}$ and $\Vert v_i\Vert = 1$ for all $i$. 
	\end{definition}
	\end{document}
