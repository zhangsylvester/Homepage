\documentclass{amsart}
\newcommand{\ff}[0]{\mathbb{F}}
\newcommand{\rr}[0]{\mathbb{R}}
\newcommand{\cc}[0]{\mathbb{C}}

\DeclareMathOperator{\s}{span}
\usepackage[margin = 1in]{geometry}
\theoremstyle{definition}
\newtheorem{prob}{Problem}
\begin{document}
\leftline{{\bf MATH 4242} \hfill Name: \underline{\quad\quad\quad\quad\quad}}

\leftline{{\bf Summer 2024} \hfill Student ID: \underline{\quad\quad\quad\quad\quad}}

\leftline{{\bf Time limit: 115 minites}}



\vspace{2em}

\begin{prob}
	Below are 4 statements about $n\times n$ invertible matrices $A$ and $B$. Circle $T$ for true and $F$ for false statements. Provide a counter-example for the false statements, and a proof for the correct statements. \begin{enumerate}
	\item $(AB)^{-1}=B^{-1}A^{-1}$
	\item $(A+B)$ must be non-singular
	\item $\ker(A)=0$.
	\item 
\end{enumerate}
\end{prob}

\begin{enumerate}
\item Compute the LU factorization of the following matrix. (Hint: Use Gaussian elimination)
\[\begin{bmatrix}
	1&5&0\\3&7&11\\0&-4&2
\end{bmatrix}\]
\item Let $V=\ff_{\leq m}[x]$ be the vector space of polynomials with degree at most $m$. 

(a) Let $U=\{f\in V|f(1)=0\}$. Is $U$ a subspace of $V$?

(b) Let $W=\{f\in V| \text{deg}(f)\text{ is even}\}$. If $W$ a subspace of $V$?
\item Let $V=M_{n\times n}(\rr)$ the vector space of all $n\times n$ matrices. Let $B$ be the set of all upper triangular matrices. Is $B$ a subspace of $V$?
\item Suppose $v_1,\cdots,v_4$ are some vectors in $\rr^4$, and $A$ is the matrix whose columns are $v_1,\cdots,v_4$. Suppose the row echelon form of $A$ is
\[\begin{bmatrix}
	3&1&7&1\\0&-4&8&3\\0&0&0&4\\0&0&0&0
\end{bmatrix}\]

(a) Does $v_1,\cdots,v_4$ form a basis of $\rr^4$?

(b) Is $v_3$ in $\s(v_1,v_2,v_4)$? If so, write $v_3$ as a linear combination of them.

(Hint: You may use: permuted LU factorization; some properties about matrix multiplication and invertible maps. )
\item Let $V=\rr^2$ with basis $v_1 = (1,2)$ and $v_2= (0,1)$. Let $T$ be the map $T(a v_1 +b v_2)=bv_1 + av_2$ for all $a,b\in\rr$.

(a) Show that $T$ is linear;

(b) Find the matrix of $T$;

(c) Find a basis for $\ker(T)$. (Hint: use row echelon form of $\mathcal{M}(T)$. 
\item Let $V=\text{End}({\rr^2})$. Let $W$ be the set of non-invertible linear maps in $V$. Show that $W$ is not a vector space.

(Hint: give an example of two non-invertible maps that sums to an invertible map. You might try to use some theorems to do this in terms of matrices).

\item Let $V=\rr^2$ under the standard basis and $U=\{(x,0):x\in\rr\}$ a subspace of $V$. Describe the quotient space $V/U$. 

(Hint: What is the map whose kernel is $U$?)
%\item Consider $\rr^4$ with the usual inner product and norm. Let $v_1=(1,0,0,-1)$, $v_2=(2,0,0,0)$, $v_3=(1,2,2,1)$. And $W=\s(v_1,v_2,v_3)$.
%
%(a) Find an orthonormal basis of $W$.
%
%(b) Find a basis for the orthogonal complement of $W$.
\item Is the following matrix positive definite?
\[A=\begin{pmatrix}
	1&-1&3\\-1&3&-1\\3&-1&12
\end{pmatrix}\]
Does $\langle x,y\rangle =x^TAy$ define an inner product on $\rr^3$?
\item Find the value for $a,b$ such that the matrix $A$ is orthogonal.
\[A={1\over 3}\begin{pmatrix}2&-2&b\\1&2&2\\a&1&2
\end{pmatrix}\]
\end{enumerate}

%\noindent Part a (4 pts)

%Prove that both $U$ and $W$ are subspaces of $V$.
%\vspace{6cm}
%
%\noindent Part 2 (6 pts)
%
%Describe their sum $U+W$. Is it a direct sum?

\end{document}