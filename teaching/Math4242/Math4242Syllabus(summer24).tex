\documentclass[11pt, a4paper]{article}
%\usepackage{geometry}
\usepackage[inner=1.5cm,outer=1.5cm,top=2.5cm,bottom=2.5cm]{geometry}
\pagestyle{empty}
\usepackage{enumitem}
\usepackage{graphicx}
\usepackage{fancyhdr, lastpage, bbding, pmboxdraw}
\usepackage[usenames,dvipsnames]{color}
\definecolor{darkblue}{rgb}{0,0,.6}
\definecolor{darkred}{rgb}{.7,0,0}
\definecolor{darkgreen}{rgb}{0,.6,0}
\definecolor{red}{rgb}{.98,0,0}
\usepackage[colorlinks,pagebackref,pdfusetitle,urlcolor=darkblue,citecolor=darkblue,linkcolor=darkred,bookmarksnumbered,plainpages=false]{hyperref}
\renewcommand{\thefootnote}{\fnsymbol{footnote}}

\pagestyle{fancyplain}
\fancyhf{}
\lhead{ \fancyplain{}{MATH 4242 (Section 001)} }
\chead{ \fancyplain{}{Course Syllabus} }
\rhead{ \fancyplain{}{Summer 2023} }
%\rfoot{\fancyplain{}{page \thepage\ of \pageref{LastPage}}}
\fancyfoot[RO, LE] {page \thepage\ of \pageref{LastPage} }
\thispagestyle{plain}

%%%%%%%%%%%% LISTING %%%
\usepackage{listings}
\usepackage{caption}
\DeclareCaptionFont{white}{\color{white}}
\DeclareCaptionFormat{listing}{\colorbox{gray}{\parbox{\textwidth}{#1#2#3}}}
\captionsetup[lstlisting]{format=listing,labelfont=white,textfont=white}
\usepackage{verbatim} % used to display code
\usepackage{fancyvrb}
\usepackage{acronym}
\usepackage{amsthm}
\VerbatimFootnotes % Required, otherwise verbatim does not work in footnotes!



\definecolor{OliveGreen}{cmyk}{0.64,0,0.95,0.40}
\definecolor{CadetBlue}{cmyk}{0.62,0.57,0.23,0}
\definecolor{lightlightgray}{gray}{0.93}



\lstset{
%language=bash,                          % Code langugage
basicstyle=\ttfamily,                   % Code font, Examples: \footnotesize, \ttfamily
keywordstyle=\color{OliveGreen},        % Keywords font ('*' = uppercase)
commentstyle=\color{gray},              % Comments font
numbers=left,                           % Line nums position
numberstyle=\tiny,                      % Line-numbers fonts
stepnumber=1,                           % Step between two line-numbers
numbersep=5pt,                          % How far are line-numbers from code
backgroundcolor=\color{lightlightgray}, % Choose background color
frame=none,                             % A frame around the code
tabsize=2,                              % Default tab size
captionpos=t,                           % Caption-position = bottom
breaklines=true,                        % Automatic line breaking?
breakatwhitespace=false,                % Automatic breaks only at whitespace?
showspaces=false,                       % Dont make spaces visible
showtabs=false,                         % Dont make tabls visible
columns=flexible,                       % Column format
morekeywords={__global__, __device__},  % CUDA specific keywords
}

%%%%%%%%%%%%%%%%%%%%%%%%%%%%%%%%%%%%
\begin{document}


\begin{center}
{\Large \textsc{MATH 4242 -- Applied Linear Algebra}}
\end{center}
\begin{center}
Summer 2024
\end{center}


\begin{center}
\rule{6in}{0.4pt}
\begin{tabular}{lllll}
\textbf{Instructor} & Sylvester Zhang & \textbf{Time} & MWF 10:10-11:00, TTh 10:10-12:05 \\
\textbf{Email} &  \href{mailto:swzhang@umn.edu}{swzhang@umn.edu} & \textbf{Location} & Keller 3--111
\end{tabular}

\rule{6in}{0.4pt}
\end{center}
\vspace{.5cm}
\setlength{\unitlength}{1in}
\renewcommand{\arraystretch}{2}

\noindent\textbf{Course Webpage:} \href{https://sylvesterzhang.com/teaching/Math4242}{https://sylvesterzhang.com/teaching/Math4242} (Capital M in the link)
\vskip.15in
\noindent\textbf{Office Hours:} After each lecture in Keller 3-111.
\vskip.15in
\noindent\textbf{Credit:}   This is a 4-credit upper-division Math course. The expected workload is 7.5 hrs lectures + 15 hrs individual study (per week on average).

\vskip.15in
\noindent\textbf{Textbook and References:} %\footnotemark
The primary textbook of this course is:
\begin{itemize}
\item Peter J. Olver and Chehrzad Shakiban, {\textit{Applied Linear Algebra}}, Springer, 2nd ed., 2018. \\ (\url{https://link.springer.com/book/10.1007/978-3-319-91041-3})
\end{itemize}
Other useful reference:
\begin{itemize}[noitemsep]
\item Sheldon Axler, \emph{Linear Algebra Done Right}, 4th ed., 2024. (available \href{https://linear.axler.net/LADR4e.pdf}{online}).
\item Isaiah Lankham, Bruno Nachtergaele and Anne Schilling, \emph{Linear Algebra}, 2016. (available \href{https://math.libretexts.org/Bookshelves/Linear_Algebra/Book%3A_Linear_Algebra_(Schilling_Nachtergaele_and_Lankham)}
{online}).

\end{itemize}

% \footnotetext{Downloadable ebook versions are available on AeLP.}

\vskip.15in
\noindent\textbf{Objectives:}  Linear equations, vector spaces, subspaces, bases, linear transformations, matrices, determinants, eigenvalues, canonical forms, quadratic forms, various decompositions and applications.

\vskip.15in
\noindent\textbf{Prerequisites:}
MATH 2243 or 2373 or 2573, with a grade of at least C-. In other words, prior understanding of linear algebra is expected. This is a 4000-level course, so expect to build upon the linear algebra skills you developed at the 2000-level. Ability to write rigorous mathematical proofs is preferred, but not strictly required.


\vspace*{.15in}

\noindent \textbf{Tentative Course Outline:}
Chapters 1,2,3,4,7,8 of Olver-Shakiban book, chapters 3,8 of Axler book, and selected topics (Markov chains, matrix-tree theorem, flag manifold). See course webpage for the detailed schedule.

\vspace*{.15in}
\noindent\textbf{Grading Policy:} Course grades will be determined by the following components:
\begin{center} 
\begin{minipage}{4in}
\begin{flushleft}
Homework ($6\times 6.66\% = 40\%$) \\
Quizzes ($10\times 2\% = 20\%$)\\
Exams ($2\times 20\%=40  \%$) \\
\end{flushleft}
\end{minipage}
\end{center}
Default grade-line: A: 92--100 B: 85--91.99 C: 75 -- 84.99 D: 65 -- 74.99 F: 0 -- 64.99. (Final grade-line will be adjusted based on performance, but will only be lowered.)
\vskip.15in


\vskip.15in
\noindent\textbf{Important Dates:}
\begin{center} \begin{minipage}{4in}
\begin{flushleft}
%Exam \#1      \dotfill ~June 13, 2024 \\
Exam \#1 (2 hours)     \dotfill ~June 27, 2024  \\
%Exam \#3   \dotfill ~July 11, 2024  \\
Exam \#2 (2 hours)  \dotfill ~July 25, 2024  \\

\end{flushleft}
\end{minipage}
\end{center}


\vskip.15in
\noindent\textbf{Homework:} 
\begin{itemize}[noitemsep]
	\item Homework will be assigned every week (except for the last week) and due on Sundays 11:59 pm.
	\item There are 7 assignments in total, and 1 lowest scores will be dropped. After dropping, each homework will contribute 6.667\% to the final grade.
	\item Late homework assignments are NOT accepted.
	\item Students are allowed, and in fact encouraged, to collaborate on homework problems. But each student must write up homework assignments on their own, demonstrating individual understanding. Students must acknowledge their collaborators in their individual submission.
	\item Students who typeset (\LaTeX, or MarkDown) all homework assignments will receive 5pt extra credits.
	\item Use of AI (such as ChatGPT) is strictly forbidden.
\end{itemize}


\vskip.15in
\noindent\textbf{Exams:}  
\begin{itemize}[noitemsep]
	\item There are 2 exams in total. Each worth the same amount of credit.
	\item Although the exams are not designed to be accumulative, it is expected that every exam requires material from all previous sections.
	\item Exams are closed-book and calculator-free.
\end{itemize}
\vskip.15in
\noindent\textbf{Quizzes:}  
\begin{itemize}[noitemsep]
	\item There are 12 quizzes in total. Two lowest quiz scores will be dropped.
	\item Quiz dates can be found on course website.
	\item Each quiz is about 10 - 15 minutes.
	\item Quizzes are closed-book and calculator-free.
	\item NO make-up quiz is allowed.
\end{itemize}


\vskip.15in
\noindent\textbf{Attendance Policy:}  
Although attendance is not taken into account of the final grade, students are expected to attend all lectures. The MWF sessions are lecture-only, and T/Th sessions consist of a short lecture and a discussion session (in-class activities). In the discussion session, students will work collaboratively on practice problems and homework problems with the help of the instructor. %Attendance to those sessions are of particular importance. %While students are expected to attend class in-person everyday, the course will formally offer a \textit{hybrid option} for those unable to. Attendance will not be taken during class, but this will not make student attendance any less important. The instructor will broadcast each lecture live on Zoom, and a recording of each lecture will be made available on Canvas for future viewing. Those who require Zoom accommodation can access the course Zoom link \href{https://umn.zoom.us/j/97145977169?pwd=V2I3SzkvZE15N1Y3STZtRXZUejkwQT09}{here} and on Canvas. In the event that in-person instruction cannot take place on a given day, the instructor will give advanced notice to all students along with a Zoom link for that day's lecture. \textit{It is recommended that students attend Tuesday \& Thursday lectures in-person as in-class activities will often be given on those days.} Regardless, in-person attendance will be required during Week 1 and on Final Exam Day. Students who need to miss extended time will need to inform the instructor immediately and provide a legitimate excuse.

\vskip.15in
\noindent\textbf{Drop Deadline:}   The schedule for drop deadlines can be found at the following site: \\ \url{https://onestop.umn.edu/dates-and-deadlines/canceladd-deadlines}.

\vskip.15in
\noindent\textbf{Incomplete Grade:}   A course grade of ``Incomplete" (I) will only be granted in extreme circumstances and must be negotiated between the student and the instructor near the end of the term.


\vskip.15in
\noindent\textbf{Disability Accommodations:}   The University of Minnesota is committed to providing equal access to learning opportunities for all
students. The Disability Resource Center (DRC) is the campus office that collaborates with students who have disabilities to provide and/or arrange reasonable accommodations. Information is available on their website \url{https://disability.umn.edu}, by calling (612) 626-1333 (for both voice
and TTY), or by sending an email to \url{drc@umn.edu}. \textit{Students who require DRC accommodations are asked to notify the instructor immediately and provide valid documentation received from their counselor.}

\vskip.15in
\noindent\textbf{University Grading Policies:}   Information on grade accountability and grading policies at the University of Minnesota can be found at the following websites: \\
\url{https://policy.umn.edu/education/gradeaccount} (grade accountability), \\
\url{https://policy.umn.edu/education/gradingtranscripts} (grading policies).

\vskip.15in
\noindent\textbf{Student Conduct and Scholastic Dishonesty:}   The University of Minnesota Student Conduct Code governs all activities at the University, including
this course. Students who engage in behavior that disrupts the learning environment for others may be subject to disciplinary action under the Code. This includes any behavior that substantially or repeatedly interrupts either the instructor’s ability to teach or the learning of other students. The classroom extends to any setting where a student is engaged in work toward academic credit or satisfaction of program-based requirements or related activities. Students responsible for such behavior may be asked to cancel their registration (or have their registration canceled). Scholastic dishonesty includes plagiarizing, cheating on assignments or exams, engaging in unauthorized collaboration on academic work, and taking, acquiring, or using exam materials without faculty permission. Scholastic dishonestly in any portion of the academic work for a course shall be grounds for awarding a grade of (F) or (N) for the entire course. For more information, see the following link: \\ \url{https://regents.umn.edu/sites/regents.umn.edu/files/2022-07/policy_student_conduct_code.pdf}.

\vskip.15in
\noindent\textbf{Academic Freedom and Responsibility:} Academic freedom is a cornerstone of the University. Within the scope and content of the course as defined by the instructor, it includes the freedom to discuss relevant matters in the classroom. Along with this freedom comes responsibility. Students are encouraged to develop the capacity for critical judgement and to engage in a sustained and independent search for truth. Students are free to take reasoned exception to the views offered in any course of study and to reserve judgement about matters of opinion, but they are responsible for learning the content of any course of study for which they are enrolled. 
\\ \\
Reports of concerns about academic freedom are taken seriously, and there are individuals and offices available for help. Contact the instructor, the Department Chair, your adviser, the Associate Dean of the College, or the Vice Provost for Faculty and Academic Affairs in the Office of the Provost. 
\\ \\
(Language adapted from the American Association of University Professors \textit{``Joint Statement on Rights and Freedoms of Students.”})

\vskip.15in
\noindent\textbf{Diversity, Equity, \& Inclusion (DEI):} Please see the following link for the University of Minnesota policy on Equity, Diversity, Equal Opportunity Employment, and Affirmative Action: \\
\url{https://regents.umn.edu/sites/regents.umn.edu/files/2019-09/policy_equity_diversity_equal_opportunity_and_affirmative_action.pdf}

\vskip.15in
\noindent\textbf{Syllabus:} This syllabus will be followed as closely as possible by all parties involved (the students, the instructor). Should any changes to the syllabus need to be made, the instructor will immediately notify students of these changes and will post an updated syllabus accordingly on the course webpage.


%%%%%% THE END 
\end{document} 