\documentclass[12pt]{amsart}
\usepackage[utf8]{inputenc}
\usepackage{setspace}
\usepackage[english]{babel}
\usepackage{enumerate}
\usepackage{thmtools}
\usepackage{thm-restate}
\usepackage[shortlabels]{enumitem}
\usepackage[T1]{fontenc}
\usepackage{tikz}
\usepackage{mathtools}
\usepackage{verbatim}
\usepackage{comment}
\usetikzlibrary{arrows}
\newcommand{\midarrow}{\tikz \draw[-triangle 90] (0,0) -- +(.1,0);}
\newcommand{\miduparrow}{\tikz \draw[-triangle 90] (0,0) -- +(.1,0);}
\usepackage{comment}
\usepackage{graphicx}
\usepackage{color}
\usepackage{etoolbox}
\usepackage[margin=1in]{geometry}
\usepackage{amsmath,amsthm,amssymb,graphicx,tikz,tikz-cd}
\usepackage{hyperref}
\hypersetup{
    colorlinks = true,
    citecolor = orange,%
}
\usepackage{cleveref}
\usepackage{subfig}

\makeatletter
\patchcmd{\@settitle}{\uppercasenonmath\@title}{}{}{}
\patchcmd{\@setauthors}{\MakeUppercase}{}{}{}
\patchcmd{\section}{\scshape}{}{}{}
\makeatother
\makeatletter
\patchcmd{\@maketitle}
  {\ifx\@empty\@dedicatory}
  {\ifx\@empty\@date \else {\vskip2ex %vertical position of date
  \centering\footnotesize\@date\par\vskip1ex}\fi
   \ifx\@empty\@dedicatory}
  {}{}
\patchcmd{\@adminfootnotes}
  {\ifx\@empty\@date\else \@footnotetext{\@setdate}\fi}
  {}{}{}
\makeatother






\theoremstyle{plain}
\newtheorem{theorem}{Theorem}[section]
\newtheorem{restate}{Theorem}[section]
\newtheorem{lemma}[theorem]{Lemma}
\newtheorem{prop}[theorem]{Proposition}
\newtheorem{conj}[theorem]{Conjecture}
\newtheorem{question}[theorem]{Question}
\newtheorem{corollary}[theorem]{Corollary}
\newtheorem{claim}[theorem]{Claim}
\newtheorem{fact}[theorem]{Fact}
\theoremstyle{definition}
\newtheorem{remark}[theorem]{Remark}
\newtheorem{example}[theorem]{Example}
\newtheorem{definition}[theorem]{Definition}





\newcommand{\tld}[1]{\widetilde{#1}}
\renewcommand{\bar}[1]{\overline{#1}}
\newcommand{\ct}[1]{\widehat{#1}}
\newcommand{\acton}{\rotatebox[origin=c]{90}{ $ \circlearrowleft $ }}
\newcommand{\mb}[1]{\mathbb{#1}}
\newcommand{\mc}[1]{\mathcal{#1}}
\newcommand{\Image}{\text{Im}}
\newcommand{\Cr}{\text{cr}}
\newcommand{\Le}{\scalebox{-1}[1]{L}}
\newcommand{\calP}{\mathcal{P}}
\newcommand{\todo}[1]{{\color{red}[To Do: #1]}}


\DeclareMathOperator{\Hom}{Hom}


\title{Math 1271 Practice Midterm 3 }
\author{Section 010 Spring 2021}
\date{\today}

\begin{document}
	\maketitle
	
	\setcounter{tocdepth}{1}
	
\subsection*{Question 1 (section 4.4)} Find the limits
\begin{enumerate}
\item $\displaystyle \lim_{x\rightarrow 0} \frac{\cos 2x-1}{e^{3x}-3x+2}$\vspace{3cm}
\item $\displaystyle \lim_{x\rightarrow \infty} (2x)^{1/\ln x}$\vspace{3cm}
\end{enumerate}
\subsection*{Question 2 (section 4.7)} 
\begin{enumerate}
\item Find the dimensions of a right circular cylinder of maximum volume that can be inscribed in a sphere of radius 10 cm. What is the
maximum volume?\vspace{4cm}

\item Find the shortest distance from the origin to the curve $x^2y^4 = 1$\vspace{3cm}
\end{enumerate}\newpage
\subsection*{Question 3 (section 4.8)}
Use the intermediate Value Theorem to show
that $f(x)=x^3+2x- 4$ has a root between $x=1$ and $x=2$.
Then by using the Newton's method find the root to four decimal places.
\vspace{4cm}
\subsection*{Question 4 (section 4.9)}
\begin{enumerate}
\item Find $f$ if
$\displaystyle f'(x)=2x-3\sin x+\frac{4}{1+x^2}, \ \ f(0)=5$\vspace{4cm}

\item A car braked with a constant deceleration of 16 $ft/s^2$, producing skid marks measuring 200 ft before coming to a stop. How fast was the car traveling when the brakes were first applied?\vspace{4cm}
\end{enumerate}
\subsection*{Question 5 (section 5.1-5.2)}
Consider the integral \[\int_{0}^2(x^2 + x)dx\].
\begin{enumerate}
    \item Give a Riemann sum approximation of the integral with $n=4$ rectangles, taking right endpoints of subintervals.\vspace{3cm}
    \item Evaluate the integral using the definition of the definite integral (that is, taking a limit of Riemann sums).\vspace{3cm}
    \item Now evaluate the integral using the Fundamental Theorem of Calculus.\vspace{3cm}
\end{enumerate}
\vspace{1cm}

For (2) the following identities may be useful: \[\sum_{i=1}^n c = nc \ \ \ \ \ \sum_{i=1}^n i = \frac{n(n + 1)}{2} \ \ \ \ \ \sum_{i=1}^n i^2 = \frac{n(n + 1)(2n + 1)}{6}\]




\vspace{1cm}



\subsection*{Question 6 (section 5.3)}
Evaluate $\displaystyle \int_1^4 \frac{1 + \sqrt{x}e^x - x^3}{\sqrt{x}}dx$.\vspace{3cm}




\vspace{1cm}


\subsection*{Question 7 (other questions from 5.1-3)}
\begin{enumerate}
    \item Do question 5, but with the integral \(\displaystyle\int_{0}^1(1 - 2x^2)dx\)\vspace{4cm}
    \item Do question 5, but with the integral \(\displaystyle\int_{-4}^{5}(-3x + 2)dx\)\vspace{4cm}
    \item Compute the derivative \(\displaystyle\frac{d}{dx}\int_{-x^2}^{\sqrt{x}}\frac{\sin(t)}{1 + t + t^2}dt.\)\vspace{4cm}
    \item Compute the derivative \(\displaystyle\frac{d}{dx}\int_{\ln(x)}^{2 - x}e^t\sin(t)dt.\)\vspace{4cm}
    \item Evaluate \(\displaystyle \int_{-1}^3 |4 - 2x|dx.\)\vspace{4cm}
    \item If $\displaystyle \int_0^{x^2}f(t)dt=x\sin(\pi x)$ and $f$ is continuous, find $f(4)$.
\end{enumerate}\newpage
\subsection*{Question 8 (section 5.2)}
Rewrite the following limit as definite integral, then evaluate it using FTC.
\[\lim_{n\to\infty}\sum_{i=1}^n{1\over n+i^2/n}\]\vspace{4cm}
\subsection*{Question 9 (section 5.4-5)}Evaluate the following definite or indefinite integrals, some of which requires substitution method.
\begin{enumerate}
	\item $\displaystyle \int_{0}^{\pi/2}\frac{\cos^2\theta+1}{\cos^2\theta}\ d\theta$\vspace{3cm}
	\item $\displaystyle \int\frac{x^2}{x^3+1}\ dx$\vspace{3cm}
	\item $\displaystyle \int_e^{e^4}\displaystyle\frac{1}{x\sqrt{\ln{x}}}\ dx$\vspace{3cm}
	\item $\displaystyle \int_0^2x\sqrt{4-x^2}\ dx$\vspace{3cm}
	\item $\displaystyle \int \sin(x)\sin\left(\cos (x)\right)\ dx $\vspace{3cm}
	\item $\displaystyle \int \frac{{(\ln x)}^2}{x} dx$\vspace{3cm}
\end{enumerate}
\subsection*{Question 10 (additional problems)}
\begin{enumerate}
	\item Evaluate the following limit.
	\[\lim_{x\to 2}\frac{e^x}{x-2}\int_2^x \frac{t-1}{t}dt \]
	Hint: $\displaystyle \int_a^af(t)dt=0$, FTC and L'H\^{o}pital’s rule.\vspace{4cm}
	\item Let $\displaystyle f(x)=\int_0^x\left(\int_1^{\sin \theta}\sqrt{1+t^4}\ dt\right)d\theta$. Find the second derivative of $f(x)$.\vspace{4cm}
	
\end{enumerate}
\end{document}