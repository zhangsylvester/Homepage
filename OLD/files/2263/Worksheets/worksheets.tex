\documentclass[12pt]{amsart}%

%packagrs

\usepackage{ifthen}
\usepackage[utf8]{inputenc}
\usepackage{setspace}
\usepackage[english]{babel}
\usepackage{enumerate}
\usepackage{thmtools}
\usepackage[shortlabels]{enumitem}
\usepackage[T1]{fontenc}
\usepackage{tikz}
\usetikzlibrary{shapes.geometric,intersections,decorations.markings,snakes}
\usetikzlibrary{calc,intersections,through,backgrounds}
\usetikzlibrary{patterns}


\renewcommand{\l}[2]{\lambda_{#1#2}}
\newcommand{\bs}[1]{\boldsymbol{#1}}
\newcommand{\tei}[0]{Teichm\"uller}


\DeclareMathOperator{\Hom}{Hom}
\DeclareMathOperator{\psl}{PSL}
\DeclareMathOperator{\rr}{\mathbb{R}}
\DeclareMathOperator{\osp}{Osp}
\DeclareMathOperator{\wt}{wt}
\DeclareMathOperator{\twt}{twt}
\DeclareMathOperator{\crs}{cross}
\DeclareMathOperator{\ber}{Ber}
\DeclareMathOperator{\st}{st}
\DeclareMathOperator{\tp}{t}
\DeclareMathOperator{\id }{id}

\newcommand{\mb}[1]{\mathbb{#1}}
\newcommand{\tld}[1]{\widetilde{#1}}
%
\renewcommand{\arraystretch}{1.45}
\newcommand{\ospmatrix}[9]{
\left(\begin{array}{cc|c}
#1&#2& #3 \\
#4 &#5& #6 
\\
\hline
#7 &#8 &#9
\end{array}
\right)
}

\newcommand{\A}[2]{A\left({#1}\middle| {#2}\right)}

\newcommand{\hookdoubleheadrightarrow}{%
  \hookrightarrow\mathrel{\mspace{-15mu}}\rightarrow
}


\usepackage[]{appendix}
\include{theorems}

\usepackage{comment}
\usepackage{tcolorbox}
\usepackage{xcolor}
\usepackage{pagecolor}

\usepackage{mathtools}
\usepackage{verbatim}
\usepackage{comment}
\usetikzlibrary{arrows}
\tikzset {->-/.style={decoration={markings, mark=at position .5 with {\arrow{latex}}}, postaction={decorate}}}
\tikzset {-->-/.style={decoration={markings, mark=at position .5 with {\arrow[scale=2]{latex}}}, postaction={decorate}}}
\newcommand{\midarrow}{\tikz \draw[-triangle 90] (0,0) -- +(.1,0);}
\newcommand{\miduparrow}{\tikz \draw[-triangle 90] (0,0) -- +(.1,0);}
\newcommand{\midrevarrow}{\tikz \draw[-triangle 90] (0,0) -- +(-.1,0);}
\usepackage{comment}
\usepackage{graphicx}
\usepackage{color}
\usepackage{etoolbox}
\usepackage[margin=1in]{geometry}
\usepackage{amsmath,amsthm,amssymb,graphicx,tikz,tikz-cd}
\usepackage{hyperref}
\hypersetup{
    colorlinks = true,
    citecolor = orange,%
}
\usepackage[noabbrev]{cleveref}
\usepackage{subcaption}

\makeatletter
\patchcmd{\@settitle}{\uppercasenonmath\@title}{}{}{}
\patchcmd{\@setauthors}{\MakeUppercase}{}{}{}
\patchcmd{\section}{\scshape}{}{}{}
\makeatother
\makeatletter
\patchcmd{\@maketitle}
  {\ifx\@empty\@dedicatory}
  {\ifx\@empty\@date \else {\vskip2ex %vertical position of date
  \centering\footnotesize\@date\par\vskip1ex}\fi
   \ifx\@empty\@dedicatory}
  {}{}
\patchcmd{\@adminfootnotes}
  {\ifx\@empty\@date\else \@footnotetext{\@setdate}\fi}
  {}{}{}
\makeatother


\renewcommand{\l}[2]{\lambda_{#1,#2}}
%\newcommand{\d }[0]{{\ d}}

\theoremstyle{plain}
\newtheorem{theorem}{Theorem}[section]
\newtheorem{restate}{Theorem}[section]
\newtheorem{lemma}[theorem]{Lemma}
\newtheorem{prop}[theorem]{Proposition}
\newtheorem{conj}[theorem]{Conjecture}
\newtheorem{question}[theorem]{Question}
\newtheorem{corollary}[theorem]{Corollary}
\newtheorem{claim}[theorem]{Claim}

\theoremstyle{definition}
\newtheorem{prob}[theorem]{Problem}
\newtheorem{remark}[theorem]{Remark}
\newtheorem{example}[theorem]{Example}
\newtheorem{definition}[theorem]{Definition}
\newcommand{\pt}[1]{{\partial \over \partial #1}}
\newcommand{\p}[0]{\partial}
\newcommand{\lp}[0]{\left(}
\newcommand{\rp}[0]{\right)}

\newtheoremstyle{special}% name
    {\topsep}%   Space above
    {\topsep}%   Space below
    {\normalshape}%  Body font
    {}%          Indent amount
    {\bfseries}% Theorem head font
    {}%          Punctuation after theorem head -- blank
    {0.5em}%     Space after theorem head (0.5em is the default)
    {{\thmname{#1}\thmnumber{ #2$^{\bm*}\!$.}\thmnote{\ \textmd{(#3)}}}}% Theorem head spec

\theoremstyle{special}
\newtheorem{sprob}[theorem]{Problem}

\newcommand{\todo}[1]{{\color{red}[To Do: #1]}}
\newcommand{\syl}[1]{{\color{orange}[Sylvester: #1]}}
\newcommand{\new}[1]{{#1}}
\newcommand{\sol}[1]{
{\begin{proof}[Solution]#1\end{proof}}
}

\newcommand{\Prob}[1]{\begin{tcolorbox}%[opacityfill=0,height=10cm]
\begin{prob}
	#1
\end{prob}
\end{tcolorbox}	
}
\renewcommand{\vec}{\mathbf}
\DeclareMathOperator{\cur}{curl}
\DeclareMathOperator{\Div}{div}

\title[MATH 2263 Spring 2022]{Math 2263 Problem Sets}



\author[MATH 2263 Spring 2022]{Sylvester W. Zhang}

\thanks{School of Mathematics, University of Minnesota, Minneapolis, MN 55455, USA}
\thanks{\emph{Email}:
\href{mailto:swzhang@umn.edu}{swzhang@umn.edu}}
\thanks{Last updated on \today}
\date{\today}
\begin{document}
\date{Spring 2022}



\maketitle
\setcounter{tocdepth}{1}
\tableofcontents
\setlength{\parindent}{0em}
\setlength{\parskip}{0.618em}

\tableofcontents

\newpage
\section{Vectors and the Three-Dimensional Space}

\Prob{
	Determine if the given three points are co-linear (i.e. lie on one line).%
	\begin{enumerate}
		\item $A=( 2,0,-1) $, $B=(1,-1,-2)$ and $C=(-3,1,0)$
		\item $A=(-1,4,3)$, $B=(-2,4,1)$ and $C=(2,0,1)$
	\end{enumerate}}
\sol{
Three points $A,B,C$ are co-linear if and only if the two vectors $\overrightarrow{AB}$ and $\overrightarrow{BC}$ have the same direction (or equivalently, $\overrightarrow{AB}$ and $\overrightarrow{AC}$, or $\overrightarrow{BC}$ and $\overrightarrow{AC}$). Recall two vectors have the same direction if and only if one is a scalar multiple of another.

(1) We calculate that $\overrightarrow{AB}=B-A=\langle-1,-1,-1\rangle$ and $\overrightarrow {BC}= C-B=\langle -4,2,2\rangle$. $\overrightarrow{AB}$ is not a scalar multiple of $\overrightarrow{BC}$, therefore $A,B,C$ are not co-linear.

(2): Similarly, $\overrightarrow{AB}=B-A=\langle -3,0,-2\rangle$ and $\overrightarrow{BC}=\langle 4,-4,0\rangle$. So $\overrightarrow{AB}$ is not a scalar multiple of $\overrightarrow{BC}$, therefore $A,B,C$ are not co-linear.\qedhere
}
\Prob{
	Describe and find the equation of the set of all points that are equidistant to the two points $A=(-1,5,3)$ and $B=(6,2,-2)$.
}
\sol{It is a plane that is perpendicular to the line $AB$ and contains the middle point of $A$ and $B$.

Algebraically, it has all the points $(x,y,z)$ which satisfies the following equation
\[\sqrt{(x+1)^2+(y-5)^2+(z-3)^2}=\sqrt{(x-6)^2+(y-2)^2+(z+2)^2},\]
namely, the distance to point $A$ (LHS) equals the distance to point $B$ (RHS).

Now we simplify the above equation.
%
\begin{align*}
	(x+1)^2+(y-5)^2+(z-3)^2&=(x-6)^2+(y-2)^2+(z+2)^2\\
	x^2+2x+1+y^2-10y+25+z^2-6y+9&=x^2-12x+36+y^2-4y+4+z^2+4z+4\\
	14 x- 6 y - 10 z - 9&=0
\end{align*}
where we end up with a linear equation, which is plane in $\rr^3$.
}
\Prob{
For each of the vectors given below, find a unit vector that has the same direction.
\[{\bf v}= \langle 2,1,-2\rangle\quad\quad{\bf w}=\langle -4,0,3 \rangle\]
Further, find vectors of length $2$ with the same direction.
}

\sol{To scale a vector $\bf v$ into a unit vector, we simply divide by its magnitude: ${1\over |{\bf v}|}{\bf v}$.

So the unit vector for $\bf v$ is
\[{1\over |{\bf v}|}{\bf v}={1\over \sqrt{2^2+1^2+(-2)^2}}\langle 2,1,-2\rangle={1\over 3}\langle 2,1,-2\rangle=\left\langle {2\over 3},{1\over 3},-{2\over 3}\right\rangle\]
And similarly
\[{1\over |{\bf w}|}{\bf w}={1\over \sqrt{(-4)^2+0^2+3^2}}\langle -4,0,3\rangle={1\over 5}\langle -4,0,3\rangle=\left\langle -{4\over 5},0,{3\over 5}\right\rangle\]
To find the vectors with length $2$, we simply multiply the unit vectors by $2$.
$${2\over |{\bf v}|}{\bf v}=2\left\langle {2\over 3},{1\over 3},-{2\over 3}\right\rangle=\left\langle {4\over 3},{2\over 3},-{4\over 3}\right\rangle$$
\[{2\over |{\bf w}|}{\bf w}=2\left\langle -{4\over 5},0,{3\over 5}\right\rangle=\left\langle -{8\over 5},0,{6\over 5}\right\rangle\qedhere \]
}

\Prob{
	In $\rr^2$, ${\bf v}$ is a unit vector which lies in the first quadrant. Suppose the angle between ${\bf v}$ and the positive $y$-axis is $\pi/4$, find ${\bf v}$ in component form.
}

\sol{We may assume that $\bf v$ starts at the origin.
\begin{center}
	\begin{tikzpicture}[scale=0.5]

\draw[->,ultra thick] (-1,0)--(5,0) node[right]{$x$};
\draw[->,ultra thick] (0,-1)--(0,5) node[above]{$y$};
\draw[->,thick] (0,0)--(4,4)node [right] {$\bf v$};
\draw [dashed](4,0) node [below] {$\sqrt{2}\over 2$ }--(4,4)--(0,4) node [left] {$\sqrt{2}\over 2$ };
\end{tikzpicture}
\end{center}
The ${\bf v}$ forms an angle of $\pi/4=45^\circ$ with the $y$-axis, as depicted in the diagram above. Since the length of ${\bf v}$ is $1$, it follows that the `head' of ${\bf v}$ is $(\sqrt{2}/2,\sqrt{2}/2)$, therefore ${\bf v}=\langle \sqrt{2}/2,\sqrt{2}/2\rangle$.
}
\Prob{
Let ${\bf a}=\langle 2,1,1\rangle$ and ${\bf b}=\langle-1,x,3\rangle$. Find the value of $x$ such that ${\bf a}$ is orthogonal to ${\bf b}$
.}	
\sol{Two vectors are orthogonal if and only if their dot product is zero. Therefore we need to find the $x$ such that
\[\langle 2,1,1\rangle\cdot\langle-1,x,3\rangle=-2+x+3=0\]
Solving for $x$ we get $x=-1$.\qedhere }

\bigskip
\section{Cross Product, Lines and Planes}
\Prob{
Find a non-zero vector that is orthogonal to the plane containing the three points
\[A=(2, -3, 4)\quad B=(-1, -2, 2)\quad C=(3, 1, -3)\]	
}

\sol{
We first calculate the vectors $\overrightarrow{AB}$ and $\overrightarrow{BC}$.
\[\overrightarrow{AB}=B-A=\langle-3,1,-2 \rangle\]
\[\overrightarrow{BC}=C-B=\langle 4,3,-5\rangle \]A vector that is perpendicular to both $\overrightarrow{AB}$ and $\overrightarrow{BC}$ will be perpendicular to the plane of $ABC$. We find such a vector using the cross product.
\begin{align*}&\overrightarrow{AB}\times\overrightarrow{BC}=\left|\begin{matrix}
	{\bf i}&{\bf j}&{\bf k}\\-3&1&-2\\4&3&-5
\end{matrix} \right|={\bf i}\left|\begin{matrix}
	1&-2\\3&-5
\end{matrix} \right|
-{\bf j}\left|\begin{matrix}
	-3&-2\\4&-5
\end{matrix}\right|+{\bf k}\left|
\begin{matrix}
	-3&1\\4&3
\end{matrix}
\right|=\langle 1,-23,-13\rangle\qedhere
\end{align*}
}
\Prob{
Determine whether the following points are co-planer.
\[A=(1,3,2)\quad B=(3,-1,6)\quad C=(5,2,0) \quad D=(3,6,-4)\]
}
\sol{We use the triple product method.
Consider the vectors
\[\overrightarrow{AB}=\langle 2,-4,4\rangle\quad\overrightarrow{AC}=\langle4,-1,-2\rangle \quad\overrightarrow{AD}=\langle2,3,-6\rangle.\]
The four points are coplaner if and only if the volume of the parallelepiped determines by these three vectors is zero.
Said volume is the given by the triple product
\begin{align*}&\overrightarrow{AB}\cdot(\overrightarrow{AC}\times \overrightarrow{AD})\\
=&\overrightarrow{AB}\cdot\left|\begin{matrix}
	i&j&k\\4&-1&-2\\2&3&-6
\end{matrix} \right|\\
=&\langle4,-1,-2\rangle\cdot\langle 12,20 ,14\rangle\\
=&0
  \end{align*}
 Therefore the four points are indeed coplaner.
}
\Prob{Use equations of lines to determine whether the following three points are colinear. 
\[A=(2,4,-3)\quad B=(3,-1,1)\quad C=(1,9,1)\]
\emph{Hint:} Find the equation of the line through $AB$ and check if $C$ is on the line.}
\sol{The equation of a line through two points ${\bf r}_0$ and ${\bf r}_1$ is given by
\[{\bf r}(t)=(1-t){\bf r}_0+t{\bf r}_1\]
We use this to calculate the equation of $\overline{AB}$:
\begin{align*}{\bf r}(t)&=(1-t)\langle2,4,-3\rangle+t\langle 3,-1,1\rangle\\
&=\langle2(1-t)+3t,4(1-t)-t,-3(1-t)+t \rangle\\
&=\langle 2+t,4-5t,-3+4t \rangle\end{align*}
If $C$ is on $\overline{AB}$, then we need to have $2+t=1\implies t=-1$ in order for the first component to match up.
\[{\bf r}(-1)=(1,9,-7)\neq C\]
Therefore $C$ does not lie on the line $\overline{AB}$, hence $A,B$ and $C$ are not co-linear.
}
\Prob{\label{q2.4}Find the equation of the plane through $A=(2,4,-3)$, $B=(3,-1,1)$, and $C=(1,9,1)$.}
\sol{We first calculate the vectors
\(\overrightarrow{AB}=\langle 1,-5,4\rangle\) and \(\overrightarrow{AC}=\langle-1,5,4\rangle\).
Their cross product is
\(\overrightarrow{AB}\times\overrightarrow{AC}=\langle-40,-8,0 \rangle \)
This is a vector that is orthogonal to both $AB$ and $AC$, hence is orthogonal to the plane. Therefore it is a normal vector.
Hence the equation of the plane is
\[-40(x-2)-8(y-4)+0(z+3)=0\]
which can be simplified to
\[5x+y-14=0\qedhere\]
}
\Prob{Find the equation of the line through $(3,2,-4)$ with direction $\langle -1,2,5\rangle$. Find its intersection with the plane from \Cref{q2.4}.}
\sol{
The line has parametric equation
\[{\bf r}(t)=\langle 3-t,2+2t,-4+5t\rangle,\]
and the equation of the plane from previous problem is $5x+y=14$. Substitute the \emph{parametric} equation of the line to the \emph{standard} equation of the plane
\[5(3-t)+(2+2t)=14.\]
Solving for $t$ we get $t=1$. Therefore the intersection is ${\bf r}(1)=(2,4,1)$.\qedhere
}\bigskip
\section{Multivariable Functions, Limits and Partial Derivatives}
\Prob{Find the domains and level curves of the functions \[f(x,y)=\sqrt{4-x^2-y^2}\quad\text{and}\quad f(x,y)=x+\sqrt{y},\]
and sketch their graphs.
}
\sol{\hfill
\begin{enumerate}
	\item The domain for $f(x,y)$ is the points where $4-x^2-y^2\geq 0$, i.e. $x^2+y^2\leq 4$, which is the set of points inside the circle centered at $(0,0)$ with radius $2$ (including boundary).
	
	The level curves are
	\begin{align*}
	f(x,y)=0&\implies x^2+y^2=4\\
	f(x,y)=1&\implies x^2+y^2=3\\
	f(x,y)=2&\implies x^2+y^2=0	
	\end{align*}
	There are no level curves for $L>2$ or $L<0$. (Why?) The level curves are circles. And the graph is a sphere.

	\item We only need $y\geq 0$ for the domain, so it is the upper half of the plane.
	
	The level curves are%
	\begin{align*}
	x+\sqrt{y}=-1&\implies y=(x+1)^2,\ x\leq -1\\
		x+\sqrt{y}=0&\implies y=x^2,\ x\leq 0\\
		x+\sqrt{y}=1&\implies y=(x-1)^2,\ x\leq 1\\
		x+\sqrt{y}=2&\implies y=(x-2)^2,\ x\leq 2
	\end{align*}%
	These are (half) parabolas, so the graph of $f(x,y)$ is a parabolic cylinder.\qedhere

\end{enumerate}
}
\Prob{Find the following limits, or demonstrate if not exists.
\begin{enumerate}
\item 	
\(\displaystyle \lim_{(x,y)\to(2,-1) }{x^2y+xy^2\over x^2-y^2}\)\vspace{0.3cm}
\item \(\displaystyle \lim_{(x,y)\to (0,0)}{xy^3\over x^4+y^4}\vspace{0.3cm} 
\)
\item\(\displaystyle \lim_{(x,y)\to (0,0)} {5y^2\cos^2x\over x^2+y^2 } \)
\end{enumerate}
}
\sol{(1) This is a rational function, which is continuous everywhere in its domain. (Recall that the domain of a rational function is the set of points where the denominator is non-zero.) $(2,-1)$ is in the domain, so the limit is
\[\lim_{(x,y)\to(2,-1) }f(x,y)=f(2,-1)={2^2\cdot(-1)+2\cdot{(-1)}^2\over 2^2-(-1)^2}=-{2\over 3}\]

(2) Taking the limit in the direction of $y=0$, we have
\[\lim_{x\to 0}f(x,0)=\lim_{x\to 0}{x\cdot 0\over x^2+0}=0\]
And taking the limit through $y=x$ we have
\[\lim_{x\to 0}f(x,x)=\lim_{x\to 0}{x\cdot x^3\over x^4+x^4}={1\over 2}\]
Since $0\neq 1/2$, the limit DNE.

(3) With $x=0$, the limit is
\[\lim_{y\to 0}f(0,y)=\lim_{y\to 0}{{5y^2\cos^2(0)\over y^2}}=\lim_{y\to 0}{5y^2\over y^2}=5.\]
For $y=0$, the limit is
\[\lim_{x\to 0}f(x,0)=\lim_{x\to 0}{5\cdot 0\cdot \cos(x)\over x^2}=0\]
Since $0\neq 5$, the limit DNE.\qedhere

}
\Prob{Determine the set of points where the function is continuous.
\begin{enumerate}
	\item \(\displaystyle f(x,y)={2x^2+y\over 1-x^2-y^2}\)\vspace{0.3cm}
	\item \(\displaystyle f(x,y)=\begin{cases}{\displaystyle{2xy\over x^2+y^2+xy}}&\text{if }(x,y)\neq(0,0)\\0&\text{if }(x,y)=(0,0)	
\end{cases}
\)
\end{enumerate}
}
\sol{
(1) The function is a rational function, which is continuous everywhere in its domain. The domain of the function is $\{(x,y)\in\rr^2|1-x^2-y^2\neq 0\}$.

(2) the function $\displaystyle{2xy\over x^2+y^2+xy}$ is continuous whenever the denominator is non-zero. First we show that the denominator $x^2+y^2+xy$ equals $0$ only when $(x,y)=(0,0)$, by solving the equation $x^2+y^2+xy=0$.
\begin{align*}
x^2+y^2+xy&=0\\
4x^2+4y^2+4xy&=0\\
(4x^2+4xy+y^2)+3y^2&=0\\
(2x+y)^2+3y^2&=0	
\end{align*}
Since both $(2x+y)^2$ and $3y^2$ are non-negative, it follows that the solution will satisfy both
\[(2x+y)^2=0 \quad\text{and}\quad 3y^2=0.\]
Clearly then the only solution is $x=0,y=0$. Therefore the rational function $\displaystyle{2xy\over x^2+y^2+xy}$ is not continuous only at $(0,0)$.

Now the function $f(x,y)$ is defined to be $0$ at $(0,0)$. So it would be continuous if
\[\lim_{(x,y)\to(0,0)}{2xy\over x^2+y^2+xy}=0\]
This is false because, the limit with direction $y=0$ is
\[\lim_{x\to 0}{0\over x^2+0+0}=0\]
while the limit with direction $y=x$ is
\[\lim_{x\to 0}{2x^2\over x^2+x^2+x\cdot x}={2\over 3}\neq 0.\]
Therefore the limit DNE, so the function $f(x,y)$ is continuous at $\{(x,y)\in\rr^2|(x,y)\neq (0,0)\}$.
}

\Prob{
Evaluate the following second partial derivatives.
\begin{enumerate}
	\item $\displaystyle{{\partial^2\over \partial x\partial y}\ln(x+y)}$\vspace{0.5cm}
	\item $\displaystyle{{\partial^2\over \partial x\partial y}e^{xy}\sin(x)}$
\end{enumerate}
}
\sol{
(1)
\(\displaystyle
 {\partial \over \partial x}\left({\partial\over\partial y}\ln(x+y)\right)	=\pt x\left({1\over x+y}\right)
 =-{1\over (x+y)^2}
 \)
 
 (2)\begin{align*}
\pt y(e^{xy}\sin x)=\sin x\left(\pt y e^{xy}\right)=\sin x\cdot{\partial e^{xy}\over \partial (xy)}\cdot {\partial xy\over \partial y}=\sin x\cdot e^{xy}\cdot x
\end{align*}

\begin{align*}
&\pt x\left(\pt y e^{xy}\sin x\right)\\
=&\pt xxe^{xy}\sin x\\
=&\sin x\left(\pt x xe^{xy}\right)+xe^{xy}\left(\pt x \sin x\right)\\
=&\sin x\left(e^{xy}+ x\left(\pt x e^{xy}\right)\right)+xe^{xy}\cos x\\
=&\sin x\left(e^{xy}+x\left(e^{xy}y\right)\right)+xe^{xy}\cos x
\qedhere\end{align*}
}\bigskip
\section{Chain Rule and Directional Derivatives}
\Prob{Find $dz/dt$ for $z=\sqrt{xy+1}$, $x=\tan t$ and $y=\arctan(t)$.}
\sol{We use chain rule.
\begin{align*}
{dz\over dt}&={\partial z\over\partial x}{\partial x\over \partial t}+{\partial z\over\partial y}{\partial y\over \partial t}	\\
&=\left(y\over 2\sqrt{xy+1}\right)\cdot\sec^2(t)+\left(x\over 2\sqrt{xy+1}\right)\cdot\left(1\over t^2+1\right)\qedhere
\end{align*}
}
\Prob{Find $\partial u/\partial s$ and $\partial u/\partial t$ for
\[u=ze^{xy}\quad x=s+t\quad y=s-t\quad z=st\]}
\sol{Use chain rule.
\begin{align*}{\p u\over \p s}&=
{\p u\over \p x}{\p x\over \p s}+{\p u\over \p y}{\p y\over \p s}+{\p u\over \p z}{\p z\over\p s}\\
&=\left(yz\cdot e^{xy}\right)\cdot 1+ \left(xz\cdot e^{xy}\right)\cdot 1+e^{xy}\cdot t\\
&=e^{xy}(yz+xz+t)\\
{\p u\over \p t}&=
{\p u\over \p x}{\p x\over \p t}+{\p u\over \p y}{\p y\over \p t}+{\p u\over \p z}{\p z\over\p t}\\
&=\left(yz\cdot e^{xy}\right)\cdot 1+ \left(xz\cdot e^{xy}\right)\cdot (-1)+e^{xy}\cdot s\\
&=e^{xy}(yz-xz+s)\qedhere
\end{align*}
}

\Prob{Find $\partial z/\partial x$ and $\partial z/ \partial y$, where
\[x^2+4y^2+z^2-2z=6\]
}
\sol{We use chain rule and implicit differentiation. The above equation can be written as%
\[F(x,y,z)=x^2+4y^2+z^2-2z-6=0.\]
Therefore, 
\[{\p z\over \p x}=-\left. {\p F\over \p x} \middle/{\p F\over \p z} \right. =-{ 2x\over 2z-2} \]
\[{\p z\over\p y}=-\left.{\p F\over\p y }\middle/{\p F\over\p z}\right.=-{8y\over 2z-2}\qedhere\]
}%
\Prob{For each function $f$, find the gradient $ \nabla f$ and the directional derivative $D_{\bf u}f.$
\begin{enumerate}
\item $f(x,y,z)=x^2z+xyz+yz^2$, ${\bf u}=\langle 1,-1,1 \rangle$.
\item $f(x,y)=e^{x}\sin(xy)$, ${\bf u}=\langle 2,1\rangle$.
\item $f(x,y,z)=xe^y-y^2e^{xz}$, ${\bf u}=\langle -1,0,2\rangle$.
\end{enumerate}
}
\sol{(1) $\displaystyle \nabla f = \left({\p f\over \p x},{\p f\over\p y},{\p f\over\p z}\right)=(2xz+yz,xz+z^2,x^2+xy+2yz)$
We turn ${\bf u}$ into a unit vector by dividing by its magnitude $|{\bf u}|=\sqrt{1^2+(-1)^2+1}=\sqrt{3}$. Then
\[D_{\bf u}f=\nabla f\cdot{{\bf u}\over{|{\bf u}|}}={1\over \sqrt{3}}(2xz+yz-(xz+z^2)+x^2+xy+2yz)\]
	
(2) $\displaystyle\nabla f= \left(e^x (\sin(x y) + y \cos(x y)), e^xx\cos(xy)\right)$ $$D_{\bf u}f={1\over |{\bf u}|}\nabla f\cdot {\bf u}={1\over\sqrt{5} }(2e^x (\sin(x y) + y \cos(x y))+e^xx\cos(xy))$$

(3) $\nabla f=\langle e^y - y^2 z e^{x z}, x e^y - 2 y e^{x z}, -x y^2e^{x z}\rangle $. 
\( D_{\bf u} f={1\over |{\bf u}|}\nabla f\cdot {\bf u}= {1\over \sqrt{5}}(-e^y+y^2ze^{xz}-2xy^2e^{xz})\)
	}


\Prob{Find the maximal rate of change of $f(x,y,z)=xe^y-y^2e^{xz}$ at the point $P(1,0,-1)$. In what direction does that occur?}
\sol{$\nabla f(x,y,z)=\langle e^y - y^2 z e^{x z}, x e^y - 2 y e^{x z}, -x y^2e^{x z}\rangle $. The gradient vector at $P$ is
$\nabla f(1,0,-1)=\langle 1,1,0 \rangle$. 
So the maximal rate of change is $|\nabla f(P)|=|\langle 1,1,0\rangle|=\sqrt{2}$, which happens in the direction of the gradient vector $\langle 1,1,0\rangle $. 
}
\Prob{Find the tangent plane and normal line to $xy^2=2ze^{x+y}+3$ at $(1,-1,-1)$.}
\sol{Let $F(x,y,z)=xy^2-2ze^{x+y}-3$. We first calculate the gradient vector
\[\nabla F(x,y,z)=\langle y^2 - 2 e^{x + y} z, 2 x y - 2 e^{x + y} z, -2 e^{x + y}\rangle\quad\nabla F(1,-1,-1)=\langle 3,0,-2\rangle \]
Then the tangent plane is
\[3(x-1)+0(y+1)-2(z+1)=0\implies 3x-2z-5=0 \]
The normal line is
\[r(t)=\langle 1,-1,-1\rangle +t\langle 3,0,-2\rangle =\langle 1+3t,-1,-1-2t \rangle \qedhere\]
 }
%
\begin{appendices}\section{Additional Problems I}
\Prob{Show that the following limits do not exist.
\begin{enumerate}
	\item $\displaystyle\lim_{(x,y)\to (0,0)}{x\sin y\over y^2}$\vspace{0.3cm}
	\item $\displaystyle\lim_{(x,y)\to(0,0)}{ x^3y^2\over x^6+y^4}$
\end{enumerate}
}
\sol{(1) We find two paths, $x=0$ and $y=x$, which produce different limits as follows. \[\lim_{x=0,y\to 0}{{x\sin y}\over y^2}=\lim_{y\to 0}{0\over y^2}=0\]
\[\lim_{x\to 0,y=x}{{x\sin y}\over y^2}=\lim_{x\to 0}{x\sin x\over x^2}=\lim_{x\to 0}{\sin x\over x}=1\]
(2) Use the two paths $x=0$ (or $y=0$) and $y=x^{3/2}$.
\[\lim_{x=0,y\to 0}f(x,y)=\lim_{y\to 0}{0\cdot y^2\over y^4}=0\]
\[\lim_{x\to 0,y= x^{3/2}}f(x,y)=\lim_{x\to 0}{x^3 (x^{3/2})^2\over x^6+(x^{3/2})^4}=\lim_{x\to 0}{x^6\over x^6+x^6}={1\over 2}\qedhere\]}
\Prob{Find the limit or show that it doesn't exist.
\begin{enumerate}
	\item $\displaystyle \lim_{(x,y)\to(2,1)}{x^2-2xy\over x^2-4y^2}$\vspace{0.3cm}
	\item $\displaystyle\lim_{(x,y)\to (0,1) }{y-1\over x^2+y-1}$\vspace{0.3cm}
	\item $\displaystyle\lim_{(x,y)\to (0,0)}{x^4y+x^2y^2\over 2x^6+y^3}$
\end{enumerate}
}
\sol{(1) The denominator is zero at $(2,1)$, however, since the numerator also vanishes at $(2,1)$, we can factor and simplify the rational function:
\[\lim_{(x,y)\to(2,1)}{x^2-2xy\over x^2-4y^2}= \lim_{(x,y)\to(2,1)}{x(x-2y)\over (x+2y)(x-2y)}= \lim_{(x,y)\to(2,1)}{x\over x+2y}={1\over 2}\]

(2) Along $x=0$ we have
\(\displaystyle \lim_{y\to 1}{y-1\over y-1}=1\). But when $y=1$,\(\displaystyle \lim_{x\to 0}{0\over x^2+0}=0\). So DNE.

(3) Along the path $x=0$ or $y=0$, the limit is zero (verify this). But along the path $y=x^2$, we have
\(\displaystyle \lim_{x\to 0}{x^4x^2+x^2(x^2)^2\over 2x^6+(x^2)^3}=\lim_{x\to 0}{x^6\over 2x^6+x^6}={1\over 3}\)
}
\end{appendices}
\section{Maxima and Minima}
\Prob{Find the local maxima/minima and saddle points of the function.
\[f(x,y)=x^2+y-2xy \quad\text{and}\quad f(x,y)={x^2+y^2\over e^x}\]}
\sol{(1) $f_x(x,y)=2x-2y,f_y(x,y)=1-2x$. So $f_y(x,y)=0\implies 1-2x=0\implies x=1/2$. Then $f_x(x,y)=2x-2y=2{1\over 2}-2y=0\implies x={1/2}$. So the only critical point is $(1/2,1/2)$.
Next we use the second derivative test:
\[f_{xx}=2,f_{yy}=0,f_{xy}=-2\]
Therefore $D(x,y)=f_{xx}f_{yy}-f_{xy}^2=-4$, which is a constant function. So the critical point must be a saddle point.

(2) Taking the partial derivatives
\[f_x(x,y)=-(x^2+y^2-2x)e^{-x}\]
\[f_y(x,y)=2ye^{-x}\]
We first find the critical points, if $f_x(x,y)=2ye^{-x}=0$, then since $e^{-x}\neq 0$, we must have $y=0$. Going from here, we have $f_x(x,y)=-(x^2+0-2x)e^{-x}=0$, which (for the same reason that $e^{-x}=0$) implies that $x^2-2x=0$. Then $x(x-2)=0$, which yields two solutions $x=0$ and $x=2$. Therefore there are two critical points $(2,0)$ and $(0,0)$.

Next we use 2nd derivative test to determine the types of the critical points. We have
\[f_{xx}(x,y)=e^{-x} (2 - 4 x + x^2 + y^2)\]
\[f_{yy}(x,y)=2e^{-x}\]
\[f_{xy}(x,y)=f_{yx}(x,y)=-2e^{-x}y\]
At the point $(0,0)$, we have
\(f_{xx}(0,0)=2>0\), and
\[D(0,0)=f_{xx}(0,0)f_{yy}(0,0)-f_{xy}(0,0)^2=2\times 2-0=4>0 \]
Therefore $(0,0)$ is a local minimum.
For $(2,0)$ we have
\[f_{xx}(2,0)=-2e^{-2}<0\]\[D(2,0)=-2e^{-2}\cdot 2e^{-2}-0<0\]
Therefore it's a saddle point.\qedhere
}
\Prob{Find the shortest distance from the plane $x-2y-z-3=0$ to the origin.}\sol{A point on the plane has the form $(x,y,x-2y-3)$.
Let
\[f(x,y)=\text{distance}^2=x^2+y^2+(x-2y-3)^2\]
And we would like to find the local minimum (if any) of $f$. We first find its critical points. We have
$f_x(x,y)=4x-4y-6$ and $f_y(x,y)=-4x+10y+12$. Thus we have to solve for a $2\times 2$ system of linear equations:
\[\begin{cases}
4x-4y=6&(1)\\4x-10y=12&(2)	
\end{cases}
\]
$eq.(1)-eq.(2)$ gives $6y=-6$, thus $y=-1$. And plug this back in $eq.(1)$ we get $4x=2$, thus $x=1/2$. So the only critical point is $({1\over 2},1)$. 

Now let's check if this indeed is a local minimum.

The second derivatives are
\[f_{xx}(x,y)=4\quad f_{yy}(x,y)=10\quad f_{xy}=-4\]
And 
\[D(x,y)=4\times 10-(-4)^2=24\]
(Note that all the second derivatives are constant functions.) Since $f_{xx}>0$ and $D>0$, the critical point is a local minimum. Therefore, the shortest distance is
\[\sqrt{f\left({1\over 2},-1\right)}=\sqrt{(1/2)^2+(-1)^2+(1/2+2-3)^2}={\sqrt{6}\over 2}\]
}
\Prob{Find the absolute minima of the function $f(x,y)=x^2-4xy+y^2+3y$ in the quadrilateral given by the four points $(0,0)$, $(2,0)$, $(0,3)$ and $(2,3)$.}
\sol{First, we find all the critical points.
\[f_x(x,y)=2x-4y=0\quad f_y(x,y)=2y-4x+3=0\]
This yields one solution: $(1,{1\over 2})$. Second, we examine the values of $f(x,y)$ at the boundary of the region, i.e. the four sides of the quadrilateral.

(i) $y=0,0\leq x\leq 2$. In this case, $f(x,y)|_{y=0}=x^2$
, which is an increasing function of $x$ for $x\in[0,2]$. (What is the vertex of a parabola?) Thus the minimum along this boundary is $f(0,0)=0$.

(ii) If $y=3,0\leq x\leq 2$. In this case, $f(x,y)|_{y=3}=x^2-12x+18$. For $x\in[0,2]$, this is a decreasing function in $x$, thus the minimum is $f(2,3)= -2$.

(iii) If $x=0, 0\leq y\leq 3$. Here we have $f(x,y)|_{x=0}=y^2+3y$, which is increasing for $y\in[0,3]$. Therefore the minimum is $ f(0,0)=0$.

(iv) If $x=2, 0\leq y\leq 3$, we have $f(x,y)|_{x=2}=y^2-5y+4$. The minimum is attained when $y= 5/2$. (Why? Try sketching the graph.) So the minimum is $f(2,5/2)=-9/4$.

Finally we compare the value of critical points and the minima at the boundary:
\[f(1,{1\over 2})= {3\over 4}\quad f(0,0)=0\quad f(2,3)=-2\quad f(2,{5\over 2})=-{9\over 4}\]
Hence the minimum is $-9/4$ which is attained at the boundary with $(x,y)=(2,5/2)$.
}
\Prob{Find the absolute maximum and minimum of the function $f(x,y)=x^2+2xy+y$ in the region bounded by $y=1-x^2$, $y=x-1$, the $y$-axis and $x\geq 0$.}
\newpage
\section{Lagrange Multipliers}
\Prob{Find the extreme values of $f(x,y,z)=e^{xyz}$ with constraint $2x^2 +y^2 +z^2 =24$}
\sol{
Let $g(x,y,z)=2x^2+y^2+z^2$, and we need to solve for $\nabla f(x,y,z)=\lambda \nabla g(x,y,z)$ and $g(x,y,z)=24$.
\[\begin{cases}
yze^{xyz}=4x\lambda &(1)\\
xze^{xyz}=2y\lambda &(2)\\
zye^{xyz}=2z\lambda &(3)\\
	2x^2+y^2+z^2=24&(4)
\end{cases}
 \]
 Take the ratio of equation (1) and equation (2), we get
 \[{yze^{xyz}\over xze^{xyz}}={4x\lambda\over 2y\lambda}\implies {y\over x}={2x\over y}\implies y^2=2x^2\tag{5}\]
 Take the ration of equation (1) and equation (3), we get
 \[{yze^{xyz}\over xye^{xyz}}={4x\lambda\over 2z\lambda}\implies {z\over x}={2x\over z}\implies z^2=2x^2\tag{6}\]
 Now substitute (5) and (6) into (4) we get
 \[2x^2+2x^2+2x^2=24\implies x^2=4\implies x=\pm 2\]
 Plug $x^2=4$ into (5) and (6) we get
 \(y^2=8\) and $z^2=8$, hence $y=\pm \sqrt{8}$ and $z=\pm\sqrt{8}$.
 
 So extreme value is attained at 8 points $(\pm 2,\pm\sqrt{8},\pm\sqrt{8})$. But there are only two extreme values, $f(\pm 2,\pm\sqrt{8},\pm\sqrt{8})=e^{\pm 16}$.
}
\Prob{Find the shortest distance from the plane $x-2y-z-3=0$ to the origin. Problem 5.2 once again, this time use Lagrange multiplier.}
\sol{
Let $(x,y,z)$ be an arbitrary point in the $3$-space, its distance to the origin is $\sqrt{x^2+y^2+z^2}$. Let $f(x,y,z)$ be the square of said distance: $f(x,y,z)=x^2+y^2+z^2$.

We would like to find the extreme (minimum) value of $f(x,y,z)$, when $(x,y,z)$ is on the plane, i.e. with constraint that $x-2y-z-3=0$. So set $g(x,y,z)=x-2y-z$. The system of equations is
\[\begin{cases}
2x=\lambda &(1)\\
2y=-2\lambda &(2)\\
2z=-\lambda &(3)\\
x-2y-z=3&(4)
\end{cases}
\]
Equations (1) to (3) can be rewritten as $x={\lambda\over 2}$, $y=-\lambda$, and $z=-{\lambda\over 2}$. Substitute these to equation (4) we get
\[{\lambda\over 2}-2(-\lambda)-(-{\lambda\over 2})=3\]
which yields $\lambda=1$. Now plug this back in to the equations (1) to (3) we found $x={1\over 2},y=-1$ and $z=-{1\over 2}$. So $({1\over 2},-1,-{1\over 2})$ is the point on the plane that is closest to the origin. Thus the shortest distance is $\sqrt{({1\over 2})^2+(-1)^2+(-{1\over 2})^2}={\sqrt{6}\over 2}$
}

\Prob{Find the extreme value of $f(x,y,z)=x^2+y^2+z^2$ subject to $x-y=1$ and $y^2-z^2=1$.}
\sol{
Set up the system of equations for Lagrange multipliers:
\[\begin{cases}
	2x=\lambda &(1)\\
	2y=-\lambda+2y\mu&(2)\\
	2z=-2z\mu & (3)\\
	x-y=1&(4)\\
	y^2-z^2=1&(5)
\end{cases}\]
First observe that equation (3) can be simplified to $2z(\mu+1)=0$ which has two possible solutions: $\mu=-1$ or $z=0$. We break into two cases.

(i) Suppose $\mu=-1$. Substitute $\mu=-1$ into eq.(2) gives $2y=-\lambda-2y\implies \lambda=-4y$. Combining this with  eq.(4) we get $\lambda=-4(x-1)$. Now use eq. (1) we get $2x=\lambda=-4(x-1)$, which implies $x={2\over 3}$, thus by eq.(4) $y=-{1\over 3}$. Then by eq.(5), $z^2=y^2-1={1\over 9}-1=-{8\over 9}$ which has no real solutions. (But there are two complex solutions $z=\pm {\sqrt{8}\over 3}i$. So in this cases there are two complex solutions of the equations: $(x,y,z)=({2\over 3},-{1\over 3},\pm {\sqrt{8}\over 3}i)$.)

(ii) Now suppose $z=0$. Then by equation (5) we know $y^2=1$ which means $y=\pm 1$. If $y=1$, by eq.(4) we have $x=2$, thus we have $(x,y,z)=(2,1,0)$. In case of $y=-1$, by eq. (4) we have $x=0$, giving the other solution $(x,y,z)=(0,-1,0)$.

Finally, in $\rr^3$ the function $f$ attains extreme value at two points $(2,1,0)$ and $(0,-1,0)$. The extreme values are $f(2,1,0)=2^1+1^2=5 $ (the maximum) and $f(0,-1,0)=1$ (the minimum).
}
\newpage
\section{Basic Double Integrals}
\Prob{Evaluate the following integrals.
\begin{enumerate}
\item $\displaystyle \int_0^\pi \int_0^1 2x+\sin (y)\ dx \ dy$\vspace{0.5cm}
	\item $\displaystyle\int_1^3\int_1^{1\over 3}{\ln y\over xy}\ dy\ dx $\vspace{0.5cm}
	\item $\displaystyle\iint_R {2xy^2\over x^2+1}\ dA$, where $R=[0,1]\times [-3,3]$. (i.e. $0\leq x\leq 1$, $-3\leq y\leq 3$.)
\end{enumerate} }
\sol{
\begin{align*}
	\tag{1}&\int_0^\pi \left(\int_0^1 2x+\sin (y)\ dx\right) \ dy\\
	=&\int_0^\pi\left(\left[{x^2}+x\sin(y)\right]_0^1 \right)\ dy
	=\int_0^\pi \left(1+\sin y\right)\ dy
	=\left[y-\cos(y)\right]_0^\pi=2+\pi\\
	\tag{2}&\int_1^3\int_1^{1\over 3}{\ln y\over xy}\ dy\ dx=\left(\int_1^3{1\over x}\ dx\right)\left(\int_1^{1\over 3}{\ln y\over y}\ dy\right)=\ln 3\cdot \left[ \ln(y)^2\over 2\right]_1^{1\over 3}={\ln(3)^3\over 2} \\
	\tag{3}&\iint_R {2xy^2\over x^2+1}\ dA=\int_0^1\int_{-3}^3 {2xy^2\over x^2+1}\ dy\ dx\\
	=&\int_0^1 {2x\over x^2+1}\ dx\cdot \int_{-3}^3 y^2\ dy= \left[\ln(x^2+1)\right]_0^1\cdot \left[{y^3\over 3}\right]_{-3}^{3}=18\ln(2)\qedhere
\end{align*}
}
\Prob{Fill in the boxes so that the following equality holds
\[\int_{0}^{2}\int_{-1}^{x^2-1} xy \ d y\ d x=\int_{\Box}^{\Box} \int_{\Box}^{\Box} xy\ dx\ dy.\]
Then evaluate the integral using one of the above.}
\sol{The region is given by $D=\{0\leq x\leq 2,-1\leq y\leq x^2-1\}$. We rewrite these inequalities: $y\leq x^2-1\implies y-1\leq x^2\implies \sqrt{y-1}\leq x$. Plug in $x=2$ to $y\leq x^2-1$ we get $y\leq 3$. Thus $D=\{\sqrt{y-1}\leq x\leq 2,-1\leq y\leq 3\}$. Therefore we have
\[\int_{0}^{2}\int_{-1}^{x^2-1} xy \ d y\ d x=\int_{-1}^3 \int_{\sqrt{y+1}}^{2} xy\ dx\ dy.\]
\[\int_{-1}^3 \int_{\sqrt{y+1}}^{2} xy\ dx\ dy=\int_{-1}^3\left[yx^2\over 2\right]_{\sqrt{y+1}}^{2}\ dy=\int_{-1}^3{4y-y(y+1)\over 2}\ dy={4\over 3}\qedhere\]

\qedhere}
\section{More on Double Integrals}
\Prob{Evaluate the following double integrals.
\begin{enumerate}
	\item $\displaystyle\int_0^{\pi\over 2}\int_0^x x\sin y\ dy  \ dx$\vspace{0.2cm}
	\item $\displaystyle\iint_D e^{y^2} \ d A $, where $D=\{(x,y):0\leq y\leq 1,0\leq x\leq y\}$
\end{enumerate}
}
{\sol{
(1) $\displaystyle =\int_0^{\pi/2}\big[-x\cos y\big]_0^x\ dx=\int_0^{\pi/2}(-x\cos x+x)\ dx =\left[-x\sin x-\cos x+{x^2\over 2} \right]_0^{\pi/2}=1-{\pi\over 2}+{\pi^2\over 8}$. (Need to use integration by part for the integrand $x\cos x$.)
\hfill  \hspace{1cm}{ (2)} $\text{(2)}\displaystyle = \int_0^1\int_0^ye^{y^2}\ dx\ dy=\int_0^1\left[xe^{y^2}\right]_0^y\ dy=\int_0^1ye^{y^2}\ dy=\left[e^{y^2}\over 2\right]_0^1 ={e-1\over 2}$
}}
\Prob{Evaluate the following integrals.
\begin{enumerate}
	\item $\displaystyle\iint_D (x^2+2y)\ dA$, where $D$ is bounded by $y=x,y=x^3,x\geq 0$.\vspace{0.2cm}
	\item $\displaystyle\iint_D(2x-y)\ dA$, where $D$ is the circle centered at the origin with radius $2$.
\end{enumerate}}
{\sol{(1) $\displaystyle \int_0^1 \int^x_{x^3}(x^2+2y)\ dy\ dx=\int_0^1\big[x^2y+y^2\big]^x_{x^3} dx=\int_0^1(x^3+x^2-x^5-x^6)\ dx={23\over 84}$.

(2)$\displaystyle\int_{-2}^2\int_{-\sqrt{4-y^2}}^{\sqrt{4-y^2}} (2x-y) \ dx\ dy =\int_{-2}^2\big[x^2-xy\big]_{-\sqrt{4-y^2}}^{\sqrt{4-y^2}}\ dy=\int_{-2}^22y\sqrt{4-y^2}\ dy=0$.}
}
\Prob{Find the volume of the solid bounded by the cylinders $x^2+y^2=r^2$ and $y^2+z^2=r^2$.}
{\sol{First we find the volume above the $xy$-plane.
\begin{align*}
&\int_{-r}^r\int_{-\sqrt{r^2-y^2}}^{\sqrt{r^2-y^2}}\sqrt{r^2-y^2}\ dx\ dy	\\
=&\int_{-r}^r \left[x\sqrt{r^2-y^2}\right]_{-\sqrt{r^2-y^2}}^{\sqrt{r^2-y^2}} \ dy=\int_{-r}^r 2(r^2-y^2)\ dy=2\left[r^2y-{y^3\over 3}\right]_{-r}^{r}={8\over 3}r^3
\end{align*}
Finally by symmetry, we multiply by $2$ to get the volume of the solid, $\displaystyle {16\over 3}r^3$.
}
}
\section{Double Integral with Polar Coordinates}
\Prob{[Problems 8.2 (2)]
Evaluate $\displaystyle\iint_D(2x-y)\ dA$, where $D$ is the circle centered at the origin with radius $2$.
}
\sol{
\begin{align*}&=\int_0^{2\pi} \int_0^2 r(2r\cos(\theta)-r\sin(\theta))\ dr\ d\theta\\&=\int_0^2r^2 \ dr \int_0^{2\pi}(2\cos\theta-\sin\theta)\ d\theta \\
&=\left[r^3\over 3\right]_0^2\cdot\big [2\sin(x)+\cos(x)\big ]_0^{2\pi}={8\over 3}\cdot 0=0 \end{align*}}

\Prob{Find the following integral using polar coordinates.
\[\int_0^a\int_{0}^{\sqrt{a^2-y^2}}xy^2\ dx\ dy\]
}

\sol{
$\displaystyle\int_{0}^{\pi\over 2}\int_0^a r(r\cos(\theta)r^2\sin^2(\theta))\ d r\ d \theta=\left(\int_{0}^{\pi\over 2} \sin^2(\theta)\cos(\theta) \ d\theta\right)\left(\int_0^a r^4  dr\right)$

$\displaystyle \left[\sin^3(\theta)\over 3\right]_{0}^{\pi\over 2}\cdot \left[ r^5\over 5\right]_0^a={1\over 3}\cdot {a^5\over 5}={a^5\over 15}$
}

\Prob{Find the $\displaystyle\iint_R (x^2+y^2)\ dA$ where $R$ is in the first quadrant bounded by $x^2+y^2=1$, $x^2+y^2=9$, $y=x$ and $y=0$.}
\sol{
$\displaystyle
\iint_R (x^2+y^2)\ dA=\int_0^{\pi/4}	\int_1^3 r^2\cdot r\ dr\ d\theta=\int_0^{\pi/4}\left[r^4\over 4\right]^3_1\ d\theta=5\pi
$

}
\newpage
\section{Triple integrals}
\Prob{ Evaluate the integral $\displaystyle \int _0^1\int_y^{2y}\int_0^{x+y}6xy\ dz \ dx \ dy $}
\sol{
\begin{align*}
	=&\int _0^1\int_y^{2y}\big [ 6xyz\big ]_0^{x+y} \ dx \ dy\\=&\int _0^1\int_y^{2y}6xy(x+y) \ dx \ dy\\
	=&\int _0^1\left[6y\left( {x^3\over 3}+{x^2y\over 2}\right) \right]_y^{2y} \ dy\\
	=&\int _0^1 23y^4 \ dy= {23\over 5}
\end{align*}

}
\Prob{Evaluate the integral $\displaystyle\iiint_E e^{z/y}\ dV$, where $E$ is bounded by $E=\{(x,y,z)|0\leq y\leq 1,y\leq x\leq 1,0\leq z\leq xy\}$. }
\sol{
\begin{align*}
&\int_0^1 \int_y^1 \int _0^{xy}e^{z\over y}\ dz\ dx\ dy\\
=&\int_0^1\int_y^1\left[ye^{z\over y}\right]_0^{xy}\ dx\ dy\\
=&\int_0^1\int_y^1(ye^x-y)\ dx\ dy\\
=&\int_0^1 [ye^x-yx]_y^1\ dy\\
=&\int_0^1(ey-y-ye^y+y^2)\ dy\\
=&\left[ {y^3\over 3}+{(e-1)y^2\over 2}-e^y(y-1)\right]_0^1\\
=&\ {e\over 2}- {7\over 6}
\end{align*}

}
\Prob{Evaluate $\displaystyle \iiint_E x^2 \ d V$ where $E$ is the solid bounded by $x^2+y^2=4$, $x+z=2$, and $z=0$. (Hint: You may use the fact that $\int_{0}^{2\pi}\cos^3(\theta)\ d\theta=0$.)}
\sol{We can rewrite the integral as
\(\displaystyle \iint_D\int_{0}^{2-x}x^2\ dz\ dA \),
where $D$ is the the region given by $x^2+y^2=4$ (the circle). Going from here, we have

\centerline{\(\displaystyle\iint_D\big[{x^2z}\big]_0^{2-x}\ dA=\iint_D\big[{x^2z}\big]_0^{2-x}\ dA =\iint_Dx^2(2-x)\ dA\)}

From here we switch to polar coordinates\footnote{Note that this is essentially using cylindrical coordinates (in the next section)}:\begin{align*}&\int_0^{2\pi} \int_0^2 (2r^2\cos^2(\theta) -r^3\cos^3(\theta)) r \ dr\ d\theta
\\
=&\int_0^{2\pi}\left[{2r^4\over 4}\cos^2(\theta)-{r^5\over 5}\cos^3(\theta)\right]_0^2\\
=&\int_0^{2\pi}\left(8\cos^2\theta -{2^5\over 5}\cos^3\theta\right)\ d\theta =\int_0^{2\pi}8\cos^2\theta\ d\theta\\
=&\int_0^{2\pi}8\left({1\over 2}+{1\over 2}\cos(2\theta)\right)\ d\theta\\
=&\ 8\pi+4\int_0^{2\pi}\cos(2\theta)\ d\theta=8\pi\qedhere
\end{align*}
}
\Prob{ Find the volume of the solid bounded by the cylinders $x^2+y^2=r^2$ and $x^2+z^2=r^2$.}
\sol{
\begin{align*}
&\int_{-r}^{r}\int_{-\sqrt{r^2-x^2}}^{\sqrt{r^2-x^2}}\int_{-\sqrt{r^2-x^2}}^{\sqrt{r^2-x^2}}\ dz\ dy\ dx\\
=&	8\int_{0}^{r}\int_{0}^{\sqrt{r^2-x^2}}\int_{0}^{\sqrt{r^2-x^2}}\ dz\ dy\ dx\\
=& 8 \int_0^{r}\int_{0}^{\sqrt{r^2-x^2}}\sqrt{r^2-x^2} \ dy\ dx\\
=&8 \int_0^{r}r^2-x^2 \ dx=8\left[r^2x-{x^3\over 3} \right]^r_0\\=&8\cdot {2\over 3}r^3={16\over 3}r^3\qedhere
\end{align*}
}
\section{Cylindrical, spherical coordinates, and change of variables.}
\Prob{Set up the integral to calculate the volume bounded by the sphere $x^2+y^2+z^2=16$ and the cone $z=\sqrt{3(x^2+y^2)}$ using Cartesian coordinates, cylindrical coordinates and spherical coordinates respectively.}
\sol{}
\Prob{Rewrite the integral $\displaystyle \iiint_E xe^{x^2+y^2+z^2} dV$ where $E$ is the portion of the sphere $x^2+y^2+z^2=1$ in the first octant.}
\sol{}
\Prob{Evaluate $\iint_R(4x+8y)dA$ where $R$ is the parallelogram wit vertices $(-1,3),(1,-3),(3,-1)$ and $(1,5)$. Use the transformation $x={1\over 4}(u+v)$ and $y={1\over 4}(v-3c)$.}
\sol{}

%\begin{appendices}
%\newpage
	%\section{Additional Problems II}
%\end{appendices}
\newpage
\section{Vector Fields and Line Integral}
\Prob{ Find the gradient vector fields of the following functions and sketch them.
\[f(x,y)={1\over 2}(x^2-y^2),\quad f(x,y)=(x+y)^2\]
}
\sol{The gradients are
\[\langle x,y\rangle \quad \langle 2(x+y),2(x+y)\rangle \]
}
\Prob{ Find the gradient vector fields of
\[f(x,y,z)=x^2ye^{y\over z},\quad  f\left( {x,y,z} \right) = {z^2}{{\bf{e}}^{{x^{\,2}} + 4y}} + \ln \left( {\frac{{xy}}{z}} \right)
 \]
}
\sol{
 \[\nabla f=\left\langle{2 e^{y/z} x y, {e^{y/z} x^2 (y + z)\over z}, -{e^{y/z} x^2 y^2\over z^2}}\right\rangle\]
 \[{\nabla f = \left\langle {2x{z^2}{{\bf{e}}^{{x^{\,2}} + 4y}} + \frac{1}{x},4{z^2}{{\bf{e}}^{{x^{\,2}} + 4y}} + \frac{1}{y},2z{{\bf{e}}^{{x^{\,2}} + 4y}} - \frac{1}{z}} \right\rangle }\qedhere\]
 }
 \Prob{ Compute the line integral $\displaystyle\int_C e^x\ dx$ where $C$ is the arc of the curve $x=y^3$ from $(-1,-1)$ to $(1,1)$.}
 \sol{The curve $C$ is parametrized by $r(t)=(x(t),y(t))=(t^3,t)$. The end points are $r(-1)=(-1,-1)$ and $r(1)=(1,1)$. Note that $x(t)=t^3$. Therefore the line integral is
 \[\int_{-1}^1 e^{t^3} x'(t)\ dt=\int_{x(-1)}^{x(1)}e^x \ dx=e^x\big|_{x(-1)}^{x(1)}=e^x\big|_{-1}^{1}=e-{e}^{-1}.\qedhere\]
 }\newpage
 \Prob{Compute the line integral $\displaystyle \int_C y^2z\ d s$ where $C$ is the line segment from $(3,1,2)$ to $(1,2,5)$.}
 
 \sol{
 First we parametrize the line $C$: $r(t)=(1-t)\langle3,1,2\rangle +t\langle 1,2,5\rangle =\langle 3-2t,1+t,2+3t\rangle$. Note that from this parametrization we automatically have $r(0)=(3,1,2)$ and $r(1)=(1,2,5)$ Then
 \[\int_C y^2z\ ds=\int_0^1(1+t)^2(2+3t)\sqrt{(-2)^2+1^2+3^2}\ dt\]
 \[=\sqrt{14}\int_0^1(1+t)^2(3(1+t)-1)\ dt=\sqrt{14}\int_0^1 3t^3+8t^2+7t+2\ dt={107\over 12}\sqrt{14}\qedhere\]
 }
\Prob{Find the line integral $\displaystyle\int_C\textbf{F}\cdot d\mathbf{r}$ where $\mathbf{F}(x,y,z)=(x^2+y)~\mathbf{i}+xz~\mathbf{j}+(y+z)~\mathbf{k}$, and $C$ is given by the function $\mathbf{r}(t)=t^2~\mathbf{i}+t^3~\mathbf{j}-2t~\mathbf{k}$, $0\leq t\leq 2$.}
\sol{
\begin{align*}
	&\int_0^2 \mathbf{F}(\mathbf{r}(t))\cdot \mathbf{r}'(t)\ dt\\
	=&\int_0^2 \langle t^4+t^3,-2t^3,t^3-2t \rangle\cdot \langle 2t,3t^2,-2\rangle\ dt\\
	=&\int_0^2 (4 t - 2 t^3 + 2 t^4 - 4 t^5) \ dt\\
	=&-{538\over 15}\qedhere
\end{align*}
}
\newpage
\section{Conservative vector fields and fundamental theorem of path integrals.}
\Prob{Determine whether or not $\mathbf{F}$ is a conservative vector field, and if so, find the function $f$ such that $\mathbf{F}=\nabla f$.
\begin{enumerate}
	\item $\displaystyle \vec{F}(x,y)=(y^2-2x)\vec{i}+2xy\vec{j}$
	\item $\displaystyle \vec{F}(x,y)=ye^x\vec{i}+(e^x+e^y)\vec{j}$
\end{enumerate}}
\sol{
(1) ${\partial\over \partial y}(y^2-2x)=2y={\partial\over \partial x}2xy$, so $\vec F$ is conservative. First take antiderivative w.r.t. $x$:
\(f(x,y)=\int (y^2-2x)\ \partial x=xy^2-x^2+g(y)\)
Then we take partial derivative w.r.t. $y$: \({\partial\over \partial y}(xy^2-x^2+g(y))=2xy+g'(y)=2xy\).
Therefore $g(y)=C$, so $f(x,y)=xy^2-x^2+C$.

(2) ${\partial\over\partial y}ye^x=e^x={\partial\over\partial x}(e^x+e^y) $, so $\vec F$ is conservative. First taking antiderivative w.r.t. $x$, we have
\(f(x,y)=\int ye^x\ \partial x=ye^x+g(y)\). Then taking partial derivative w.r.t $y$, we get
\({\partial\over\partial y}(ye^x+g(y))=e^x+g'(y)=e^x+e^y\).
So $g'(y)=e^y$, which means that $g(y)=e^y+C$. Thus $f(x,y)=ye^x+e^y+C$.
}
\Prob{Evaluate the following line integrals $\int_C\nabla f\ d\vec r$.
\begin{enumerate}
	\item $f\left( {x,y} \right) = {x^3}\left( {3 - {y^2}} \right) + 4y$ and $C$ is given by $\vec r\left( t \right) = \left\langle {3 - {t^2},5 - t} \right\rangle
$ with $-2\leq t\leq 3$
\item $f\left( {x,y} \right) = y{e^{{x^{\,2}} - 1}} + 4x\sqrt y$ and $C$ is given by $\vec r\left( t \right) = \left\langle {1 - t,2{t^2} - 2t} \right\rangle$ with $0\leq t\leq 2$.
\end{enumerate}}
\sol{
(1) $  \int\limits_{C}{{\nabla f\centerdot d\vec r}} = f\left( {\vec r\left( 3 \right)} \right) - f\left( {\vec r\left( { - 2} \right)} \right) = f\left( { - 6,2} \right) - f\left( { - 1,7} \right) = 224 - 74 = {{150}}
$.\ \ \ 
(2) $ \int\limits_{C}{{\nabla f\centerdot d\vec r}} = f\left( {\vec r\left( 2 \right)} \right) - f\left( {\vec r\left( 0 \right)} \right) = f\left( { - 1,4} \right) - f\left( {1,0} \right) =  - 4 - 0 = {{ - 4}}$.
}

\Prob{ Evaluate $\int_C\vec F \ d\vec r$, where 
\(\vec F(x,y,z)=(y^2z+2xz^2)\vec i+2xyz\vec j+(xy^2+2x^2z)\vec k\)
and $C$ is given by $\langle \sqrt{t},t+1,t^2\rangle$ with $0\leq t\leq 1$.}
\sol{
First we find $f(x,y,z)$ such that $\nabla f=\vec F$. Taking antiderivative w.r.t. $x$ we get:
\(f(x,y,z)=\int (y^2z+2xz^2)\ \partial x=xy^2z+x^2z^2+g(y,z)\). Then take partial derivatives:

(i)\centerline{\({\partial\over \partial y}(xy^2z+x^2z^2+g(y,z))=2xyz+{\partial g(y,z)\over \partial y}=2xyz\)}
(ii)\centerline{\({\partial\over \partial z}(xy^2z+x^2z^2+g(y,z))=xy^2+2x^2z+{\partial g(y,z)\over \partial z}=xy^2+2x^2z\)}

Eq. (i) implies that ${\partial\over\partial y}g(y,z)=0$ and eq. (ii) implies that ${\partial\over\partial z}g(y,z)=0$. Therefore $g(y,z)$ is a constant. So $f(x,y,z)=xy^2z+x^2z^2+C$.

Then apply fundamental theorem of path integrals, we have
\(\int_C\vec F \ d\vec r=f(1,2,1)-f(0,1,0)=(4+1)-0=5\).
}


\newpage
\section{ Green's Theorem}
\Prob{ Evaluate the integral $ \displaystyle \int_C y^4\ dx+2xy^3\ dy$ where $C$ is the ellipse $x^2+2y^2=2$ oriented positively.}
{
\sol{Let $D$ be the region enclosed by $C$, by Green's theorem, we have
\begin{align*}
\int_C 	y^4\ dx+2xy^3\ dy &=\iint_D\left({\partial\  2xy^3\over \partial x}-{\partial \ y^4\over \partial y} \right)\ dA=\iint_D -2y^3\ dA.
\end{align*}
The parametrization for $C$ is $x= \sqrt{2}\cos\theta,y=\sin\theta$, so points in $D$ has the form $$x=r\sqrt{2}\cos\theta,y=r\sin\theta$$ for $0\leq r\leq 1$ and $0\leq \theta\leq 2\pi$. The Jacobian for this change of variable is
\[J=\left|\begin{matrix}
	\sqrt{2}\cos\theta&-r\sqrt{2}\sin\theta\\\sin\theta&r\cos\theta
\end{matrix}\right|=\sqrt{2}r.\]
Thus we have
\[\iint_D-2y^3\ dA=\int_0^{2\pi} \int_0^1-2(r\sin\theta)^3(\sqrt{2}r)\ dr\ d\theta=-2\sqrt{2}\left(\int_0^{2\pi}\sin^3\theta\ d\theta\right)\left(\int_0^1r^4 \ dr\right)\]
Note that $\displaystyle\int_0^{2\pi}\sin^3 \theta\ d\theta=\int_{-\pi}^\pi\sin^3(\theta)\ d\theta=0$, because $\sin^3\theta$ is an odd function. So by substitution, the above integral is $0$.
}
}
\Prob{Evaluate $\int_C\vec F\cdot d\vec r$ where $\vec F=(x^2+y)\ \vec i+(2x-y^2)\ \vec j$ and 
$C$ is a positively oriented circle given by $(x-2)^2+(y-7)^2=4$.
}
{
\sol{
By Green's theorem
\(\int_C \vec F\cdot\ d\vec r=\iint_D\left({\partial\ 2x-y^2\over \partial x}-{\partial\ x^2+y\over\partial y}\right)dA=\iint_D\ dA\)
which is the area of the circle, i.e. $4\pi$.
}
\Prob{Find the area of the polar curve $ r=1-\cos\theta$. (Use calculator.)}
\sol{The curve is parametrized by $x=(1-\cos \theta)\cos \theta$ and $y=(1-\cos\theta)\sin\theta$. By inverse Green's theorem, the area is
\[\int_Cx\ dy=\int_0^{2\pi}(1-\cos\theta)\cos\theta \ dy=\int_0^{2\pi}(1-\cos\theta)\cos(\theta)(\sin^2\theta-\cos^2\theta+\cos\theta)\ d\theta\]
\[=\int_0^{2\pi}(2\cos^4\theta-3\cos^3\theta+\cos\theta)\ d\theta=\int_0^{2\pi}2\cos^4\theta={3\pi\over 2}.\qedhere\]
}
}
\section{Curl and Divergence}
\Prob{Find the curl and divergence of the vector fields.
\begin{enumerate}
	\item $\vec F(x,y,z)=\sin(yz)\ \vec i+\sin(xz)\ \vec j+\sin(xy)\ \vec k$
	\item $\vec F(x,y,z)=xyz^4\ \vec i+x^2z^4\ \vec j+4x^2yz^3\ \vec k$
\end{enumerate}
}{
\sol{(1) $\cur \vec F=
\lp {\partial \sin(xy)\over\partial y}-{\partial \sin(xz)\over \partial z}\rp\vec i+
\lp {\partial \sin(yz)\over\partial z}-{\partial \sin(xy)\over \partial x}\rp\vec j+
\lp {\partial \sin(xz)\over \partial x}-{\partial \sin(yz)\over\partial y}\rp\vec k$ $= x (\cos(x y) - \cos(x z)) \vec i+ y ( \cos(y z)-\cos(x y)) \vec j+ z (\cos(x z) - \cos(y z)) \vec k$, and $  \Div \vec F=0$.

(2) $\cur \vec F=-4xyz^3\ \vec j+xz^4\ \vec k$,\ $\Div\vec F=yz^2(12x^2+z^2)$.
}
}
\Prob{Show that $\vec F= \langle y   e^{x y} + y z + z, x (e^{x y} + z) - z \sin(y z), x y + x - y \sin(y z)\rangle $ is a conservative vector field and find the function $f$ such that $\vec F=\nabla f$.}
\sol{
The first step is to show that $\cur \vec F=0$, and that $\vec F$ has continuous partial derivatives, details of this step is omitted.
	First we take the partial antiderivative w.r.t. $x$:
	\[f(x,y,z)=\int (ye^{xy}+yz+z)\ \partial x=e^{xy}+xyz+xz+g(y,z)\]
	Next we take the partial derivative of $f$ w.r.t. $y$ and $z$:
	\[f_y= xe^{xy}+xz+g_y=xe^{xy}+xz-z\sin(yz)\]
	\[f_z= xy+x+g_z=xy+z-y\sin(yz)\]
	These give us that
	\[\nabla g(y,z)=\langle g_y, g_z\rangle=\langle-z\sin(yz),-y\sin(yz)\rangle\]
	To find $g(x,y)$, we take the partial antiderivative of $g_y$ w.r.t $y$:
	\[g(x,y)=\int -z\sin yz\ \partial y=\cos(yz)+h(z)\]
	Then we take the partial derivative of $g(y,z)$ w.r.t $z$:
	\[g_z=-y\sin(yz)+h'(z)=-y\sin(yz)\]
	Therefore $h'(z)=0$, which means that $h(z)=C$. Thus $g(y,z)=\cos(yz)+C$, and hence $f(x,y,z)=e^{xy}+xyz+xz+\cos(yz)+C$.
}

\newpage
\section{Parametric surface and surface integrals}
\Prob{Find a parametrization for the following surfaces.
\begin{enumerate}
	\item The plane that passes through the point $(0,-1,5)$ and contains the vectors $\langle 2,1,4\rangle$ and $\langle -3,2,1\rangle$.
	\item The part of the ellipsoid $ x^2+4y^2+9z^2=1$ which lies to the left of $xz$-plane.
	\item The parts of the plane $x+2y+z=1$ which lies inside the cylinder $x^2+y^2=1$.
\end{enumerate}
}{
\sol{(1) $\vec r(u,v)=\langle 0,-1,5\rangle+\langle 2,1,4\rangle u+\langle -3,2,1\rangle v=\langle 2u-3v,-1+u+2v,5+4u+v\rangle $

(2) $\vec r(u,v)=\langle\sin(u)\cos(v),{1\over 2}\cos(v),{1\over 3}\sin(u)\sin(v) \rangle $, where $ 0\leq u\leq 2\pi$ and ${\pi\over 2}\leq v\leq {\pi}$.

(3) For the cylinder we need $x=u\sin v$ and $y=u\cos v$ with $0\leq u\leq 1$, $0\leq v\leq 2\pi$. Plug this into the plane equation we get
\(z=1-x-2y=1-u\sin v-2u\cos v\). Therefore the parametrization is
\[\vec r(u,v)=\langle r\cos v,r\sin v,-x-2y=1-u\sin v-2u\cos v\rangle\qedhere\].
}}

\Prob{Find the tangent plane to surfaces $ \vec r(u,v)=(u^2+1)\vec i+(v^3+1)\vec j+(u+v)\vec k$ at $(5,2,3)$.}
{
\sol{
First we find the values of $u,v$ such that $\vec r(u,v)=(5,2,3)$.  We have $v^3+1=2$, which means $v=1$. Then $u+v=3$ implies that $u=2$. So the point is $\vec r(2,1)$.

Next we calculate the tangent vectors: $\vec r_u(u,v)=\langle 2u,0,1\rangle$, hence $\vec r_u(2,1)=\langle 4,0,1\rangle$. Next we have $\vec r_v(u,v)=\langle 0,3v^2,1\rangle$, thus $\vec r_v(2,1)=\langle 0,3,1\rangle $.

Therefore, the tangent plane is parametrized by
\[\langle 5,2,3\rangle +\langle 4,0,1\rangle u+\langle 0,3,1\rangle v=\langle 5+4u,2+3v,3+u+v \rangle \qedhere\]
}
}
\Prob{ Evaluate the surface integral $\displaystyle\iint_{S}(x^2+y^2)\ dS$, where $S$ is given by $\vec r{(u,v)}=\langle 2uv,u^2-v^2,u^2+v^2\rangle  $, $ u^2+v^2\leq 1$.}
{
\sol{First we compute
\(|\vec r_u\times \vec r_v|%=\left|\begin{matrix}\vec i&\vec j&\vec k\\2v&2u&2u\\2u&-2v&2v\end{matrix} \right|
=|\langle 8 u v, 4 u^2 - 4 v^2, -4 u^2 - 4 v^2\rangle |=4\sqrt{2}(u^2+v^2).\)
Use polar coordinates:
\[\iint_S(x^2+y^2)\ d S=\iint_D (4u^2v^2+(u^2-v^2)^2)\cdot 4\sqrt{2}(u^2+v^2)  \ d A =\iint_D4\sqrt{2}(u^2+v^2)^3\ d A\]
\[=\int_0^1 \int_0^{2\pi} 4\sqrt{2}\ r^7\ d\theta\ d r=8\pi\cdot \sqrt{2} \int_0^1r^7\ d r={\sqrt{2}\pi}.\qedhere\]
}
}


\Prob{Find the surface area of part of the sphere $x^2+y^2+z^2=4$ which lies inside the cylinder $x^2+y^2=2x$. }

{
\sol{The projection of the part of the sphere (inside the cylinder) on the $xy$-plane is the circle given by $ x^2+y^2=2x$, which is given by
\[\left\{(r,\theta):0\leq r\leq 2\cos \theta,\ -{\pi\over 2}\leq \theta\leq {\pi\over 2}\right\}\]
in polar coordinates.
%
The upper-half sphere is represented by $z=\sqrt{4-x^2-y^2}$, which can be written as $z=\sqrt{4-r^2}$ in polar coordinates, so we parametrize the upper-half sphere as
\[\vec s(r,\theta) =(r\cos(\theta),r\sin(\theta),\sqrt{4-r^2}).\]
We calculate that $|\vec s_r\times \vec s_\theta |={2r\over \sqrt{4-r^2} } $.
So the surface area of the upper-half sphere inside the cylinder is
\[\iint_D|\vec s_r\times \vec s_t|\ d A=\iint_D{2r\over \sqrt{4-r^2}}\ dA=\int_{-\pi/2}^{\pi/2} \int_0^{2\cos(\theta)}{2r\over \sqrt{4-r^2}}\ dr\ d\theta=2(2\pi-4).\]
\[=\int_{-\pi/2}^{\pi/2}\left[-2\sqrt{4-r^2}\right]_0^{2\cos\theta}\ d\theta=\int_{-\pi/2}^{\pi2}-2\sqrt{\sin^2\theta}+4\ d\theta=\int_{-\pi/2}^{\pi/2}4-|\sin\theta|\ d\theta\]
\[=2\int_0^{\pi/2}4-\sin\theta\ d\theta=4\pi-8\]
Finally multiplying by $2$ we get $4(2\pi-4) $.
}
}

\Prob{Evaluate the surface integral $\displaystyle\iint_S z^2\ dS $ where $S$ is the part of the sphere $x^2+y^2+z^2=1$ which lies inside the cone $ z=\sqrt{x^2+y^2}$.}
{
\sol{The parametrization of the sphere is $$\vec r(u,v)=\langle \cos(u)\sin(v),\sin(u)\sin(v),\cos(v)\rangle\quad 0\leq u\leq 2\pi,0\leq v\leq \pi. $$
We want the part of the sphere under the cone, i.e. satisfy the equation $z\leq \sqrt{x^2+y^2}$.
\[\cos(v)\leq\sqrt{(\cos(u)\sin(v))^2+(\sin(u)\sin(v))^2}=|\sin(v)|,\]which gives ${\pi\over 4}\leq v\leq \pi $. Now back to the integral
\[\iint_S z^2\ dS=\iint_D \cos^2(v)|\vec r_u\times \vec r_v |\ d A=
%\footnote{See Example 10 of section 16.6 of the textbook (P.1117).}
\int_0^{2\pi} \int_{\pi\over 4}^{\pi}\cos^2(v)\sin(v)\ dv\ du\]
\[=2\pi \int_{\pi\over 4}^\pi\cos^2(v)\sin(v)\ dv=2\pi\left[ -{1\over 3}\cos^3(v)\right]_{\pi/4}^\pi ={2+\sqrt{3}\over 6}\qedhere\]
}
}

%\vspace{8cm}

\newpage
\section{Flux integral}
\Prob{Let $S$ be the part of the cone $z=x^2+y^2$ which lies above the region given by $x^2+y^2\leq 1$ and $x\geq 0$. Assuming downward orientation, calculate the surface integral of $\mathbf{ F}=\langle x,y,xy\rangle$ over $S$.}
\sol{The cone is parametrized by $r(x,y)=\langle x,y,x^2+y^2\rangle$.
		The normal vector is.
		
		\[r_x \times r_y = \langle 1,0,2x\rangle\times \langle 0,1,2y\rangle = \langle -2x,-2y,1\rangle\]
		We want the downward orientation, so negate the normal vector:
		$\mathbf n=\langle 2x,2y,-1\rangle$.
		
		\begin{align*}
		\iint_S \mathbf{F}\cdot d\mathbf{S} &= \iint_D \mathbf{F}\cdot \mathbf n\ dS\\
		&=\iint_D\langle x,y,xy\rangle \cdot \langle 2x,2y,-1\rangle \ dA
		\\
		&=\iint_D2x^2+2y^2-xy\ dA\\
		&=\int_{-{\pi\over 2}}^{\pi\over 2}\int_0^1 (2r-r\cos(\theta)r\sin(\theta))r\ dr\ d\theta\\
		&=\int_{-{\pi\over 2}}^{\pi\over 2}\int_0^1 (2r^2-r^3\cos\theta\sin\theta )\ dr\ d\theta\\
		&={\pi\over 2}\qedhere
		\end{align*}}
		
\Prob{ Find $ \iint \vec F\cdot d \vec S$  for $\vec F(x,y,z)=\langle y,-x,2z\rangle$, where $S$ is the hemisphere $x^2+y^2+z^2=4$ ($z\geq 0$) oriented downward.}
\sol{The semisphere is the graph of the function $z=g(x,y)=\sqrt{4-x^2-y^2}$. Thus the integral with upward orientation is
\[\vec F\cdot d \vec S=\iint_D\left((-y)\left(-\frac{x}{\sqrt{4-x^2-y^2}}\right)+x\left(-\frac{y}{\sqrt{4-x^2-y^2}}\right)+2\sqrt{4-x^2-y^2}\right)\ dA\]
\[=2\iint_D\sqrt{4-x^2-y^2} \ dA=2\int_0^{2\pi} \int_0^{2}\sqrt{4-r^2}\ r\ dr\ d\theta=2\cdot 2\pi \cdot \int_4^0-{1\over 2}\sqrt{u}\ du={32\pi\over 3}.\]
Finally, we negate the result, and get $ -{32\pi\over 3}$.}
\Prob{ Evaluate \( \iint\limits_{S}{{\vec F\cdot \,d\vec S}}\) where \(\vec F =  - x\,\vec i + 2y\,\vec j - z\,\vec k\) and \(S\) is the portion of \(y = 2{x^2} + 2{z^2}\) that lies behind \(y = 8\) oriented in the positive \(y\)-axis direction.}
\sol{Set the two equations equal $2x^2+2z^2=8$, we get $x^2+z^2=4$. So $D$ is the circle $x^2+y^2\leq 4$. Write the surface as $f\left( {x,y,z} \right) = 2{x^2} + 2{z^2} - y = 0$, so the normal vector is
\[\vec n={\nabla f\over |\nabla f|}={1\over |\nabla f|}\langle 4x,-1,4z \rangle. \]
We leave the magnitude of $\nabla f$ uncalculated because it will eventually get canceled. Note that we need the normal vector to point at the positive $y$-direction, we by negating it we obtain a normal vector with positive $y$-component: \(\vec n=-{\nabla f\over |\nabla f|}={1\over |\nabla f|}\langle -4x,1-4z \rangle\).
Next
\[\iint_S \vec F\cdot d \vec S=\iint_S\vec F\left( {x,2{x^2} + 2{z^2},z} \right)\cdot \vec n\ d\vec S=\iint_S \left\langle { - x,2\left( {2{x^2} + 2{z^2}} \right), - z} \right\rangle \cdot \frac{{\left\langle { - 4x,1, - 4z} \right\rangle }}{{\left| {\nabla f} \right|}}  d \vec S\]
\[=\iint_S {1\over |\nabla f|}8(x^2+z^2)\ d \vec S=\iint_D8(x^2+z^2)\ dA=\int_0^{2\pi} \int_0^2 8r^3\ dr\ d\theta=64\pi.\qedhere \]}
\newpage
\section{Stokes' theorem and divergence theorem}
\Prob{Use Stokes' Theorem to evaluate \( \iint\limits_{S}{{{\mathop{\rm curl } \nolimits} \vec F\cdot d\vec S}}\) where \(\vec F = y\,\vec i - x\,\vec j + y{x^3}\,\vec k\) and \(S\) is the portion of the sphere of radius 4 with \(z \ge 0\) with upwards orientation.}

{
\sol{By Stoke's theorem, we have $\iint\limits_{S}{{{\mathop{\rm curl}\nolimits} \vec F\centerdot d\vec S}} = \int\limits_{C}{{\vec F\centerdot d\vec r}}$, where $C$ is the intersection of the sphere with the $xy$-plane, i.e. the circle with radius $4$. Thus $C$ is parametrized by
\(\langle 4\cos \theta,4\sin\theta,0\rangle\) where \(0\leq \theta\leq 2\pi \).
We have
$\vec F\left( {\vec r\left( t \right)} \right)\cdot \vec r'\left( t \right)  = \left\langle {4\sin t, - 4\cos t,256\sin t{{\cos }^3}t} \right\rangle \centerdot \left\langle { - 4\sin t,4\cos t,0} \right\rangle   =  - 16({\sin ^2}t+{\cos ^2}t)  =  - 16$. 
Then \[\int_C \vec F\cdot \ d\vec r=\int_0^{2\pi}-16\ d\theta=-32\pi\qedhere\]
}}
\Prob{Use Stokes' theorem to evaluate $\int_C \vec F\cdot\ d\vec r $ where $\vec F(x,y,z)=\langle 1,x+yz,xy-\sqrt{z} \rangle $ and $C$ is the boundary of the plane $3x+2y+z=1$ in the first octant.}
{
\sol{First we calculate the curl of $\vec F$:
\(\cur \vec F=\langle x-y,-y,1\rangle\). By Stokes' theorem, we have
\[\int_C \vec F\cdot\ d\vec r =\iint_S\cur \vec F\cdot\ d S\]
The surface $S$ can be written as a graph of a function $z=g(x,y)=1-3x-2y$, thus
\[\iint_S\cur \vec F\cdot\ d S=\iint_D\left(-(x-y){\partial g\over\partial x}-(-y){\partial g\over\partial y}+1 \right)\ dA=\iint_D(3x-5y+1)\ dA.\]
Next we need to figure our $D$, which is the triangle made from the intersection of the plane and the first quadrant of the $xy$-plane. Set $z=0$, the plane becomes $3x+2y=1$, which has $x$-intercept ${1\over 3}$ and $y$-intercept $1\over 2$. Therefore the integral is
\[\int_0^{1\over 2}\int_{0}^{{1-2y}\over 3}(3x-5y+1)\ dx\ dy={1\over 24}\qedhere\]
}}
\Prob{Use divergence theorem to calculate $\iint_S \vec F\cdot\ d \vec S$ where $\vec F(x,y,z)=\langle 3xy^2,xe^z,z^3\rangle$ and $S$ is the surface bounded by the cylinder $y^2+z^2=1$ and planes $x=-1$ and $x=2$.}

\sol{First we calculate the divergence: $\Div\vec F= 3(y^2+z^2)$. Then by divergence theorem
\[\iint_S\vec F\cdot\ d\vec S=\iint_E3(y^2+z^2)\ dV\]
Use polar coordinates on $yz$-plane, we have
\[=\int_{-1}^2\int_0^1\int_0^{2\pi}3r^3\ d\theta\ dr\ dx=3\cdot 2\pi\cdot {3\over 4}={9\pi\over 2}\qedhere \]
}
\newpage
\begin{thebibliography}{9}
\bibitem{calctextbook}
,James Stewart, Daniel K Clegg, and Saleem Watson, (2020) \emph{Calculus: early transcendentals}, Cengage Learning

\bibitem{lamport94}
Paul Dawkins (2003) \emph{Paul's Online Notes Calculus III}, \url{https://tutorial.math.lamar.edu/GetFile.aspx?file=B,11,N}
\end{thebibliography}
\end{document}%
