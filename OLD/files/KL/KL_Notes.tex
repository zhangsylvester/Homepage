\documentclass[12pt]{amsart}
\usepackage{mathptmx}
\usepackage{stmaryrd}
\usepackage{amssymb}
\usepackage{mathtools}
\newcommand{\ff}[0]{\mathbb{F}}
\newcommand{\rr}[0]{\mathbb{R}}
\newcommand{\cc}[0]{\mathbb{C}}

\DeclareMathOperator{\s}{span}
\numberwithin{equation}{section}
\usepackage{xcolor}
\usepackage{ytableau}
%\include{theorems}

\newcommand{\llb}[0]{\llbracket}
\newcommand{\rrb}[0]{\rrbracket}
\newlength\friezelen 
\settowidth{\friezelen}{$\xi_{m}$} % calculate width of widest element

\usepackage{array} % for "\newcolumntype" macro
\newcolumntype{Q}{>{\centering}p{\friezelen}<{}}


\usepackage{mathtools}
\usepackage{verbatim}
\usepackage{comment}
\usetikzlibrary{arrows}
\tikzset {->-/.style={decoration={markings, mark=at position .5 with {\arrow{latex}}}, postaction={decorate}}}
\tikzset {-->-/.style={decoration={markings, mark=at position .5 with {\arrow[scale=2]{latex}}}, postaction={decorate}}}
\newcommand{\midarrow}{\tikz \draw[-triangle 90] (0,0) -- +(.1,0);}
\newcommand{\miduparrow}{\tikz \draw[-triangle 90] (0,0) -- +(.1,0);}
\newcommand{\midrevarrow}{\tikz \draw[-triangle 90] (0,0) -- +(-.1,0);}
\usepackage{comment}
%\usepackage{graphicx}
\usepackage{color}
\usepackage{etoolbox}
\usepackage[margin=1in]{geometry}
%\usepackage{amsmath,amsthm,amssymb,graphicx,tikz,tikz-cd}
\usepackage{hyperref}
\hypersetup{
    colorlinks = true,
    citecolor = orange,%
}
\usepackage[noabbrev]{cleveref}
\usepackage{subcaption}

\makeatletter
\patchcmd{\@settitle}{\uppercasenonmath\@title}{}{}{}
\patchcmd{\@setauthors}{\MakeUppercase}{}{}{}
\patchcmd{\section}{\scshape}{}{}{}
\makeatother
\makeatletter
\patchcmd{\@maketitle}
  {\ifx\@empty\@dedicatory}
  {\ifx\@empty\@date \else {\vskip2ex %vertical position of date
  \centering\footnotesize\@date\par\vskip1ex}\fi
   \ifx\@empty\@dedicatory}
  {}{}
\patchcmd{\@adminfootnotes}
  {\ifx\@empty\@date\else \@footnotetext{\@setdate}\fi}
  {}{}{}
\makeatother


\renewcommand{\l}[2]{\lambda_{#1,#2}}
\newcommand{\ls}[0]{\prescript{*}{}}
\newcommand{\cal}[1]{\mathcal{#1}}

\theoremstyle{plain}
\newtheorem{theorem}{Theorem}[section]
\newtheorem{restate}{Theorem}[section]
\newtheorem{lemma}[theorem]{Lemma}
\newtheorem{prop}[theorem]{Proposition}
\newtheorem{conj}[theorem]{Conjecture}
\newtheorem{question}[theorem]{Question}
\newtheorem{corollary}[theorem]{Corollary}
\newtheorem{claim}[theorem]{Claim}
\theoremstyle{definition}
\newtheorem{remark}[theorem]{Remark}
\newtheorem{example}[theorem]{Example}
\newtheorem{definition}[theorem]{Definition}
\newtheorem{exercise}[theorem]{Exercise}


\newcommand{\todo}[1]{{\color{red}[To Do: #1]}}
\newcommand{\syl}[1]{{\color{orange}[Sylvester: #1]}}
\newcommand{\new}[1]{{#1}}
\newcommand{\calG}[0]{\mathcal{G}}
\newcommand{\sn}[0]{\mathfrak{S}_n}
\DeclareMathOperator{\mani}{\textbf{Man}}
\DeclareMathOperator{\s}{\textbf{S}}
\DeclareMathOperator{\sgn}{sgn}

\title{Topics in Combinatorics: Kazhdan-Lusztig Theory}
%\author{Sylvester W. Zhang}

\thanks{School of Mathematics, University of Minnesota, Minneapolis, MN 55455, USA}
\thanks{\emph{Email}:
\href{mailto:swzhang@umn.edu}{swzhang@umn.edu}}
\thanks{Last updated: \today}
\begin{document}


\maketitle
\begin{abstract}
	Topics course in combinatorics at University of Minnesota (\href{https://sites.google.com/view/8680spring2023/home}{MATH 8680}) taught by Prof. Pavlo Pylyavskyy.
\end{abstract}
\setcounter{tocdepth}{1}
\tableofcontents
\setlength{\parindent}{0em}
\setlength{\parskip}{0.618em}

\tableofcontents
\section{Introduction}

The \emph{Hecke algebra} $H_W(q)$ associated to a Coxeter group $W$ is, loosely speaking, a $q$-deformation of the group algebra of $W$. It is defined via a set of generators $T_w$ for each $w\in W$, with relations inherited from the Coxeter group $W$. The seminal work of Kazhdan and Lusztig \cite{KL79} showed that, $H_W(q)$ admits a different basis $\{C_w\}$ which better controls the representation theory of $H_W(q)$. 

The Schur Weyl duality between a Weyl group $W$ and a Lie algebra $\mathfrak{g}$, extend to a duality between the Hecke algebra $H_W(q)$ and the $q$-deformed universal enveloping algebra $U_q(\mathfrak{g})$ (quantum group). The KL basis $\{C_w\}$, under the Schur-Weyl duality, is exactly Lusztig's canonical basis for quantum groups. In other words, the KL basis is the ``canonical basis'' for $H_W(q)$. The canonical basis was independently discovered (dually?) by Kashiwara under the name of global basis. Taking modulo $q$, it specializes to the \emph{crystal basis}, which has rich combinatorial properties.

Alternatively, considering vector space dual of the quantum group, we have the \emph{dual canonical basis} of the quantized coordinate ring. To study elements of dual canonical basis, Fomin and Zelevinsky introduced \emph{cluster algebras}\footnote{a combinatorial ``machine'' designed to produce elements of dual canonical basis}, which has now grown to a important area of research of its own.

A \emph{Coxeter system} is a pair $(W,S)$, where $W$ is the \emph{Coxeter group} and $S$ is the set of simple generators, satisfying relations like
\[
s_i^2=1,\quad
(s_is_j)^{m_{ij}}=1	
\]for some positive integers $m_{ij}$. Our main example will be the symmetric group $\sn$. The \emph{reduced expression} of $w\in W$ is the minimal way to write $w$ in terms of simple generators: $w=s_{a_1}\cdots s_{a_l}$. This leads to the notion of \emph{length} of $w$, denoted $\ell(w)$, that is the number of $s_i$'s in the reduced expression of $w$. For example, in $\mathfrak{S}_3$, we have $321=s_1s_2s_1=s_2s_1s_2$, and $\ell(321)=3$.
The subword order on reduced expressions is called the \emph{Bruhat order} on $W$. For example, the Hasse diagram of the Bruhat order of $\mathfrak{S}_3$ is the follows.
\begin{center}
\begin{tikzpicture}
	\node (321) at (0,3) {$s_1s_2s_1$};
	\node (213) at (-1,1) {$s_1$};
	\node (132) at (1,1) {$s_2$};
	\node (123) at (0,0) {$1$};
	\node (231) at (1,2) {$s_2s_1$};
	\node (312) at (-1,2) {$s_1s_2$};
	\draw (321) --(231)--(213) -- (123) --(132)--(312)--(321);
	\draw (213) --(312);
	\draw (132)--(231);
	 \end{tikzpicture}	
\end{center}

The Hecke algebra $H_W(q)$ of a Coxeter group $W$ is generated by a set of generators $\{T_w\}$, for each $w\in W$. They satisfy the following relations.
\[\begin{cases}
T_sT_w =T_{sw}&\text{if }\ell(sw)>\ell(w)\\
T_s^2 = (q-1)T_s+qT_1	
\end{cases}
\]
where $q$ is a formal parameter. It can been seen that $H_W(q)$ is a $q$-deformation of the group algebra of $W$, by setting $q\mapsto 0$.

One may ask what are the inverses of these $T_w$'s. The answer of this question leads to the following theorem/definition.
\begin{theorem}[KL?]There exists a family of polynomials $R_{v,w}(q)$ for every $v,w\in W$, such that
\[(T_w)^{-1}=q\sum_{x\leq w}R_{xw}(q) T_x\]	
These polynomials $R_{v,w}(q)$ are called \emph{the $R$-polynomials}.
\end{theorem}
The Hecke algebra $H_W(q)$ has an automorphism $\eta$ (an involution) defined as follows.
\begin{align*}
q&\mapsto q^{-1}\\	
T_w&\mapsto (T_{w^{-1}})^{-1}
\end{align*}
We want a new basis $\{C_w\}$ for $H_W(q)$ such that 
\begin{enumerate}
	\item $C_w$ is a linear combination of $T_x$ for $x\leq w$.
	\item $\eta(C_w)=C_w$
	\item coefficients of $C_w$ (as in (1)) are ``as simple\footnote{being simple means it's a polynomial whose degree is not too big.} as possible''.
\end{enumerate}

It turns out that the KL basis is the unique one satisfying these properties, which makes it ``canonical''. [Ben asked the geometric meaning of 
canonicity, fill in later.]
\begin{theorem}[KL]
	There exists a unique family polynomials $P_{x,w}(q)$ for each $x\leq w\in W$ such that
	\[C_w'=q^{\cdots} \sum_{x\leq w}P_{x,w}(q)T_x\]
	These polynomials are called \emph{Kazhdan-Lusztig polynomials} and can be computed recursively.
\end{theorem}
%KL polynomials are notoriously hard to compute, and patterns about then are usually hard to believe. 

It is promised that Kazhdan-Lusztig polynomials give rise to beautiful combinatorics. One example is that they break Coxeter groups into ``cells''.

The KL basis defines a preorder\footnote{a preorder is a partial order without antisymmetry. In other words, it is possible that $a\leq b,b\leq a$ while $a\neq b$.} on $W$ as follows. We say that $x\prec_L w$ if any left ideal spanned by KL basis containing $C_w$ also contains $C_x$, called the KL left preorder. Similarly we may define the right preorder $\prec_*$ by looking at right ideals. The Hasse diagram of $\prec_*$ is constructed by drawing an arrow $w\to v$ for $x\prec_{*} w$, and since a preorder doesn't have to be antisymmetric, the graph may contain double arrows. An example of $\mathfrak{S}_3$ is given in \Cref{fg:KL_cells_S3}.

\begin{figure}[h]
	\begin{center}
   \begin{tikzpicture}[scale=0.8]
    \node [] (a123) at (0,-3) {$\color{blue}123$};
    \node [] (a132) at (-1.8,-1.5) {$\color{green}132$};
    \node [] (a213) at (1.8,-1.5) {$\color{red}213$};
    \node [] (a231) at (1.8,0.5) {$\color{red}231$};
    \node [] (a312) at (-1.8,0.5) {$\color{green}312$};
    \node [] (a321) at (0,2) {$\color{orange}321$};
    
    \draw[->] (a321) -- (a312);
    \draw[->] (a321) -- (a231);
    \draw [<->] (a312)--(a132);
    \draw [<->] (a231)--(a213);
    \draw[->] (a132) -- (a123);
    \draw[->] (a213) -- (a123);
    
    \end{tikzpicture} 
    \quad\quad
   \begin{tikzpicture}[scale=0.8]
    \node [] (a123) at (0,-2.5) {\color{blue}\ytableausetup{boxsize=1.2em}\begin{ytableau}
1 & 2 & 3 
\end{ytableau}};
    %\node [color=green] (a132) at (-1.8,-0.9) {$132$};
    %\node [color=red] (a213) at (1.8,-0.9) {$213$};
    \node [] (a231) at (1.8,-0.5) {\color{red}\begin{ytableau}
1 & 3 \\ 2 
\end{ytableau}};
    \node [] (a312) at (-1.8,-0.5) {\color{green}\begin{ytableau}
1 & 2 \\ 3 
\end{ytableau}};
    \node [] (a321) at (0,1.5) {\color{orange}\begin{ytableau}
1 \\2 \\ 3 
\end{ytableau}};
    
%    \draw (a321) -- (a312);
%     \draw (a321) -- (a231);
%   
%        \draw (a132) -- (a123);
%         \draw (a213) -- (a123);
    
    \end{tikzpicture} 

\end{center}
\caption{KL right preorder and cells. The colors represent the (right) cells.}
\label{fg:KL_cells_S3}
\end{figure}

Now if we ignore the single arrows, the `doubly' connected components are called the KL left (right) cells. Each cell induces a representation of $W$, known as the (KL) cell representation. In the case of $\sn$, each cells are indexed by a standard Young tableau $T$, and the corresponding cellular representation is the same as the irreducible representation of $\lambda=\text{shape}(T)$. Moreover, the left (resp. right) cells contain those permutations which have the same recording (resp. insertion) tableaux under the Robinson-Schensted correspondence.



\section{Basics of Coxeter Groups and Hecke Algebras}
\begin{theorem}\label{thm:Hecke}
	
\end{theorem}
The most technical step of proving this theorem if the following lemma, whose proof we defer after proving \Cref{thm:Hecke}
\begin{lemma}\label{lem:lam_rho_commute}
	$\lambda$ and $\rho$'s commute.
\end{lemma}

\begin{lemma}
	$\varphi$ is surjective.
\end{lemma}
\begin{lemma}
	$\varphi$ is injective.
\end{lemma}
\begin{proof}{proof of \Cref{thm:Hecke}}
	Need to check the relations.
	1) $\lambda_s\lambda_w=\lambda_{sw}$ if $sw>w$. This is true by definition.
	
	2) Need to show that $\lambda_s^2 = a_s \lambda_s+b_s\lambda_1$. Now take any basis element of the vector space $\mathcal{E}$, $T_w$, need to show that
	\[\lambda_s^2T_w = a_s\lambda_s T_2 + b_s\lambda_1 T_w\]
	for any $w$.
	If $\ell(sw)>\ell(w)$,
	\[\lambda_s(\lambda_s T_w)=\lambda_s(T_{sw})=a_sT_{sw}+bs T_w=(a_s\lambda_s+b_s\lambda_1)T_w\]
	If $sw<w$, then
	\[\lambda_s^2(T_w)=\lambda_2(a_sT_w+b_sT_{sw})=a_s\lambda_s T_w+b_s\underbrace{\lambda_s T_{sw}}_{=T_{ssw}=T_w}=a_s\lambda_sT_w+b_s\lambda_1 T_w\qedhere\]
\end{proof}
We are left to prove \Cref{lem:lam_rho_commute}, before doing that we need another lemma.

\begin{lemma}
	Let $w\in W$ and $s,t\in S$. If $\ell(swt)=\ell(w)$ and $\ell(sw)=\ell(wt)$, then $sw=wt$.
\end{lemma}
\begin{proof}
	Assume $w=s_1\cdots s_r$ reduced. If $\ell(sw)>\ell(w)$, then $\ell(w)=\ell(swt)<\ell(sw)$, which implies that $sw=w't$, where $w'=w$ or $s s_1\cdots \hat s_i\cdots s_r=w'$.
	
	\[wt=s\cdot swt=s\cdot w't t=sw'=s_1\cdots \hat{s_i}\cdots s_r\implies \ell(wt)\neq \ell(sw)\]
\end{proof}
We are now ready to prove \Cref{lem:lam_rho_commute}.
\begin{proof}[Proof of \Cref{lem:lam_rho_commute}]Consider the following cases.
	
	a) $\ell(w)=\ell(wt)=\ell(sw)<\ell(swt)$
	\todo{Make some pictures to illustrate the cases}
	\[\lambda_s\rho_t(T_w)=\lambda_s(T_{wt})=T_{swt}\]
	\[\rho_t\lambda_s(T_w)=\rho_t(T_{sw})=T_{swt}\]
	b) $\ell(swt)=\ell(w)<\ell(wt)=\ell(sw)$. By lemma $sw=wt$.
	\[\lambda_s\rho_t(T_w)=\lambda_s(T_{wt})=a_sT_{wt}+b_sT_{swt}\]
	\[\rho_t\lambda_s(T_w)=\rho_t(T_{sw})=a_tT_{sw}+b_tT_{swt}\]
	
	c) $\ell(swt)<\ell(sw)=\ell(wt)<\ell(w)$
	\[\lambda_s\rho_t(T_w) \lambda_s(a_t T_w+b_t T_{wt}=a_t a_s T_w+a_t b_s T_{sw} +b_t a_s T_{wt} +b_t b_s T_{sw} \]
	Can check that $\rho_t \lambda_s(T_w)$ gives the same result.
	
	d)$\ell(wt)=\ell(sw)<\ell(w)=\ell(swt)$. By previous lemma, we have $sw=wt$.
	\[ \lambda_s\rho_t(T_w)=\lambda_s(a_tT_w+b_t T_{wt})=a_ta_sT_w+a_tb_sT_{sw}+b_t T_{swt} \]
	\[\rho_t\lambda_t(T_w)=\rho_t(a_s T_w+b_s T_{sw})=a_sa_tT_w+a_sb_tT_{wt}+b_sT_{swt}\]
	
	e) $\ell(wt)<\ell(w)=\ell(swt)<\ell(sw)$.
	\[\lambda_s\rho_t(T_w)=\lambda_s(a_t T_w+b_tT_{wt})=a_t T_{sw}+b_tT_{swt}\]
	\[\rho_t \lambda_s(T_w)=\rho_t(T_{sw})=a_t T_{sw}+b_t T_{swt}=\cdots \cdots\] 
\end{proof}
\todo{Need to finish.}
\section{$R$-polynomials}
From now on we will work in the case of Hecke algebras with equal parameters, i.e. $a_s=q-1,b_s=q$. In other words, out Hecke algebra relations will be
\[\begin{cases}
	T_sT_w=T_{sw}&sw>w\\
	T_s^2=(q-1)T_s+qT_1
\end{cases}\]
Note that $T_s$ is invertible for all $s\in S$:
\[T_s^{-1}=q^{-1}T_s-(1-q^{-1})T_1.\]
Note that this implies that all $T_w$ are invertible. In particular, we can write down the explicit formula for $T_{w^{-1}}^{-1}$, which will be a combination of $T_x$ for $x\leq w$. This leads to the definition of $R$-polynomials.


\begin{theorem}\label{thm:def_R_poly}
Denote $\varepsilon_w =(-1)^{\ell(w)}$ and $q_w=q^{\ell(w)}$. We have
\[\left(T_{w^{-1}}\right)^{-1}=\varepsilon_w q_w^{-1}\sum_{x\leq w}\varepsilon_x R_{x,w}(q),\]
where $R_{x,w}(q)\in\mathbb{Z}[q]$ is a polynomial of degree $\ell(x,w):=\ell(w)-\ell(u)$. Note that $R_{w,w}=1$.
\end{theorem}
\begin{example}If $w=s\in S$, we have
	\[T_{s}^{-1}=-q^{-1}\left( -R_{s,s}(q)T_s+R_{1,s}(q)T_1\right).\]
	This has to be equal to $q^{-1}T_s-(1-q^{-1})T_1$. By matching up coefficients we get
	\[R_{s,s}(q)=1,\quad R_{1,s}(q)=q-1.\]
\end{example}
\begin{example}
	Consider $A_1$ with $2$ generators $S=\{s,t\}$, and let $w=st$.
	\begin{align*}
		T_{(st)^{-1}}^{-1}&=T_{ts}^{-1}=T_s^{-1}T_t^{-1}\\&=\left(q^{-1}T_s-(1-q^{-1})T_1\right)\cdot\left(q^{-1}T_t-(1-q^{-1})T_1\right)\\
		&=q^{-2}\left(T_{st}-(q-1)T_s-(q-1)T_t+(q-1)^2T_1\right)
	\end{align*}
	From here we see that
	\[R_{st,st}=1,\ R_{s,st}=q-1,\ R_{t,st}=q-1,\ R_{1,st}=(q-1)^2\]
\end{example}
\begin{example}
	Similar to the previous but this time we take $w=sts$.
	\begin{align*}
		T_{(sts)^{-1}}^{-1}&=T_s^{-1}T_t^{-1}T_s^{-1}\\
		&=q^{-3}(T_s-(q-1)T_1)\cdot(T_t-(q-1)T_1)\cdot(T_s-(q-1)T_1)\\
		&=q^{-3}\big[T_{sts}-(q-1)T_{ts}-(q-1)T_{st}-(q-1)T_s^2+2(q-1)^2T_s+(q-1)^2T_t-(q-1)^3T_1\big]\\
		&=q^{-3}\big[T_{sts}-(q-1)T_{ts}-(q-1)T_{st}+(q-1)^2 T_s+(q-1)^2T_t-(q^3-2q^2+2q-1)T_1\big]
	\end{align*}
	From this we conclude
	\[R_{sts,sts}=1,\ R_{st,sts}=R_{ts,sts}=q-1\, R_{t,sts}=R_{s,sts}=(q-1)^2,\ R_{1,sts}=q^3-2q^2+2q-1\]
\end{example}
\begin{conj}\label{conj:comb_inv}
	In any Coxeter group $W$, $R_{x,w}(q)$ only depends on the isomorphism type of the Bruhat interval $[x,w]$.
\end{conj}
\begin{remark}
	\Cref{conj:comb_inv} is know to be true in type $A_{n-1}$ [Cite], and in the case when $[x,w]$ has a special matching [cite].
\end{remark}

\begin{lemma}
	Let $s\in S,w\in W$ such that $sw<w$. Assume that $x<w$, then
	\begin{enumerate}[]
		\item if $sx<x$, then $sx<sw$.
		\item if $sx>x$, then $sx\leq w,x\leq sw$
	\end{enumerate}
\end{lemma}
\begin{proof}
	Proof is a little bit painful :( so will be omitted :)
\end{proof}

\begin{proof}[Proof of \Cref{thm:def_R_poly}]
We employ induction on $\ell(w)$.

$w=sv,s\in S$, $\ell(v)<\ell(w)$.
\begin{align*}
	(T_{w^{-1}})^{-1}&=(T_{v^{-1}}T_s)^{-1}=T_s^{-1}(T_{v^{-1}})^{-1}\\
	&=q^{-1}(T_s-(q-1)T_1)\cdot \left[\varepsilon_v q_v^{-1}\sum_{y\leq v}\varepsilon_y R_{y,v}T_y\right]\\
	&=\varepsilon_w q_w^{-1}\bigg[(q-1)\sum_{y\leq v}\varepsilon_y R_{y,v} T_y-\underbrace{\sum_{y\leq v}\varepsilon_y R_{y,v}T_sT_y}_{X}\bigg]\tag{$*$}
\end{align*}
In $X$, there are two possibilities for $y$'s, either $sy>y$ or $sy<y$.
\begin{enumerate}
	\item If $sy>y$, we have terms like
	\[\varepsilon_y R_{y,v}T_{sy}\]
	\item If $sy<y$, we have
	\[(q-1)\varepsilon_yR_{y,v}T_y+q\varepsilon_yR_{y,v}T_{sy}\]
\end{enumerate}

	Therefore ($*$) will be a sum of
	\begin{align*}
		\tag{i}(q-1)\varepsilon_y R_{y,v}T_y\quad\quad&y\leq v,sy>y\\
		\tag{ii}-\varepsilon_yR_{y,v}T_{s,y}\quad\quad&y\leq v,sy>y\\
		\tag{iii}-q\varepsilon_y R_{y,v} T_{sy}\quad\quad&y\leq v,sy<y
	\end{align*}
	Recall the previous lemma that, for $sw<w$, we know that $y<w\implies sy\leq w$. So we have $y\leq v<w$, by lemma $sy\leq w$.
	
	Each $x\leq w$ occurs either as $y\leq v$ or as $sy$ with $y\leq v$, or both. We want the coefficient of $T_x$.
	
	\begin{enumerate}
\item If $x\leq w$, $sx<x$. Only have coefficient in (ii). Since $x=sy$, we know that the coefficient is $-\varepsilon_yR_{y,v}=\varepsilon_x R_{x,w}$. This implies that \begin{equation}R_{x,w}=R_{sx,sw}\end{equation}.
\item If $x<w,x\leq sx$, we have two sub-cases.
\begin{enumerate}
\item If $sx<v$, we get coefficients of $T_x$ from (i) and (iii), which will look like $(q-1)\varepsilon_x R_{x,v}+q\varepsilon_x R_{sx,v}$. Therefore we get
\begin{equation}\label{eq:R_rec_2}
	R_{x,w}=(q-1)R_{x,sw}+qR_{sx,sw}
\end{equation}
\item If $sx\not\leq v$, then we get coefficient $(q-1)\varepsilon_xR_x,v$. But since $R_{u,v}=0$ for $u\leq v$, we still get the same recurrence \Cref{eq:R_rec_2}.
\end{enumerate}
\end{enumerate}
\end{proof}
From the above proof we also obtain a recurrence for $R$-polynomials.
\begin{corollary}\label{thm:R-recurrence}$R$-polynomials can be computed via the following recurrence.
	\begin{enumerate}
		\item $R_{u,v}(q)=0$ if $u\leq v$.
		\item $R_{v,v}(q)=1$.
		\item for $s\in D_L(v)$, we have
		\[R_{u,v}(q)=\begin{cases}
R_{su,sv}&\text{ if }s\in D_L(u)\\
(q-1)R_{u,sv}+qR_{su,sv}&\text{ if }s\notin D_L(u)	
\end{cases}
\]
\end{enumerate}
\end{corollary}
\begin{example}{\color{red}Caution: left-right convention might be wrong in the example.}
	\[R_{123,132}=(q-1)\underbrace{R_{123,123}}_{1}+q\underbrace{R_{132,123}}_{0}=q-1\]
	\[R_{123,231}=(q-1)\underbrace{R_{123,213}}_{q-1}+q\underbrace{R_{132,213}}_{0}=(q-1)^2\]
	\[R_{132,312}=(q-1)\underbrace{R_{132,312}}_{1}+q\underbrace{R_{312,132}}_{0}=q-1\]
	\[R_{123,321}=(q-1)\underbrace{R_{123,312}}_{(q-1)^2}+q\underbrace{R_{132,312}}_{q-1}=(q-1)(q^2-q
	+1)\]
		\end{example}

	



\begin{theorem}
	There exists unique $\tilde R_{u,v}\in \mathbb{Z}[q]$ with positive coefficient such that $R_{u,v}(q)=q^{\ell(u,v)} \tilde{R}(q^{1 \over 2}-q^{-{1\over2}})$.
\end{theorem}
\begin{proof}
	Existence will be proved by induction via recursion.
	
	Say $s\in D_L(v)$. If $s\in D_L(u)$ then $\tilde R_{u,v}=\tilde R_{su,sv}$. If $s\notin D_L(u)$ then,
	\[\tilde R_{u,v}(q^{1\over 2}-q^{-{1\over 2}})=q^{-{\ell(u,v)\over 2}}R_{u,v}=q^{-{\ell(u,v)\over 2}}\bigg[(q-1)q^{\ell(u,sv)\over 2}\tilde R_{u,sv}+q\cdot q^{\ell(su,sv)\over 2}\tilde R_{su,sv}\bigg]\]
	\[=(q^{1\over 2}-q^{-{1\over 2}})\tilde R_{u,sv}+\tilde R_{su,sv}\]
	We then get a recurrence for $\tilde R$-polynomials.
	\begin{equation}
		\tilde R_{u,v}=\begin{cases}
			\tilde R_{su,sv}&\text{ if }s\in D_L(u)\\
			q\tilde R_{u,sv}+\tilde R_{su,v}&\text{ otherwise}.
		\end{cases}
	\end{equation}
	
	From the above recurrence it becomes clear that $\tilde R$-polynomials have positive coefficients.
\end{proof}
\begin{example}
	$\tilde R_{123,132}=q$\quad
	$\tilde R_{123,312}=q^2$\quad
	$\tilde R_{123,321}=q^3+q$
\end{example}
Now the $\tilde R$-polynomials have positive coefficients, can we give a combinatorial interpretation for them?

\begin{definition}
	Let $\xi=(s_1,\cdots,s_r)\in S^r$. A subexpression of $\xi$ is $a=(a_1,\cdots,a_r\in (S\cup \{e\})^r$ such that $a_i\in\{r_i,e\}$. We say $||a||=\#\{i:a_i=r_i\}$. A subexpression $a$ is said to be distinguished if
	\(s_j\notin D_r(a_1\cdots a_{j-1})\) for $2 \leq j \leq r$ such that $a_j = e$. Given $u\in W$, let $D(\xi)_u=\{(a_1,\cdots,a_r)\in D(\xi):a_1\cdots a_r=u\}$.
	\end{definition}
	\begin{example}
		In $\mathfrak{S}_5$, let $\xi=(s_3,s_2,s_1,s_2,s_4)$. $(s_3,-,-,-,-)$ and $(s_3,-,s_1,s_2,-)$ are distinguished subexpressions, while $(s_3,s_2,-,-,-)$ is not.
	\end{example}
\begin{theorem}\label{thm:R-formula}
	\[\tilde R_{(u,v)}(q)=\sum_{\xi \in D(s_1,\cdots,s_r)_u}q^{\ell(v)-||\xi||}\]
	where $v=s_1\cdots s_r$ is a reduced expression of $v$.
\end{theorem}
\begin{example}
	Let $u=1234$ and $v=4321=s_1s_2s_1s_3s_2s_1$.
	$\xi=(s_1,s_2,s_1,s_3,s_2,s_1)$ has the following subexpressions which has the correct product to $u=\text{id}$
	\[101000,000000,020020,001001,120021,100001\]
	Every thing except for the last one is distinguished. Therefore
	\[\tilde R_{1234,4321}=q^6+3q^4+q^2\]
\end{example}
\begin{exercise}
	$R_{u,v}=R_{u^{-1},v^{-1}}$ (Ex 10)
\end{exercise}
\begin{exercise}
	In $\mathfrak{S}_n$ if $u\to v$ in Bruhat graph, then
	\[R_{u,v}=(q-1)(q^2-q+1)^{\ell(u,v)-1\over 2}\]
\end{exercise}
\begin{exercise}
	$(-1)^{\ell(v)-1}[q]R_{e,v}=a(e,v)$ where $a(e,v)$ is the number of atoms of the interval $[e,v]$.
\end{exercise}


\begin{proof}[Proof of \Cref{thm:R-formula}]
	We employ induction on $\ell(v)$. Let $\rho=s_1,\cdots,s_r$ and $s_r=s$.
	\paragraph{Case (1)} Suppose $s\in D_R(u)$. Define a map
	\(\varphi:D(\rho)_u\to D(s_1,\cdots,s_{r-1})_{us}\) by
	\[\varphi((a_1,\cdots,a_r))=(a_1,\cdots,a_r)\]
	We can see that $||\varphi(\xi)||=||\xi||-1$ and that $a_r$ has to be $s_r$. This map $\varphi$ is a bijection.
	Therefore,
	\[\sum_{\xi\in D(\rho)_u} q^{\ell(v)-||\xi||}=\sum_{\eta\in D(s_1,\cdots,s_{r-1})_{us}}q^{\ell(v)-||\eta||-1}=\sum_{\eta\in D(s_1,\cdots,s_{r-1})_{us}}q^{\ell(vs)-||\eta||}=\tilde R_{us,vs}\]
	\paragraph{Case (2)} Suppose $s\notin D_R(u)$. Define
	\[D^+(\rho)_u=\{(a_1,\cdots,a_r)\in D(\rho)_u | a_r=s_r\}\]
	\[D^-(\rho)_u=\{(a_1,\cdots,a_r)\in D(\rho)_u|a_r=e\}\]
	Define a map $\varphi:D(\rho)_u\to D(s_1,\cdots,s_{r-1})_{us}\cup D(s_1,\cdots,s_{r-1})_u$ by
	\[\varphi((a_1,\cdots,a_r))=(a_1,\cdots,a_{r-1})\]
	
	This map is a bijection between $\varphi(D^-(\rho)_u)=D(s_1,\cdots,s_{r-1})_u$ and $\varphi(D^+(\rho)_u)=D(s_1,\cdots,s_{r-1})_{us}$.
	
	Applying this map we get
	\[
	\sum_{\xi\in D(\rho)_u}q^{\ell(u)-||\xi||}=\underbrace{\sum_{\eta\in D(s_1,\cdots,s_{r-1})_{us}}q^{\ell(v)-||\eta||-1}}_{\tilde R_{us,vs}}+\underbrace{\sum_{\eta\in D(s_1,\cdots,s_{r-1})_u}q^{\ell(v)-||\eta||}}_{q\tilde R_{u,vs}}\]
	This completes the proof.
\end{proof}

We will next look at another combinatorial formula by Brenti, which can be used to prove the combinatorial invariance conjecture for intervals like $[\id,w]$.


\subsection{Special Matchings}

\begin{definition}
	A \emph{matching} on a graph is an involution $M:V\to V$ s.t. $(v,M(v))\in E$ for all $v$. Take $G$ to be the Hasse diagram of a poset $P$. A \emph{special matching} is a matching such that for all $x,y\in P$, $x\lessdot y\implies M(x)\leq M(y)$.
\end{definition}
\begin{remark}
	For graded poset $P$, if $x\lessdot y$ and $x\lessdot M(x)$, then $y\lessdot M(y),M(x)\lessdot M(y)$.
\end{remark}
\begin{example}
	
\end{example}
\begin{theorem}\label{thm:sm_exists}
	Let $(W,S)$ be a Coxeter system, $u\leq v\in W$ and $s\in D(v)\setminus D(u)$. Let $M(x)=sx$ for any $x\in [u,v]$. Then $M$ is a special matching.
\end{theorem}
\begin{proof}
	If $w<v$ and $s\in D(w)$, then $sw\leq v$. Therefore $M$ is a matching.

Now take $x\lessdot y\in [u,v]$, want to show that $sx\leq sy$.	

(1) If $s\in D(x)\cap D(y)$, then $y$ has reduced word expression $ss_1\cdots s_k$. We can take a subword to get $x$, which will be $x=s s_1\cdots\hat s_i\cdots s_k$. Therefore $sx=s_1\cdots\hat s_i\cdots s_k$ is a subword of $sy=s_1\cdots s_k$.

(2) If $s\notin D(x)\cup D(y)$.

(3) If $s\in D(x)\setminus D(y)$.

(4) If $s\in D(y)\setminus D(x)$, then $M(x)=y$.

	
\end{proof}
\begin{corollary}
	The interval $[1,v$ always have a special matching.
\end{corollary}
\begin{remark}
	Not all special matching can be obtained as in \Cref{thm:sm_exists}.
\end{remark}
\begin{theorem}
	Let $M$ be a special matching of $[1,v]$. For $u\leq v$, we have%
	\begin{equation}
		R_{u,v} = q^c R_{M(u),M(v)}+ (q^c-1) R_{u,M(v)}
	\end{equation}
	where $c=1$ if $M(u)\gtrdot u$.
 \end{theorem}
 \begin{example}
 \[R_{1,14}=qR_{3,11}+(q-1)R_{1,11}\]
 \[R_{1,11}=q\underbrace{R_{2,8}}_{0}+(q-1)R_{1,8}=(q-1)^3\]
 \[R_{3,11}=q \underbrace{R_{6,8}}_{0} + (q-1)R_{3,8}=(q-1)^2\]
 \[R_{1,14}=q(q-1)^2+(q-1)^4\]
 	\begin{center}
 		\begin{tikzpicture}[scale=1.5]
 			\node (1) at (0,0) {1};
 			\node (2) at (-1,1) {2};
 			\node (3) at (0,1) {3};
 			\node (4) at (1,1) {4};
 			\node (5) at (-2,2) {5};
 			\node (6) at (-1,2) {6};
 			\node (7) at (0,2) {7};
 			\node (8) at (1,2) {8};
 			\node (9) at (2,2) {9};
 			\node (10) at (-1.5,3) {10};
 			\node (11) at (-.5,3) {11};
 			\node (12) at (.5,3) {12};
 			\node (13) at (1.5,3) {13};
 			\node (14) at (0,4) {14};
 			
 			\draw (1)--(2);
 			\draw [double] (1)--(3);
 			\draw (1)--(4);
 			\draw [double](2)--(5);
 			\draw (2)--(6);
 			\draw (2)--(7);
 			\draw (3)--(5);
 			\draw (3)--(6);
 			\draw (3)--(8);
 			\draw (3)--(9);
 			\draw (4)--(7);
 			\draw (4)--(8);
 			\draw (4)--(9);
 			\draw (5)--(10);
 			\draw (5)--(12);
 			\draw (6)--(10);
 			\draw (6)--(11);
 			
 			
 			\draw (10)--(14);
 			\draw (11)--(14);
 			\draw (12)--(14);
 			\draw (13)--(14);
 		\end{tikzpicture}
 	\end{center}
 \end{example}
\begin{exercise}
	Let $P$ be a graded poset, $M$ a special matching, and $x\in P$ such that $M(x)\lessdot x$. Prove that $M$ restrict to special matching of $\{y\in P: y\leq x\}$.
\end{exercise}

\begin{exercise}
	Let $v\in \mathfrak{S}_n$ and $M,M'$ be two special matchings on $[1,v]$ such that $M(u)=M'(u)$ for $\ell(u)\leq 1$. Prove that $M=M'$. (Q: Is this true for other types?)
\end{exercise}

\begin{theorem}
	Diamonds?
\end{theorem}

\section{Kazhdan-Lusztig Polynomials}
\begin{definition}
	Define a map $\eta$ by
	\begin{align*}
	\eta:\mathbb{Z}[q,q^{-1}]&\to\mathbb{Z}[q,q^{-1}]\\
	q&\mapsto q^{-1}
	\end{align*}

Let $\eta(T_w)=\mapsto (T_{w^{-1}})^{-1}$ and extend by linearity to $H(q)$.
\end{definition}
\begin{prop}
	$\eta^2(T_s)=T_s$
\end{prop}
\begin{proof}$\eta^2(T_s)=\eta(q^{-1}T_s-(1-q^{-1})T_1)=T_s-(q-1)T_1-(1-q)T_1=T_s
$
\end{proof}
\begin{theorem}
	$\eta$ is a ring homomorphism.
\end{theorem}
\begin{proof}
	We first prove that $\eta(T_wT_s)=\eta(T_w)\eta(T_s)$.
	
	\paragraph{Case (1) $\ell(sw)>\ell(w)$}
	\[\eta(T_sT_w)=\eta(T_{sw})=T^{-1}_{(sw)^{-1}}=\left(T_{w^{-1}s}\right)^{-1}=\left(T_{w^{-1}}T_s\right)^{-1}=T_s^{-1}T_{w^{-1}}^{-1}=\eta(T_s)\eta(T_w)\]
	\paragraph{Case (2) $\ell(sw)<\ell(w)$} Let $v=w^{-1}s$.
	\begin{align*}
	\eta(T_sT_w)&=\eta(qT_{sw}+(q-1)T_w)=q^{-1}\left(T_{(sw)^{-1}}\right)^{-1}+(q^{-1}-1)T_{w^{-1}}^{-1}=q^{-1}T_v^{-1}+(q^{-1}-1) T_{w^{-1}}^{-1}
	\end{align*}
	\[T_{w^{-1}}=T_{vs}=T_vT_s\implies \left(T_{w^{-1}}\right)^{-1}=T_s^{-1}T_v^{-1}=\left(q^{-1}T_s-(1-q^{-1})T_1\right)T_v^{-1}\]
	Putting these together:
	\begin{align*}\eta(T_sT_s)&=q^{-1}T_v^{-1}+(q^{-1}-1)q^{-1}T_sT_v^{-1}-(q^{-1}-1)(1-q^{-1})T_v^{-1}\\
	&=(-q^{-1}+q^{-2}+1)T_v^{-1}-q^{-2}(q-1)T_sT_v^{-1}\end{align*}
	Now calculate
	\[\eta(T_s)\eta(T_w)=T_s^{-1}(T_{w^{-1}})^{-1}=T_s^{-2}T_v^{-1}\]
	where
	\begin{align*}T_s^{-2}&=(q^{-1}T_s-(1-q^{-1})T_1)^{2}=q^{-2}T_s^2-2q^{-1}(1-q^{-1})T_s+(1-q^{-1})^2T_1\\
	&=q^{-2}((q-1)T_s+qT_1)-2q^{-1}(1-q^{-1})T_s+(1-q^{-1})^{-2}T_1\\
	&=(-q^{-2}(q-1))T_s+(1-q^{-1}+q^{-2})T_1
	\end{align*}
	Comparing we get that $\eta(T_sT_w)=\eta(T_s)\eta(T_w)$.
	
	In general we need to show that $\eta(T_{w'}T_w)=\eta(T_{w'})\eta(T_w)$. We induct on length $\ell(w')$.
	\begin{align*}
		\eta(T_{w'}T_w)&=\eta(T_{w's}T_sT_w)
		=\eta(T_{w's})\eta(T_sT_w)\\&=\eta(T_{w's})\eta(T_s)\eta(T_w)=\eta(T_{w's})T_s)\eta(T_w)=\eta(T_{w'})\eta(T_w)\qedhere
	\end{align*}

	

\end{proof}
We want to find a basis $\{C_w\}$ which is fixed by this involution $\eta$.
\begin{example}
	Recall that $T_s^{-1}=q^{-1}T_s-(1-q^{-1})T_1$, and
	\[\eta(T_s-qT_1)=q^{-1}T_s-(1-q^{-1})T_1-q^{-1}T_S=q^{-1}(T_s-qT_1)\]
	Therefore $\eta(q^{-{1\over 2}}T_s-qT_1)=q^{-{1\over 2}}(T_s-qT_1)$, and we get
	\[ C_s=q^{-{1\over 2}}(T_s-qT_1)\]
\end{example}
The previous example gives us KL basis for $s\in S$. To obtain other $C_w$'s, one may try to simply multiply the $C_s$'s. However this doesn't work because different reduced words might give different product of $C_s$'s.
\begin{example}
	Take $w=sts=tst$. Let's calculate $C_sC_tC_s$, the naive way one would define $C_w$.
	\begin{align*}C_sC_tC_s&=q^{-{3\over 2}}(T_s-qT_1)(T_t-qT_1)(T_s-qT_1)\\
	&=q^{-{3\over 2}}\left(T_{sts}-qT_{st}-qT_{ts}+q(q+1)T_s+q^2T_t-(q^3+q^2)T_1\right)
		\end{align*}
		Note that this is not the same as $C_tC_sC_t$. To fix this, we let $C_w=C_sC_tC_s-C_s$, so that
		\[C_sC_tC_s-C_s=q^{-{3\over 2}}(T_{sts}-qT_{st}-qT_{ts}+q^2T_s+q^2T_t-q^3 T_1)\]
		Note that this equals to $C_tC_sC_t-C_t$.
\end{example}
We want a basis $\{C_w\}$ such that
\begin{itemize}
    \item $\eta(C_w)=C_w$
	\item $C_w$ is a linear combination of $T_x$ for $x\leq w$.
	\item The coefficients of $C_w$'s in $T_x$ are as simple as possible.
\end{itemize}

\begin{theorem}
\label{thm:KL_basis}
	There exist unique $C_w$'s fixed by the involution $\eta$ such that
%
	\[C_w=\varepsilon_wq_w^{1\over 2}\sum_{x\leq w}\varepsilon_xq_x^{-1}\overline{P}_{x,w} T_x\]
	where $\overline{P}=P(q^{-1})$, such that $P_{x,w}\in\mathbb{Z}[q]$ satisfy
	\begin{itemize}
		\item $P_{w,w}=1$
		\item $ \deg(P_{x,w})\leq  {1\over 2} (\ell(w)-\ell(x)-1)$ 
	\end{itemize}
	\end{theorem}
	One important feature of $KL$ polynomials is that they have positive integer coefficients, which was conjectured by Kazhdan and Luztig, and later was proved by Elias and Williamson using Soergel Bimodules to arbitrary Coxeter groups.
	\begin{theorem}
		$P_{x,w}(q)\in \mathbb{Z}_{\geq 0}[q]$ for any Coxeter group.
	\end{theorem}
	It is often more convenient to work with a slightly modified version of the KL basis
	\[C_w'=\varepsilon_w \sigma(C_w)\]
	where $\sigma(q)\mapsto q^{-1}$ and $\sigma(T_w)=\varepsilon_w q_w^{-1} T_w$. These new basis satisfy
	\[C'_w=q_w^{-{1\over 2}}\sum_{x\leq w}P_{x,w}T_x\]
\begin{proof}[Proof of uniqueness]
We will first prove the uniqueness part of \Cref{thm:KL_basis}. Denote $\alpha(x,w)=\varepsilon_w\varepsilon_x q_w^{1\over 2}q_x^{-1}$. Assume%
\begin{enumerate}[(a),noitemsep]
	\item $\eta(C_w)=C_w$
	\item $P_{w,w}=1$
	\item $P_{x,w}(q)\in\mathbb{Z}[q]$ of degree $\leq {1\over 2}(\ell(x,w)-1)$ for $x\leq w$.
\end{enumerate}

where $\ell(x,w)=\ell(w)-\ell(x)$. We will employ induction on $\ell(x,w)$.

\underline{Base Case.} $x=w$ so that $P_{x,w}=1$.

\underline{Induction Step.}	Assume unique for $x<y\leq w$
\[C_w=\sum_{y\leq w}\alpha(y,w)\bar P_{y,w} T_y.\]
Apply $\eta$ we get
\begin{align*}\label{eq:uniq_proof}
C_w&=\sum_{y\leq w}\varepsilon_w\varepsilon_y q^{-{1\over 2} }q_yP_{y,w}(T_{y^{-1}})^{-1}\\&=\sum_{y\leq w} \varepsilon_w\varepsilon_y q^{-{1\over 2} } q_yP_{y,w}\varepsilon_yq_y^{-1}\sum_{x\leq y}\varepsilon_x R_{x,y} T_x\\
&=\varepsilon_xq_w^{-{1\over 2}}\sum_{x\leq y\leq w}\varepsilon_x R_{x,y} P_{y,w} T_x
\end{align*}
Comparing coefficients of $T_x$, we have
\begin{align}
\nonumber\varepsilon_w\varepsilon_xq_w^{1\over 2}q^{-1}\bar P_{x,w}&=\varepsilon_x q_w^{-{1\over 2}}\sum_{x\leq y\leq w}\varepsilon_x R_{x,y} P_{y,w}\\
	q_w^{1\over 2} q_x^{-{1\over 2}}\bar P_{x,w}-q_w^{-{1\over 2}} q_x^{1\over 2} P_{x,w}&=q_w^{-{1\over 2}}q_x^{1\over 2}\sum_{x<y\leq w}R_{x,y}P_{y,w}
\end{align}
By induction hypothesis the right hand side is unique. It can be seen that, assumption (c) implies that
\begin{itemize}
	\item $q_w{1\over 2}q_x^{-{1\over 2}}\bar P_{x,w}$ is a polynomial in $q^{1\over 2}$ without constant term.
	\item $q_w^{-{1\over 2}}q_x^{{1\over 2}} P_{x,w}$ is a polynomial in $-q^{1\over 2}$ without constant term.
\end{itemize}
Therefore there is no cancellation in the left hand side of \Cref{eq:uniq_proof}, thus $P_{x,w}$ can be uniquely recovered from the right hand side.\footnote{Note that in general, the uniqueness of $f(q)-f(1/q)$ does not imply the uniqueness of $f$, without the degree constraint.
}

\end{proof}
\begin{exercise}
	Prove that $P_{x,w}=1$ if $\ell(x,w)=1$.
\end{exercise}

\begin{theorem}
	$P_{x,w}(0)=1$.
\end{theorem}
\begin{proof}Recall that the mobius inversion formula for a poset is
	$$g(y)=\sum_{\hat 0\leq x\leq y}f(x)\iff f(y)=\sum_{\hat 0\leq x\leq y}\mu g(x)$$

\[q^{\ell(x,w)}\bar P_{x,w} = \sum_{y\in[x,w]}R_{x,y}P_{y,w}\]
Recall that $R_{x,y}(0)=(-1)^{\ell(x,w)}$.
So set $q=0$ we get
\[0=\sum_{y\in[x,w]} (-1)^{\ell(x,y)} P_{y,w}(0)\]
Then by Mobius inversion
\[P_{x,w}(0)=-\sum_{x<y\leq w}(-1)^{\ell(x,y)}\cdot 1=-\sum_{x<y\leq w}\mu (x,y)=\mu(x,x)=1\qedhere\]
\end{proof}
\begin{proof}
	Proof by induction on $\ell(w)$
	
	\underline{Base case} $\ell(w)\leq 2$
	
	\underline{Induction Step} Find $s$ s.t. $sw<w$. Set $v= sw$.
	
	Define $C_w= C_sC_v-\sum_{z<v,sz<z}\mu(z,v)C_z$.
	
	It's clear that $\eta(C_w)=C_w$ and $C_w $ is a linear combination of $T_x$ for $x\leq w$. We need to know what is the coefficient of $T_x$.
	
	Case (1). $x<sx$.
	
	\[T_s T_{sx} = qT_x+(q-1)T_{sx} \]
	This implies that coefficient of $T_x$ in $q^{-{1\over 2}} T_s C_v$ is $q^{-{1\over 2}}q \alpha(sx,v)\bar P_{sx,v}$.
	And the coefficient of $T_x$ in $q^{-{1\over 2}} T_1 C_v$ is $-q^{1\over 2}\alpha(x,v)\bar P_{x,v}$.
	And summing up, the coefficient of $T_x$ in $C_sC_v$ is
	\[q^{-1}\alpha(x,w)\bar P_{sx,v} + \alpha(x,w)\bar P_{x,w}\]
	
	Case (2). $x>sx$.
	
	Coefficient of $T_x$ is $C_sC_v$ is
	\[\alpha(x,w)\bar P_{sx,v}+q^{-1}\alpha(x,w)\bar P_{x,v}\]
	Therefore
	\[P_{x,w}=q^{1-c}\underbrace{P_{sx,v}}_{\deg\leq {1\over 2}(\ell(sx,v)-1)}+q^cP_{x,v}-\sum_{x<v}\mu(z,v) q_z^{-{1\over 2}}q_w^{1\over 2} P_{x,z}\]
	where $c=0$ if $sx>x$ and $c=1$ if $sx<x$.
	
	We want that $\deg(P_{x,w})\leq {1\over 2}(\ell(x,w)-1)$.
	
	what if $x=z$? impossible. [??]
	
	If $c=0$, 
	%
	\[\leq {1\over 2}(\ell(x,z)-1)+{1\over 2}\ell(z,w)={1\over 2}(\ell(x,w)-1)\]	
	
	If $c=1$,
	\[\deg(q^1 P_{x,v})\leq 1+{q\over 2}(\ell(x,w)-2)={1\over 2}\ell(x,w)\]	
	
\end{proof}
[Fill In Feb 20th]
\subsection{Second formula for KL polynomials}
Let $s\in D_L(w)$ and 
$c=\begin{cases}
	1&s\in D_L(x)\\0&s\notin D_L(x)
\end{cases}$, then
\[P_{x,w}=q^{1-c}P_{sx,sw}+q^cP_{x,sw}-\sum_{z<sw}\mu(z,sw)q^{\ell(z,w)}P_{x,z}\]
where $\mu(z,v)$ is the $\mu$-coefficient of $P_{z,v}$.

\begin{corollary}
	For $s\in D_L(w)$, we have $P_{x,w}(q)=P_{sx,w}(q)$.
\end{corollary}
\begin{corollary}
	For any finite Coxeter group, $P_{u,w_0}=1$. 
\end{corollary}
\begin{corollary}
	If $\mu(z,w)\neq 0$ and $\ell(z,w)>1$, then $D_L(w)\subset D_L(u)$.
\end{corollary}
\begin{proof}
	Assume not, then there exists an $s$ such that $s\in D_L(w)\setminus D_L(u)$. then
	\[P_{z,w}=\mu(z,w)q^{{\ell(w,z)-1\over 2}}+\cdots.\]
	At the same time $P_{sz,w}=P_{z,w}$, which violates the degree conditions.
\end{proof}
\section{Billey-Warrington Theorem}
\begin{theorem}
$C'_w$ is tight if and only if $w$ avoids $321$ and
	\[46718235,46781235,56718234,56781234.\]
\end{theorem}


\begin{definition}[Heap]
	Represent $s_i$ by 
\end{definition}

\begin{lemma}[BJS]
	A permutation $w$ is $321$-avoiding if and only if no reduced word for $w$ contains consecutive subword $\cdots s_is_{i\pm 1}s_i\cdots$.
\end{lemma}
\begin{proof}
	It's easy to see that, if a permutation contains $s_is_{i\pm 1}s_i$ as a subword, then it contains $321$. This proves one direction of the lemma.
	
	For the other direction, assume $w=\cdots c\cdots b\cdots a\cdots$ where $a<b<c$. Then use some wiring diagram argument.
\end{proof}

\begin{lemma}
	$w$ is $321$-avoiding if and only if any two occurrences of $s_i$ in a reduced word of $w$ are separated by both $s_{i-1}$ and $s_{i+1}$.
\end{lemma}
\begin{proof}
	Assume $w$ is $321$-avoiding. Choose $\cdots s_i\cdots s_i\cdots$ as close as possible. Then one $s_{i\pm 1}$ must separate them: $\cdots s_i\cdots s_{i\pm 1}\cdots s_i\cdots$.
\end{proof}
Let $\xi=s_{i_1}\cdots s_{i_l}$ be an expression and $a_{i_1}\cdots a_{i_l}$ where $a_{i_j}\in\{s_{i_j},1\}$ a subexpression.

Recall that $j$ is a defect if $a_{i_1}\cdots a_{i_{j-1}}$ has right descent $s_{i_j}$. Let $P_x(\xi)$ denote the set of all subexpression of $\xi$ with product $x$, and $D(a)$ to be the defects of $a$.

\begin{example}
	Let $\xi=s_3s_2s_1s_4s_3s_2s_5s_4s_3$, and $x=s_1s_3s_5$.
	
	\begin{tabular}{|c|c|}
	\hline
	$P_x(\xi)$&$D(a)$\\
	\hline
	$--s_1---s_5-s_3$&$\emptyset$\\
	$--s_1-s_3-s_5--$&9\\
	$s_3-s_2---s_5--$&5,9\\
	$s_3-s_1-s_3-s_5-s_3$&5\\
	\hline
	\end{tabular}
\end{example}
\begin{theorem}[Deodhar] Coefficients of $T_x$ in $C'_{s_i}\cdots C'_{s_l}$ is given by $P_{x,\xi}^*=\sum_{a\in P_x(\xi)} q^{|D(a)|}$.\footnote{Note that $P_{x,\xi}$ is not exactly the Kazhdan-Lusztig polynomial. However we should think about it as someone who wants to e KL polynomial, and in many cases, it is.}
\end{theorem}
\begin{example}
	$\xi=s_1s_2s_1$. 
	$$(T_{s_1}+1)(T_{s_2}+1)(T_{s_1}+1)=T_{s_1s_2s_1}+T_{s_1s_2}+T_{s_2s_1} +(q+1)T_{s_1}+T_{s_2}+(q+1)
	$$
\end{example}
If $w_0$ is the largest element in a parabolic subgroup $S_{[i,j]}$, e.g. $143256$ in $[2,4]$.

\begin{prop}
	$C'_{w_0}=\sum_{x\leq w_0}T_x$
\end{prop}
\begin{proof}
	$P_{x,w} = P_{sx,w}$ if $s$ is left descent of $w$ but not of $x$.
\end{proof}

What if instead of factoring a product of $C'$ into $C'_s$'s, factor into $C'_{w_0}$'s where $w_0$ is longest in a certain parabolic subgroup.

Denote $C'_{[i,j]}=\sum_{x\text{ is a permutation on }[i,j]}T_x$.
We have 
\[C'_{s_1s_2} = C'_{s_1}C'_{s_2}\iff C'_{s_1s_2}=C'_{[1,2]}C'_{[2,3]}\]
\begin{example}
	\[C'_{[1,2]}C'_{[2,4]}=(T_{s_1}+1)(T_{s_2s_3s_2}+T_{s_2s_3}+T_{s_3s_2}+T_{s_3}+T_{s_2}+1)=C'_{4132}\]
\end{example}
\begin{example}
	Consider $C'_{[1,3]}C'_{[2,4]}$. From the picture, the longest permutation we can get is $4312$.
	$$(1+q)C'_{4312}=C'_{[1,3]}C'_{[2,4]}$$
\end{example}
Avoiding $3412$ and $4231$ means that the Schubert variety is smooth, and thus Kazhdan-Lusztig polynomial equals to $1$, in which case we can obtain a factorization formula for $C'$ easily. Let's look at when we don't avoid them.
\begin{example} By sorting, the candidate for $3412$ is
	 $C'_{3412}=C'_{[2,3]}C'_{[1,2]}C'_{[3,2]}C'_{[2,3]}$, which is true by Billey-Warrington.
	 
	 For 4231, there are two candidates $C'_{[1,2]}C'_{[2,4]}C'_{[1,2]}$ or $C'_{[3,4]}C'_{[1,3]}C'_{[3,4]}$. It turns out that both are correct.
	 
	 The product on the right hand side is
	 \[(T_{s_1}+1)(T_{s_2s_3s_2}+\cdots+1)(T_{s_1}+1).\]
	 Only need to look at the ``non-reduced part'', which gives $(q+1)(T_{s_1+1})+(q+1)(T_{s_1}+1)T_{s_3}$. Can be checked by length analysis.
\end{example}
\begin{example}
	In $S_5$, $C'_{45312}$ doesn't work.
\end{example}
\begin{theorem}[Agrawal-Sotirov]
	Product of $C'_{[i,j]}$-s is equal to sum of flows (through corresponding heaps) $q^{\#\text{defects}}$
\end{theorem}

\begin{theorem}
	If $k_1,\cdots,k_l$ are ``overlap'' sizes in a heap. We can factor $\prod[k_i]_q!$. If degree of $q$ in the result satisfies $\leq {1\over 2}(\ell(w)-\ell(x)-1)$ condition, the result is $C'_w$.
\end{theorem}
\input{mar15.tex}
\begin{theorem}
	If $w$ avoids $3412$ and $4231$ then $w$ has zig-zag factorization.
\end{theorem}

Question: When does $C'_w$ factor into $C'_{[i,j]}$?
\begin{conj}[Agrawal-Sotiror]
	If $w$ avoids $45312$, $456123$. Note $456123$-avoiding implies hexagon-avoiding.
\end{conj}

\begin{theorem}
	If $w$ avoids $4231$, $45312$, $45123$, $34512$. Then $C'_w$ factors.
\end{theorem}
\section{Cell Theory}

Recall for $C_w$ the Kazhdan-Lusztig basis, we have
\begin{equation}\label{eq:Cw_factor}
C_w=C_sC_v-\sum_{\substack{z<v\\ sz<z}}\mu(z,v) C_z\end{equation}
where $v=sw<w$ and $\mu(z,v)$ the $\mu$-coefficient of $P_{z,v}$.

\begin{theorem}
	a) If $sw<w$, then $T_s C_w=-C_w$.
	
	b) If $sw>w$, then 
	\[T_sC_w = q C_w + q^{1\over 2}C_{sw} + q^{1\over 2}\sum_{ \substack{z<w\\ sz<z}}\mu(z,w) C_z  \]
\end{theorem}
\begin{proof}
 b) Plug
	$C_s=q^{-{1\over 2}}T_s -q^{1\over 2} T_1$ into \Cref{eq:Cw_factor}.
	\begin{align*}
		C_{sw}&=C_s C_w - \sum_{\substack{z<w,\; sz<z}}\mu(z,v) C_z\\
		C_{sw}+\sum_{\substack{z<w,\; sz<z}}\mu(z,v) C_z&=(q^{-{1\over 2}}T_s -q^{1\over 2} T_1)C_w\\
		C_{sw}+\sum_{\substack{z<w,\; sz<z}}\mu(z,v) C_z+q^{1\over 2}C_w&=q^{-{1\over 2}}C_w\tag{*}
	\end{align*}
	
	a) First assume $w=s$.
	\[T_sC_s=q^{-{1\over 2}}T_s^2 - q^{1\over 2}T_s = -q^{1\over 2}T_s + q^{1\over 2}=-C_s\]
	
	Now assume $\ell(w)>1$. By b), we have
	\begin{align*}
	T_s C_{sw}&=qC_{sw}+q^{1\over 2} C_w + q^{1\over 2}\sum \mu(z,sw)C_z\\
	C_w	&= q^{-{1\over 2}} T_s C_{sw} - q^{1\over 2} C_{sw} - \sum_{z<w,\; sz<z}\mu(z,sw) C_z
	\end{align*}
	
	$sz<z<w$
	
	By induction, $T_sC_z=-C_z$, we have
	\[T_s C_w = q^{-{1\over 2}}T_s^2 C_{sw}-q^{1\over 2}T_s C_{sw}+\sum\mu(z,sw)C_z=-C_w\]

	
	
	
\end{proof}
\subsection{KL Graph}
We want to make a graph whose vertices are elements of $W$, and there is an arrow $u \xrightarrow{s} w$ colored by $s$, whenever $C_u$ is a summand in $T_s C_w = \sum C_u$.

\begin{definition}
	The colored KL graph is a directed graph $\bar \Gamma$ whose vertices are elements of $W$, directed edges $x \xrightarrow{\bar \mu}_{s} y$ are of two types.
	
	(i) $x\neq y$, $s\in D_L(x)\setminus D_L(y)$, and $\bar \mu(x,y)=\max\{\mu(x,y),\mu(y,x)\}$
\end{definition}

\bibliographystyle{alpha}
\bibliography{main}
\end{document}


\end{document}